%	This is written by Zhiyang Ong to document algorithms that I have implemented for my C++ -based boilerplate code repository.

%	The MIT License (MIT)

%	Copyright (c) <2014> <Zhiyang Ong>

%	Permission is hereby granted, free of charge, to any person obtaining a copy of this software and associated documentation files (the "Software"), to deal in the Software without restriction, including without limitation the rights to use, copy, modify, merge, publish, distribute, sublicense, and/or sell copies of the Software, and to permit persons to whom the Software is furnished to do so, subject to the following conditions:

%	The above copyright notice and this permission notice shall be included in all copies or substantial portions of the Software.

%	THE SOFTWARE IS PROVIDED "AS IS", WITHOUT WARRANTY OF ANY KIND, EXPRESS OR IMPLIED, INCLUDING BUT NOT LIMITED TO THE WARRANTIES OF MERCHANTABILITY, FITNESS FOR A PARTICULAR PURPOSE AND NONINFRINGEMENT. IN NO EVENT SHALL THE AUTHORS OR COPYRIGHT HOLDERS BE LIABLE FOR ANY CLAIM, DAMAGES OR OTHER LIABILITY, WHETHER IN AN ACTION OF CONTRACT, TORT OR OTHERWISE, ARISING FROM, OUT OF OR IN CONNECTION WITH THE SOFTWARE OR THE USE OR OTHER DEALINGS IN THE SOFTWARE.

%	Email address: echo "cukj -wb- 23wU4X5M589 TROJANS cqkH wiuz2y 0f Mw Stanford" | awk '{ sub("23wU4X5M589","F.d_c_b. ") sub("Stanford","d0mA1n"); print $5, $2, $8; for (i=1; i<=1; i++) print "6\b"; print $9, $7, $6 }' | sed y/kqcbuHwM62z/gnotrzadqmC/ | tr 'q' ' ' | tr -d [:cntrl:] | tr -d 'ir' | tr y "\n"

%%%%%%%%%%%%%%%%%%%%%%%%%%%%%%%%%%%%%%%%%%%%%%



%%%%%%%%%%%%%%%%%%%%%%%%%%%%%%%%%%%%%%%%%%%
\chapter{Algorithms}
\label{chp:Algorithms}


This section documents algorithms that I have implemented for my C++ -based boilerplate code repository. \\


A template for typesetting algorithms is shown in {\sc Procedure} \ref{lst:MyAlgorithm}.

\begin{codebox}
\Procname{$\proc{NAME OF THE ALGORITHM}({\it ARGUMENTS})$}
\label{lst:MyAlgorithm}
\zi \Comment {\it Input ARGUMENT \#1: Definition1}
\zi \Comment {\it Input ARGUMENT \#2: Definition2}
\li BODY OF THE PROCEDURE
\zi \Comment {\it A while loop.}
\li \While [condition]
	\Do
\li	[Something]
	\End
\zi \Comment {\it A for loop.}
\li \For \id{Var} $\gets$ [initial value] \To [final value]
	\Do
\li	[Something]
	\End
\zi \Comment {\it An if-elseif-else block.}
\li	\If $[$Condition1$]$
	\Then
\li		Blah\dots
\li	\ElseIf $[$Condition2$]$
	\Then
\li		Blah\dots
\li	\ElseIf $[$Condition3$]$
	\Then
\li		Blah\dots
%	\li	\ElseNoIf $[$Condition$]$
%		\Then
%	\li		Blah\dots
\li	\Else
\li		Blah\dots	
	\End
\zi \Comment {\it A variable assignment.}
\li $\id{blah} \gets A[j]$
\zi	\>	\Comment {\it This is indented with a tab.}
\zi	\Comment {\it What is the output of this procedure?}
\li	\Return
\end{codebox}








%%%%%%%%%%%%%%%%%%%%%%%%%%%%%%%%%%%%%%%%%%%
\section{Notes on Algorithm Analysis and Design}
\label{sec:NotesOnAlgorithmAnalysisAndDesign}

\cite[\S A.2]{Cormen2009} shows you how to manipulate and bound summations, so that we can obtain an order of (computational) complexity for summations. Specifically, \cite[\S A.2, pp. 1154--1156]{Cormen2009} shows us how to approximate the bound of summations by using finite integrals. {\it Are these related to recurrence relations???}






%%%%%%%%%%%%%%%%%%%%%%%%%%%%%%%%%%%%%%%%%%%
\section{Resources for Algorithms}
\label{sec:ResourcesForAlgorithms}

Resources for algorithms: \vspace{-0.3cm}
\begin{enumerate} \itemsep -4pt
\item Collected Algorithms (CALGO): \url{http://calgo.acm.org/}
\item Netlib Repository at UTK and ORNL \cite{Dongarra2016}: \url{http://www.netlib.org/}
\item ``The Stony Brook Algorithm Repository'' by Steven Skiena \cite{Skiena2008}: \url{http://algorist.com/algorist.html}
\item Cosmos (from OpenGenus Foundation): \url{https://github.com/OpenGenus/cosmos}
\item {\it Wikipedia}: \vspace{-0.3cm}
	\begin{enumerate} \itemsep -2pt
	\item \url{https://en.wikipedia.org/wiki/List_of_algorithm_general_topics}
	\item \url{https://en.wikipedia.org/wiki/List_of_algorithms#Graph_algorithms}
	\end{enumerate}
\end{enumerate}










