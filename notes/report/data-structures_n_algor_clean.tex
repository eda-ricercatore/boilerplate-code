%	This is written by Zhiyang Ong to document, in a report, common data structures and algorithms that are used in electronic design automation software.

%	The MIT License (MIT)

%	Copyright (c) <2014> <Zhiyang Ong>

%	Permission is hereby granted, free of charge, to any person obtaining a copy of this software and associated documentation files (the "Software"), to deal in the Software without restriction, including without limitation the rights to use, copy, modify, merge, publish, distribute, sublicense, and/or sell copies of the Software, and to permit persons to whom the Software is furnished to do so, subject to the following conditions:

%	The above copyright notice and this permission notice shall be included in all copies or substantial portions of the Software.

%	THE SOFTWARE IS PROVIDED "AS IS", WITHOUT WARRANTY OF ANY KIND, EXPRESS OR IMPLIED, INCLUDING BUT NOT LIMITED TO THE WARRANTIES OF MERCHANTABILITY, FITNESS FOR A PARTICULAR PURPOSE AND NONINFRINGEMENT. IN NO EVENT SHALL THE AUTHORS OR COPYRIGHT HOLDERS BE LIABLE FOR ANY CLAIM, DAMAGES OR OTHER LIABILITY, WHETHER IN AN ACTION OF CONTRACT, TORT OR OTHERWISE, ARISING FROM, OUT OF OR IN CONNECTION WITH THE SOFTWARE OR THE USE OR OTHER DEALINGS IN THE SOFTWARE.

%	Email address: echo "cukj -wb- 23wU4X5M589 TROJANS cqkH wiuz2y 0f Mw Stanford" | awk '{ sub("23wU4X5M589","F.d_c_b. ") sub("Stanford","d0mA1n"); print $5, $2, $8; for (i=1; i<=1; i++) print "6\b"; print $9, $7, $6 }' | sed y/kqcbuHwM62z/gnotrzadqmC/ | tr 'q' ' ' | tr -d [:cntrl:] | tr -d 'ir' | tr y "\n"

%%%%%%%%%%%%%%%%%%%%%%%%%%%%%%%%%%%%%%%%%%%%%%









%%%%%%%%%%%%%%%%%%%%%%%%%%%%%%%%%%%%%%%%%%%%%%
%	Preamble.
\documentclass[letterpaper,12pt]{report}
%%%%%%%%%%%%%%%%%%%%%%%%%%%%%%%%%%%%%%%%%%%%%
%
%	Importing LaTeX source files, without quoting the ".tex" extension.
%
%%%%%%%%%%%%%%%%%%%%%%%%%%%%%%%%%%%%%%%%%%%%%

%%%%%%%%%%%%%%%%%%%%%%%%%%%%%%%%%%%%%%%%%%%%%
%	File containing the LaTeX preamble.
% This is written by Zhiyang Ong as the preamble for all his LaTeX documents.
%
% It includes a list of LaTeX packages that he commonly uses to typeset LaTeX documents.

%	The MIT License (MIT)

%	Copyright (c) <2014> <Zhiyang Ong>

%	Permission is hereby granted, free of charge, to any person obtaining a copy of this software and associated documentation files (the "Software"), to deal in the Software without restriction, including without limitation the rights to use, copy, modify, merge, publish, distribute, sublicense, and/or sell copies of the Software, and to permit persons to whom the Software is furnished to do so, subject to the following conditions:

%	The above copyright notice and this permission notice shall be included in all copies or substantial portions of the Software.

%	THE SOFTWARE IS PROVIDED "AS IS", WITHOUT WARRANTY OF ANY KIND, EXPRESS OR IMPLIED, INCLUDING BUT NOT LIMITED TO THE WARRANTIES OF MERCHANTABILITY, FITNESS FOR A PARTICULAR PURPOSE AND NONINFRINGEMENT. IN NO EVENT SHALL THE AUTHORS OR COPYRIGHT HOLDERS BE LIABLE FOR ANY CLAIM, DAMAGES OR OTHER LIABILITY, WHETHER IN AN ACTION OF CONTRACT, TORT OR OTHERWISE, ARISING FROM, OUT OF OR IN CONNECTION WITH THE SOFTWARE OR THE USE OR OTHER DEALINGS IN THE SOFTWARE.

%	Email address: echo "cukj -wb- 23wU4X5M589 TROJANS cqkH wiuz2y 0f Mw Stanford" | awk '{ sub("23wU4X5M589","F.d_c_b. ") sub("Stanford","d0mA1n"); print $5, $2, $8; for (i=1; i<=1; i++) print "6\b"; print $9, $7, $6 }' | sed y/kqcbuHwM62z/gnotrzadqmC/ | tr 'q' ' ' | tr -d [:cntrl:] | tr -d 'ir' | tr y "\n"

%%%%%%%%%%%%%%%%%%%%%%%%%%%%%%%%%%%%%%%%%%%%%%%%%%

% Importing some standard LaTeX packages.

% To enable standard LaTeX processing for graphics. It enables PDF, JPEG, PNG, and TIFF graphics files to be included in the LaTeX document.
\usepackage{graphicx}
% For better typesetting of mathematical expressions, from the American Mathematical Society (AMS).
\usepackage{amsmath}
% For better typesetting of mathematical expressions, from the American Mathematical Society (AMS). This package includes mathematical symbols for the ``amsmath'' package.
\usepackage{amssymb}
% For better typesetting of mathematical proofs (for theorems and colloraries), from the American Mathematical Society (AMS).
\usepackage{amsthm}
%	Create definitions for new theorems, axioms, colloraries.
	\newtheorem{theorem}{Theorem}[chapter]
	\newtheorem{axiom}{Axiom}[chapter]
	\newtheorem{corollary}{Corollary}[chapter]
	\newtheorem{lemma}{Lemma}[chapter]
	\newtheorem{Rule}{Rule}[chapter]
	\newtheorem{law}{Law}[chapter]
	\newtheorem{principle}{Principle}[chapter]
% To change the style of newly defined theorems.
%		\usepackage{theorem}




%	Typesetting with the typewriter font.
%\usepackage{ttfamily}

% For better typesetting of tables (and arrays).
\usepackage{array}
% For creating tables without vertical separators.
%		\usepackage{booktabs}
% To control line spacing in LaTeX documents.
\usepackage{others/packages/setspace}
% To modify the spacing between words and letters.
%		\usepackage{microtype}
% To change the dimensions of the page(s).
%\usepackage[margin=1.5cm,vmargin={0pt,1cm},nohead]{geometry}
\usepackage[margin=1.5cm,vmargin={1.5cm,2cm}]{geometry}
% Use the packages needed to typeset algorithms. I can also use the combined ``algorithms'' bundle.
\usepackage{others/packages/algorithm}
\usepackage{others/packages/algorithmic}
% The listings package is a source code printer for LaTeX. You can typeset stand alone files as well as listings with an environment similar to verbatim as well as you can print code snippets using a command similar to \verb. Many parameters control the output and if your preferred programming language isn�t already supported, you can make your own definition.
\usepackage{others/packages/listings}
% Use the ``clrscode3e'' LaTeX package to typeset algorithms like CLRS
%	\usepackage{/data/others/notes/clrscode3e}
%/Applications/apps/comune/SienaLaTeX/rapporto/
%\usepackage{/Applications/apps/comune/SienaLaTeX/rapporto/others/clrscode3e}
\usepackage{others/packages/clrscode3e}
%\usepackage{/data/others/grappanotes/clrscode3e}
% Use the ``algpseudocode'' LaTeX package to typeset algorithms -- Alternate solution, not preferred
%\usepackage{algpseudocode}
% Alternative packages for typesetting algorithms.
%\usepackage{algorithm2e}
%\usepackage{algorithmicx}
%\usepackage{program}
%	To check for syntax errors in my LaTeX document.
\RequirePackage[l2tabu, orthodox]{others/packages/nag}

% Concatenate adjacent references together when typeset.
% That is, cite{ref1,ref2,ref3,ref4} can appear as [12-15], instead of [12] [13] [14] [15]
\usepackage{others/packages/cite}
% For automatic insertion of cross-referencing words, such as fig. for figures and eq. for equations.
%		\usepackage{cleveref}

% LaTeX support for Metafont and MetaPost logos.
\usepackage{mflogo}













% How to typeset single and double quotes for feet and inches?
% For feet, use [FEET]\textasciiacute
% For inches, use [INCHES]\textacutedbl
% For feet and inches, use [FEET]\textasciiacute\ [INCHES]\textacutedbl; force a character space between the single quote for feet and the height of the object in inches
% Don't use \textceltpal as a single quote to represent height in feet, or double \textceltpal (two concatenated \textceltpal) as a double quote to represent height in inches
% For double quotes, don't use two single quotes provided by the default settings of LaTeX. The resultant double quotes will be curly.

% The tipa package is for Phonetic Symbols -- I wanna use the \textceltpal symbol to represent a single quote, instead of using the generic ``curly'' single quote from \LaTeX (Table 10, pp.10)
\usepackage{tipa}
% The textcomp package is for Diacritics -- I wanna use the \textacutedbl symbol to represent a double quote (Table 28, pp.17), instead of using the generic ``curly'' double quotes from \LaTeX; however, when this symbol is used, I must force a character space to exist after the symbol by using the backslash followed by a character space. This package also provides the symbol for Copyleft, \textcopyleft, which is not available in LaTeX by default, and provides better looking symbols for: copyright, registered, and trademark (Table 33, pp.18). Also, it provides symbols for: \textcelsius, \textmho, \textmu, \textohm (Table 201, pp.67). It also provides symbols for Genealogical Symbols (Table 253, pp78), such as \textborn, \textdivorced, \textmarried, \textdied, and \textleaf (symbol of a leaf)... Its symbol for the Euro, EU currency, is \texteuro
\usepackage{textcomp}
% Look at \url{http://www.ctan.org/tex-archive/info/symbols/comprehensive/symbols-a4.pdf} for a list of symbols that can be used in LaTeX and its packages. Table 280, pp.88, deals with Symbol Name Clashes; hence, if the same command name refers to multiple symbols, the symbol-conflict resolution abides by this.
% In particular, check out the gensymb package (Table 197, page 67) for symbols defined to work in both math and text modes, such as \celsius, \micro, \degree, and \ohm.
% Also, check out the wasysym package (Table 198, page 67) for electrical symbols, such as that of alternating current (AC); it also provides symbols for \female, and \male (Table 212, pp.70); it also has symbols for ``Xs and Check Marks,'' which are checked boxes, \CheckedBox, squares, \Square, and crossed boxes (boxes filled with a cross), \XBox (Table 232, pp.73); it also has symbols for a clock, \clock, a Simley, \smiley, diameter, \diameter, lightning, \lightning, sun, \sun, and a tick or check mark \checked (symbol to indicate that something is correct), and a bell, \bell (Table 254, pp.78); it also has symbols for left and right turns (Table 256, pp.78), \leftturn and \rightturn; this package (Table 256, pp.78) and the arev package (Table 257, pp.78) can be used to typeset music symbols, along with Table 182, pp.62; it also has symbols for Navigation (Table 261, pp.79), such as \Forward, \RewindToStart, and \ForwardToIndex; it also has symbols for laundry (Table 262, pp.80); it also has the symbol for a heart, \Heart (Table 263, pp.80).
% In addition, check out the ifsym package (Table 199, page 67) for pulse diagram symbols; it also has symbols for weather (Table 266, pp.80), alpine and mountain climbing, such as \Summit, \Mountain, \IceMountain, \VarMountain, \Flag, \FilledHut, \Hut, \Village, and \Tent (Table 267, pp.81); it also has different symbols for clocks, such as \Interval, \StopWatchEnd, \VarClock, \showclock (to indicate the time) (Table 268, pp.81); it also has symbols for fire, letter, telephone, dice, \PaperPortrait, and \PaperLandscape. Also, has symbol for the cross to indicate that something is incorrect
\usepackage{others/packages/ifsym}
% Besides, check out the keystroke package (Table 208, page 69) for symbols of Computer Keys, such as Alt, Ctrl, Del, Page down, Esc, Enter, Shift, Space Bar, and Up Arrow.
% From the dingbat package (Table 225, page 72), it has symbols for Fists, such as \rightthumbsdown and \rightthumbsup.
%\usepackage{dingbat}
% From the pifont package (Table 234, page 73), it has symbols for Circled Numbers, such as any digit that is circled, where the space in the circle can be shaded black.
% From the dictsym package (Table 277, page 84), it has symbols for dictionaries, and indicates which type of dictionary will define this term - say a medical, technical, mathematical, or judical dictionary
% The simpsons package can be used to indicate characters from {\it The Simpsons} (Table 278, pp.85)
% The symbol for quadruple integrals \iiiint is available as an AMS Variable-sized Math Operator, or I can use this symbol from the packages txfonts, pxfonts, esint, or MnSymbol 










% The marvosym package (Table 210, page 69) is for Communication Symbols, such as \Email, \fax, \FAX (Preferred), \Letter, \Mobilefone, and \Telefon; it also has the symbol for the Cross to represent Christianity, \Cross (Table 263, pp.80); it also has symbols for checked boxes, \Checkedbox, crossed boxes (boxes marked with a cross), \Crossedbox, bicycles, \Bicycle, clocks, \Clocklogo, the industry, \Industry, taking notes manually with pen/pencil and paper, \Writinghand, coffee, \Coffeecup, providing information or important note, \Info (Table 249, pp.76)... In addition, it has the symbols for the Euro (EU currency), \EUR (OK), \EURdig (OK), \EURtm, \EURcr
\usepackage{marvosym}
% From the bbding package (Table 226, page 72), it has symbols for Fists, such as \HandPencilLeft; it also has symbols for the Cross to represent Christianity, such as \Cross and \CrossOpenShadow (Table 228, pp.72); Use of the symbol \Cross has bugs; bugs exist in the package, as it fails to correctly overwrite the \Cross symbol; also has the peace symbol, \Peace. 
%\usepackage{bbding}
% The skak contains a cross, incorrect symbol that I can use to indicate that something is wrong, e.g. \markera or \weakpt
%%%%%%%%%%%%%\usepackage{others/packages/skak}
% Package to enable the use of a strikeout/strikethrough font with LaTeX. To use the strikeout/strikethrough font, use the ``sout'' LaTeX command, or tag,  to ``strike through'' text. E.g., \sout{Bill Clinton} G.W. Bush is the pres.
\usepackage{others/packages/ulem}
% The eurosym package has the symbols for the Euro (EU currency), \geneuro, \geneuronarrow, \geneurowide, \officialeuro (GOOD)
\usepackage{eurosym}











% Create fancy headers and footers for this document
\usepackage{fancyhdr}
\setlength{\headheight}{15.2pt}
\pagestyle{fancy}
% Headers for the document
\lhead{}
%\lhead{Zhiyang Ong}
%\rhead{\today}
% Footers for the document
\lfoot{Zhiyang Ong}
\cfoot{}
\rfoot{\thepage}

% The following does not work, since it does not differentiate between odd and even pages. Hence, the last odd/even command will overwrite the previous even/odd command
%\fancyhf{}
%\fancyhead[LE]{Author's DFM}
%\fancyhead[LO]{\today EDA}
%\fancyfoot[LE]{\thepage USC}
%\fancyfoot[RO]{\thepage Adel}


% Allow for multi-line comments
\usepackage{verbatim}




% Commands for using the package for hyperlinks. Includes the package ``url''.
\usepackage[pdftex,
	pdftitle={Graphics and Color with LaTeX},
	pdfauthor={Patrick W Daly},
	pdfsubject={Importing images and use of color in LaTeX},
	pdfkeywords={LaTeX, graphics, color},
	pdfpagemode=UseOutlines,bookmarks, bookmarksopen,
	pdfstartview=FitH, colorlinks, linkcolor=blue, citecolor=blue, urlcolor=red,
]{hyperref}
\hypersetup{colorlinks, linkcolor=blue}







% Create a glossary for symbols and terms in this document
% The following attempt failed
%\makeglossaries

% The following attempt failed
%%%%%%%%%%%%%%%%%%%%%%%%%%%%%\makeglossary
%\usepackage{supertabular}
%\newcommand{\glossaryname}{Symbols Index}
%\newenvironment{theglossary}
%    {\section*{Symbols Index}
%      \begin{supertabular}{ll}}
%    {\end{supertabular}
%}
%\newcommand{\printglossary}{\InputIfFileExists{zhiyang_ong.glo}{}{\section*{Symbols Index - File not found}}}

% Another failed attempt at creating a glossary
%\input{gatech-thesis-gloss.sty}
%\usepackage{gatech-thesis-gloss}
%\glossfiles{zhiyang_ong.glo}

% Create the glossary
%%%%%%%%%%%%%\usepackage{nomencl}
%%%%%%%%%%%%%\makenomenclature


% Enable captions to be modified.
%\usepackage{caption}
% Addition support for colored text.
%\usepackage{color}
% Enable the insertion of PDF/PS files/documents.
%%%%%%%%%%%		\usepackage{others/packages/pdfpages}
% To rotate objects, including tables.
		\usepackage{rotating}
% To define multiple floats (figures and tables), with individual captions and labels, within one environment.
		\usepackage{others/packages/subfig}
% For a modular LaTeX document with multiple files (including the ``root file''), it allows the a non-empty subset of the ``child files'' to be typeset without having to typeset the ``root file'' (and/or the other ``child files'').
		\usepackage{others/packages/subfiles}
% To annotate the LaTeX document with to-do notes.
%%%%%%%%%%%		\usepackage[colorinlistoftodos]{others/packages/todonotes}
% To insert images surrounded by text.
		\usepackage{others/packages/wrapfig}
% To create trees, graphs, (commutative) diagrams, and similar things. Reference: Wikibooks contributors, ``\LaTeX/Xy-pic,'' in {\it \LaTeX}, Wikibooks: Open books for an open world, Wikimedia Foundation, San Francisco, CA, June 5, 2005. Available online at: \url{http://en.wikibooks.org/wiki/LaTeX/Xy-pic}; last accessed on December 25, 2013.		=> This package seems to have bugs in it. If I use this package, my document will not typeset properly. I have tried to use it successfully in other documents. It does not seem to be compatible with 
%\usepackage{xypic}
% Package for SI units.
%%%%%%%%%%%\usepackage{others/packages/siunitx}






%	For typesetting the symbol: \AE.
\usepackage[T1]{fontenc}
\usepackage[utf8]{inputenc}
\usepackage{lmodern}













%%%%%%%%%%%%%%%%%%%%%%%%%%%%%%%%%%%%%%%%%%%%%%%%%%
% Other helpful hints:

% To use the italic and bold font concurrently, try this: {\itshape Review the {\bfseries updated} training log}

% To use the symbol for summation, which is the capital-sigma notation, with proper super- and sub- fixes, try: $\displaystyle\sum_{i = -1}^{m} \frac{log_2 n_i}{T_i}$

% Make sure that I include the following so that I can cite references properly: \usepackage{cite}. This allows references to be included as [1-10], rather than [1], [2], [3], [4], [5], [6], [7], [8], [9], [10]

% Colors that appear well in PDF format for LaTeX text include: red, blue, and magenta

% Use \scriptsize, instead of \textsc, \sc, or \schape to use small caps. Currently, I cannot use \textsc, \sc, or \schape to write in small caps on my MacBook Pro laptop.

% The Typewriter font cannot be used concurrently with the bold font. That is, the following cannot be used: {\tt \bf text}, AND \texttt{\textbf{text}}

% Use \LaTeX for LaTeX; B{\scriptsize IB}\TeX to indicate the symbol for BibTeX; \texttrademark for trademarks; \MF for Metafont; and \MP for MetaPost




%%%%%%%%%%%%%%%%%%%%%%%%%%%%%%%%%%%%%%%%%%
%																%
%	Default colors that I can use with \LaTeX:								%
%	1) red														%
%	2) green														%
%	3) blue														%
%	4) yellow														%
%	5) cyan														%
%	6) magenta													%
%	7) black														%
%	8) white														%
%																%
%%%%%%%%%%%%%%%%%%%%%%%%%%%%%%%%%%%%%%%%%%


% Partial list of ``the 68 predefined internal colors of the {\tt dvips} PostScript driver'' \cite{Kopka04} that I can use for changing the color of text ... Use bold font for the text
%YellowOrange
%RoyalBlue
%DarkOrchid
%ForestGreen
%OliveGreen
%Mulberry
%ProcessBlue
%RubineRed
%VioletRed
%WildStrawberry
% E.g., try: \textcolor{VioletRed}{\bf hello world}

% As for changing the background color of text, choose a light colored background to make the text stand out in black colored bold font; see \url{oregonstate.edu/~peterseb/tex/samples/docs/color-package-demo.pdf} for a list of colors
% E.g., try: \colorbox{Apricot}{\bf hello world}
%	Enable the use of right-sided cases.
\usepackage{others/packages/mathtools}
%\usepackage{extarrows}
\usepackage{others/packages/proof}



%%%%%%%%%%%%%%%%%%%%%%%%%%%%%%%%%%%%%%%%%%%%%
%	New commands to typeset Vietnamese fonts
%%%%%%%%%%%%\usepackage{others/packages/stackengine}
%%%%%%%%%%%%\usepackage{others/packages/scalerel}
\def\stackalignment{r}
%	Required shortcut for creating commands.
\def\shortcomma{\vstretch{.7}{,}}
%	command ``\uhorn'' produces: u'
\newcommand\uhorn{\topinset{\shortcomma}{u}{.05ex}{-.1ex}}
%	References:
%	Steven B. Segletes, answer to ``Horn accent in LaTeX,'' Stack Exchange Inc., New York, NY, August 7, 2013. Available online from {\it Stack Exchange Inc.: Stack Overflow: Questions} at: \url{http://tex.stackexchange.com/a/127173}; March 16, 2016 was the last accessed date.

 



%%%%%%%%%%%%%%%%%%%%%%%%%%%%%%%%%%%%%%%%%%%%%
%
%	Start of LaTeX document
%
%%%%%%%%%%%%%%%%%%%%%%%%%%%%%%%%%%%%%%%%%%%%%
\begin{document}

%%%%%%%%%%%%%%%%%%%%%%%%%%%%%%%%%%%%%%%%%
%	File containing a list of ``the 68 predefined internal colors of the {\tt dvips} PostScript driver'' \cite{Kopka04} 
%	This allows me to use any of these ``68 predefined internal colors''
% This is written by Zhiyang Ong to recreate the ``68 predefined internal colors of the {\tt dvips} PostScript driver'' \cite{Kopka04}
%
%
% I have written this because I cannot get the aforementioned 68 predefined internal colors to work for any \LaTeX document on my MacBook Pro. Perhaps the \LaTeX\ setup is the cause of the problem.
% I have used the \usepackage{color} command with the ``dvipsnames'' and ``usenames'' options to no avail
% The \usepackage{color} command with the aforementioned options are added into the preamble \cite{Kopka04} and see \url{http://oregonstate.edu/~peterseb/tex/samples/color-package.html} (last viewed Monday, May 25, 2009 @ 0900 hrs PST)
% \usepackage[dvipsnames]{color}
% \usepackage[usenames]{color}
%
%
% Hence, I used a file ``dvipsnam.def'' containing the list of these 68 predefined internal colors to create the definitions of these 68 colors for use in my \LaTeX\ document(s)
% The original source file is located at \url{http://spib.ece.rice.edu/E-Pub/color/dvipsnam.def}
% This source file is provided in the Signal Processing Information Base (SPIB), which is made available by courtesy of the Department of Electrical and Computer Engineering, Rice University
%
%	The MIT License (MIT)
%
%	Copyright (c) <2014> <Zhiyang Ong>
%
%	Permission is hereby granted, free of charge, to any person obtaining a copy of this software and associated documentation files (the "Software"), to deal in the Software without restriction, including without limitation the rights to use, copy, modify, merge, publish, distribute, sublicense, and/or sell copies of the Software, and to permit persons to whom the Software is furnished to do so, subject to the following conditions:
%
%	The above copyright notice and this permission notice shall be included in all copies or substantial portions of the Software.
%
%	THE SOFTWARE IS PROVIDED "AS IS", WITHOUT WARRANTY OF ANY KIND, EXPRESS OR IMPLIED, INCLUDING BUT NOT LIMITED TO THE WARRANTIES OF MERCHANTABILITY, FITNESS FOR A PARTICULAR PURPOSE AND NONINFRINGEMENT. IN NO EVENT SHALL THE AUTHORS OR COPYRIGHT HOLDERS BE LIABLE FOR ANY CLAIM, DAMAGES OR OTHER LIABILITY, WHETHER IN AN ACTION OF CONTRACT, TORT OR OTHERWISE, ARISING FROM, OUT OF OR IN CONNECTION WITH THE SOFTWARE OR THE USE OR OTHER DEALINGS IN THE SOFTWARE.
%
%	Email address: echo "cukj -wb- 23wU4X5M589 TROJANS cqkH wiuz2y 0f Mw Stanford" | awk '{ sub("23wU4X5M589","F.d_c_b. ") sub("Stanford","d0mA1n"); print $5, $2, $8; for (i=1; i<=1; i++) print "6\b"; print $9, $7, $6 }' | sed y/kqcbuHwM62z/gnotrzadqmC/ | tr 'q' ' ' | tr -d [:cntrl:] | tr -d 'ir' | tr y "\n"

%%%%%%%%%%%%%%%%%%%%%%%%%%%%%%%%%%%%%%%%%%%%%%



%%%%%%%%%%%%%%%%%%%%%%%%%%%%%%%%%%%%%%%
% Text retained from the header of the original ``dvipsnam.def'' file
%%
%% This is file `dvipsnam.def',
%% generated with the docstrip utility.
%%
%% The original source files were:
%%
%% drivers.dtx  (with options: `dvipsnames')
%% 
%% drivers.dtx Copyright (C) 1994      David Carlisle Sebastian Rahtz
%%             Copyright (C) 1995 1996 1997 1998 1999 David Carlisle
%%
%% This file is part of the Standard LaTeX `Graphics Bundle'.
%% It may be distributed under the terms of the LaTeX Project Public
%% License, as described in lppl.txt in the base LaTeX distribution.
%% Either version 1.0 or, at your option, any later version.
%%




%%%%%%%%%%%%%%%%%%%%%%%%%%%%%%%%%%%%%%%
% Commence definition of ``the 68 predefined internal colors of the {\tt dvips} PostScript driver'' \cite{Kopka04}
\definecolor{GreenYellow}{cmyk}{0.15,0,0.69,0}
\definecolor{Yellow}{cmyk}{0,0,1,0}
\definecolor{Goldenrod}{cmyk}{0,0.10,0.84,0}
\definecolor{Dandelion}{cmyk}{0,0.29,0.84,0}
\definecolor{Apricot}{cmyk}{0,0.32,0.52,0}
\definecolor{Peach}{cmyk}{0,0.50,0.70,0}
\definecolor{Melon}{cmyk}{0,0.46,0.50,0}
\definecolor{YellowOrange}{cmyk}{0,0.42,1,0}
\definecolor{Orange}{cmyk}{0,0.61,0.87,0}
\definecolor{BurntOrange}{cmyk}{0,0.51,1,0}
\definecolor{Bittersweet}{cmyk}{0,0.75,1,0.24}
\definecolor{RedOrange}{cmyk}{0,0.77,0.87,0}
\definecolor{Mahogany}{cmyk}{0,0.85,0.87,0.35}
\definecolor{Maroon}{cmyk}{0,0.87,0.68,0.32}
\definecolor{BrickRed}{cmyk}{0,0.89,0.94,0.28}
\definecolor{Red}{cmyk}{0,1,1,0}
\definecolor{OrangeRed}{cmyk}{0,1,0.50,0}
\definecolor{RubineRed}{cmyk}{0,1,0.13,0}
\definecolor{WildStrawberry}{cmyk}{0,0.96,0.39,0}
\definecolor{Salmon}{cmyk}{0,0.53,0.38,0}
\definecolor{CarnationPink}{cmyk}{0,0.63,0,0}
\definecolor{Magenta}{cmyk}{0,1,0,0}
\definecolor{VioletRed}{cmyk}{0,0.81,0,0}
\definecolor{Rhodamine}{cmyk}{0,0.82,0,0}
\definecolor{Mulberry}{cmyk}{0.34,0.90,0,0.02}
\definecolor{RedViolet}{cmyk}{0.07,0.90,0,0.34}
\definecolor{Fuchsia}{cmyk}{0.47,0.91,0,0.08}
\definecolor{Lavender}{cmyk}{0,0.48,0,0}
\definecolor{Thistle}{cmyk}{0.12,0.59,0,0}
\definecolor{Orchid}{cmyk}{0.32,0.64,0,0}
\definecolor{DarkOrchid}{cmyk}{0.40,0.80,0.20,0}
\definecolor{Purple}{cmyk}{0.45,0.86,0,0}
\definecolor{Plum}{cmyk}{0.50,1,0,0}
\definecolor{Violet}{cmyk}{0.79,0.88,0,0}
\definecolor{RoyalPurple}{cmyk}{0.75,0.90,0,0}
\definecolor{BlueViolet}{cmyk}{0.86,0.91,0,0.04}
\definecolor{Periwinkle}{cmyk}{0.57,0.55,0,0}
\definecolor{CadetBlue}{cmyk}{0.62,0.57,0.23,0}
\definecolor{CornflowerBlue}{cmyk}{0.65,0.13,0,0}
\definecolor{MidnightBlue}{cmyk}{0.98,0.13,0,0.43}
\definecolor{NavyBlue}{cmyk}{0.94,0.54,0,0}
\definecolor{RoyalBlue}{cmyk}{1,0.50,0,0}
\definecolor{Blue}{cmyk}{1,1,0,0}
\definecolor{Cerulean}{cmyk}{0.94,0.11,0,0}
\definecolor{Cyan}{cmyk}{1,0,0,0}
\definecolor{ProcessBlue}{cmyk}{0.96,0,0,0}
\definecolor{SkyBlue}{cmyk}{0.62,0,0.12,0}
\definecolor{Turquoise}{cmyk}{0.85,0,0.20,0}
\definecolor{TealBlue}{cmyk}{0.86,0,0.34,0.02}
\definecolor{Aquamarine}{cmyk}{0.82,0,0.30,0}
\definecolor{BlueGreen}{cmyk}{0.85,0,0.33,0}
\definecolor{Emerald}{cmyk}{1,0,0.50,0}
\definecolor{JungleGreen}{cmyk}{0.99,0,0.52,0}
\definecolor{SeaGreen}{cmyk}{0.69,0,0.50,0}
\definecolor{Green}{cmyk}{1,0,1,0}
\definecolor{ForestGreen}{cmyk}{0.91,0,0.88,0.12}
\definecolor{PineGreen}{cmyk}{0.92,0,0.59,0.25}
\definecolor{LimeGreen}{cmyk}{0.50,0,1,0}
\definecolor{YellowGreen}{cmyk}{0.44,0,0.74,0}
\definecolor{SpringGreen}{cmyk}{0.26,0,0.76,0}
\definecolor{OliveGreen}{cmyk}{0.64,0,0.95,0.40}
\definecolor{RawSienna}{cmyk}{0,0.72,1,0.45}
\definecolor{Sepia}{cmyk}{0,0.83,1,0.70}
\definecolor{Brown}{cmyk}{0,0.81,1,0.60}
\definecolor{Tan}{cmyk}{0.14,0.42,0.56,0}
\definecolor{Gray}{cmyk}{0,0,0,0.50}
\definecolor{Black}{cmyk}{0,0,0,1}
\definecolor{White}{cmyk}{0,0,0,0}









%%%%%%%%%%%%%%%%%%%%%%%%%%%%%%%%%%%%%%%%%%%%%
%%%%%%%%%%%%%%%%%%%%%%%%%%%%%%%%%%%%%%%%%%%%%
%
%	Information for Top Matter / Title page
%
%%%%%%%%%%%%%%%%%%%%%%%%%%%%%%%%%%%%%%%%%%%%%
%%%%%%%%%%%%%%%%%%%%%%%%%%%%%%%%%%%%%%%%%%%%%
%	Beginning of FRONT MATTER: title page, table of contents and prefaces
%	Note that the \vspace{length} command does NOT work for the front matter
%\frontmatter
\title{\Huge \bf Boilerplate Code: Data Structures and Algorithms for Design Automation}

%	Indicate the date of the report.
\date{\today}

\author{{\LARGE Zhiyang Ong}
\thanks{Email correspondence to: \href{mailto:ongz@acm.org}{\Email\ ongz@acm.org}}
\ \\
\vspace{-4.0in}
\ \\
\ \\
\ \\
{\bf \LARGE
	Design Automation Renegades
	\vspace{0.1cm}} \\
\hline
\ \\
{\Large \sc Globetrotting Division} \\
\ \\
\ \\
\ \\
\ \\
\ \\
\vspace{2.0in}
\ \\
{\large \sc Report on } \\
{\large Common Data Structures and Algorithms} \\
{\large Found in Bolierplate Code for} \\
{\large Design Automation Software}
}

%	Create the title page.
\maketitle


%%%%%%%%%%%%%%%%%%%%%%%%%%%%%%%%%%%%%%%%%%%%%
%
%	Abstract

\begin{abstract} 
This report describes the design and implementation of common data structures and algorithms, as well as ``computational engines'' that are found in electronic design automation (EDA) software. \\

Data structures and algorithms for digital VLSI and cyber-physical system design include: binary decision diagrams (BDDs), AND-inverter graphs (AIGs), and their associated algorithms for optimization, traversal, and other operations (such as graph matching). Common computational engines for digital systems would include: optimization and verification engines for deterministic and nondeterministic finite state machines; decision procedures for the boolean satisfiability problem (SAT solvers) and satisfiability modulo theories (SMT solvers); quantified boolean formula (QBF) solvers; and SAT and SMT solvers for maximum satisfiability (i.e., Max-SAT and Max-SMT solvers). \\

Regarding EDA problems that require numerical computation (in digital, analog, or mixed-signal VLSI design), the data structures and algorithms for circuit simulation based on sparse graph would be required. In addition, techniques for model order reduction shall be implemented. \\

Computational engines for statistical and probabilistic analyses or stochastic modeling can include data structures and algorithms for partially observable Markov decision processes (POMDPs) and Markov chains. Tools for analyses of queueing systems (based on queueing theory) should be included. \\

Regarding cyber-physical systems and mixed-signal circuits, hybrid automata can be used to represent these circuits and systems. \\

Optimization engines for EDA include: solvers for different types of mathematical programming, such as linear programming (LP), integer linear programming (ILP), mixed-integer linear programming (MILP), quadratic programming (QP), convex programming (CP), geometric programming (GP), and second-order conic programming (SOCP); solvers for pseudo-boolean optimization (PBO solvers) and weighted-boolean optimization (WBO); and meta-heuristics (e.g., evolutionary algorithms, simulated annealing, and ant colony optimization). \\

Algorithms shall be implemented using parallel programming, in a scalable style. In addition, considerations shall be given to the use of constraint programming. \\
\ \\
\ \\
More stuff to be included\dots
\end{abstract}

%%%%%%%%%%%%%%%%%%%%%%%%%%%%%%%%%%%%%%%%%%%%%
%%%%%%%%%%%%%%%%%%%%%%%%%%%%%%%%%%%%%%%%%%%%%

% Set the page numbering to lowercase Roman numerals
\pagenumbering{roman}
% Set the initial page number of the pages in the content section to be ``i''
\setcounter{page}{1}

%%%%%%%%%%%%%%%%%%%%%%%%%%%%%%%%%%%%%%%%%
%	Revision History
%	This is written by Zhiyang Ong to record significant changes made to the report.

%	The MIT License (MIT)

%	Copyright (c) <2014> <Zhiyang Ong>

%	Permission is hereby granted, free of charge, to any person obtaining a copy of this software and associated documentation files (the "Software"), to deal in the Software without restriction, including without limitation the rights to use, copy, modify, merge, publish, distribute, sublicense, and/or sell copies of the Software, and to permit persons to whom the Software is furnished to do so, subject to the following conditions:

%	The above copyright notice and this permission notice shall be included in all copies or substantial portions of the Software.

%	THE SOFTWARE IS PROVIDED "AS IS", WITHOUT WARRANTY OF ANY KIND, EXPRESS OR IMPLIED, INCLUDING BUT NOT LIMITED TO THE WARRANTIES OF MERCHANTABILITY, FITNESS FOR A PARTICULAR PURPOSE AND NONINFRINGEMENT. IN NO EVENT SHALL THE AUTHORS OR COPYRIGHT HOLDERS BE LIABLE FOR ANY CLAIM, DAMAGES OR OTHER LIABILITY, WHETHER IN AN ACTION OF CONTRACT, TORT OR OTHERWISE, ARISING FROM, OUT OF OR IN CONNECTION WITH THE SOFTWARE OR THE USE OR OTHER DEALINGS IN THE SOFTWARE.

%	Email address: echo "cukj -wb- 23wU4X5M589 TROJANS cqkH wiuz2y 0f Mw Stanford" | awk '{ sub("23wU4X5M589","F.d_c_b. ") sub("Stanford","d0mA1n"); print $5, $2, $8; for (i=1; i<=1; i++) print "6\b"; print $9, $7, $6 }' | sed y/kqcbuHwM62z/gnotrzadqmC/ | tr 'q' ' ' | tr -d [:cntrl:] | tr -d 'ir' | tr y "\n"

%%%%%%%%%%%%%%%%%%%%%%%%%%%%%%%%%%%%%%%%%%%%%
\chapter*{Revision History}
\addcontentsline{toc}{chapter}{Revision History}
\label{chp:revisionhistory}


Revision History: \vspace{-0.3cm}
\begin{enumerate} \itemsep -4pt
\item Version 0.1, December 23, 2014. Initial copy of the report.
\item Version 0.1.1, September 16, 2015. Added sections for mathematics and statistics, and the abstract.
\item Version 0.1.2, November 10, 2018. Added sections for graphs, including directed graphs (digraphs), directed acyclic graphs (DAGs), and undirected graphs.
\end{enumerate}








%%%%%%%%%%%%%%%%%%%%%%%%%%%%%%%%%%%%%%%%%
%	Create the table of contents
\tableofcontents
%	Start the numbering of chapters from 1, instead of 0.
%\setcounter{chapter}{1}
%	Increase the depth of each section in the Table of Contents to 4.
\setcounter{secnumdepth}{4}
%\setcounter{tocdepth}{4}

%	List of Figures		=> Insert Here!!!
%	List of Tables			=> Insert Here!!!
%	List of To-Do Tasks	=> Insert Here!!!
%\listoftodos





%%%%%%%%%%%%%%%%%%%%%%%%%%%%%%%%%%%%%%%%%%%%%
%%%%%%%%%%%%%%%%%%%%%%%%%%%%%%%%%%%%%%%%%%%%%
\newpage
%\mainmatter

%	Body, or main section, of the document
%	Set the page numbering to normal (Arabic) numerals
\pagenumbering{arabic}
% Set the page numbers for the body of the document
\setcounter{page}{1}




%%%%%%%%%%%%%%%%%%%%%%%%%%%%%%%%%%%%%%%%%
%	Algorithms to be Implemented
%	This is written by Zhiyang Ong to document algorithms that I have implemented for my C++ -based boilerplate code repository.

%	The MIT License (MIT)

%	Copyright (c) <2014> <Zhiyang Ong>

%	Permission is hereby granted, free of charge, to any person obtaining a copy of this software and associated documentation files (the "Software"), to deal in the Software without restriction, including without limitation the rights to use, copy, modify, merge, publish, distribute, sublicense, and/or sell copies of the Software, and to permit persons to whom the Software is furnished to do so, subject to the following conditions:

%	The above copyright notice and this permission notice shall be included in all copies or substantial portions of the Software.

%	THE SOFTWARE IS PROVIDED "AS IS", WITHOUT WARRANTY OF ANY KIND, EXPRESS OR IMPLIED, INCLUDING BUT NOT LIMITED TO THE WARRANTIES OF MERCHANTABILITY, FITNESS FOR A PARTICULAR PURPOSE AND NONINFRINGEMENT. IN NO EVENT SHALL THE AUTHORS OR COPYRIGHT HOLDERS BE LIABLE FOR ANY CLAIM, DAMAGES OR OTHER LIABILITY, WHETHER IN AN ACTION OF CONTRACT, TORT OR OTHERWISE, ARISING FROM, OUT OF OR IN CONNECTION WITH THE SOFTWARE OR THE USE OR OTHER DEALINGS IN THE SOFTWARE.

%	Email address: echo "cukj -wb- 23wU4X5M589 TROJANS cqkH wiuz2y 0f Mw Stanford" | awk '{ sub("23wU4X5M589","F.d_c_b. ") sub("Stanford","d0mA1n"); print $5, $2, $8; for (i=1; i<=1; i++) print "6\b"; print $9, $7, $6 }' | sed y/kqcbuHwM62z/gnotrzadqmC/ | tr 'q' ' ' | tr -d [:cntrl:] | tr -d 'ir' | tr y "\n"

%%%%%%%%%%%%%%%%%%%%%%%%%%%%%%%%%%%%%%%%%%%%%%



%%%%%%%%%%%%%%%%%%%%%%%%%%%%%%%%%%%%%%%%%%%
\chapter{Algorithms}
\label{chp:Algorithms}


This section documents algorithms that I have implemented for my C++ -based boilerplate code repository. \\


A template for typesetting algorithms is shown in {\sc Procedure} \ref{lst:MyAlgorithm}.

\begin{codebox}
\Procname{$\proc{NAME OF THE ALGORITHM}({\it ARGUMENTS})$}
\label{lst:MyAlgorithm}
\zi \Comment {\it Input ARGUMENT \#1: Definition1}
\zi \Comment {\it Input ARGUMENT \#2: Definition2}
\li BODY OF THE PROCEDURE
\zi \Comment {\it A while loop.}
\li \While [condition]
	\Do
\li	[Something]
	\End
\zi \Comment {\it A for loop.}
\li \For \id{Var} $\gets$ [initial value] \To [final value]
	\Do
\li	[Something]
	\End
\zi \Comment {\it An if-elseif-else block.}
\li	\If $[$Condition1$]$
	\Then
\li		Blah\dots
\li	\ElseIf $[$Condition2$]$
	\Then
\li		Blah\dots
\li	\ElseIf $[$Condition3$]$
	\Then
\li		Blah\dots
%	\li	\ElseNoIf $[$Condition$]$
%		\Then
%	\li		Blah\dots
\li	\Else
\li		Blah\dots	
	\End
\zi \Comment {\it A variable assignment.}
\li $\id{blah} \gets A[j]$
\zi	\>	\Comment {\it This is indented with a tab.}
\zi	\Comment {\it What is the output of this procedure?}
\li	\Return
\end{codebox}



%%%%%%%%%%%%%%%%%%%%%%%%%%%%%%%%%%%%%%%%%%%
\section{Resources for Algorithms}
\label{sec:ResourcesForAlgorithms}

Resources for algorithms: \vspace{-0.3cm}
\begin{enumerate} \itemsep -4pt
\item Collected Algorithms (CALGO): \url{http://calgo.acm.org/}
\item Netlib Repository at UTK and ORNL \cite{Dongarra2016}: \url{http://www.netlib.org/}
\item ``The Stony Brook Algorithm Repository'' by Steven Skiena \cite{Skiena2008}: \url{http://algorist.com/algorist.html}
\item Cosmos (from OpenGenus Foundation): \url{https://github.com/OpenGenus/cosmos}
\item {\it Wikipedia}: \vspace{-0.3cm}
	\begin{enumerate} \itemsep -2pt
	\item \url{https://en.wikipedia.org/wiki/List_of_algorithm_general_topics}
	\item \url{https://en.wikipedia.org/wiki/List_of_algorithms#Graph_algorithms}
	\end{enumerate}
\end{enumerate}
















%%%%%%%%%%%%%%%%%%%%%%%%%%%%%%%%%%%%%%%%%
%	Data Structures to be Implemented
%	This is written by Zhiyang Ong to document data structures that I have implemented for my C++ -based boilerplate code repository.

%	The MIT License (MIT)

%	Copyright (c) <2014> <Zhiyang Ong>

%	Permission is hereby granted, free of charge, to any person obtaining a copy of this software and associated documentation files (the "Software"), to deal in the Software without restriction, including without limitation the rights to use, copy, modify, merge, publish, distribute, sublicense, and/or sell copies of the Software, and to permit persons to whom the Software is furnished to do so, subject to the following conditions:

%	The above copyright notice and this permission notice shall be included in all copies or substantial portions of the Software.

%	THE SOFTWARE IS PROVIDED "AS IS", WITHOUT WARRANTY OF ANY KIND, EXPRESS OR IMPLIED, INCLUDING BUT NOT LIMITED TO THE WARRANTIES OF MERCHANTABILITY, FITNESS FOR A PARTICULAR PURPOSE AND NONINFRINGEMENT. IN NO EVENT SHALL THE AUTHORS OR COPYRIGHT HOLDERS BE LIABLE FOR ANY CLAIM, DAMAGES OR OTHER LIABILITY, WHETHER IN AN ACTION OF CONTRACT, TORT OR OTHERWISE, ARISING FROM, OUT OF OR IN CONNECTION WITH THE SOFTWARE OR THE USE OR OTHER DEALINGS IN THE SOFTWARE.

%	Email address: echo "cukj -wb- 23wU4X5M589 TROJANS cqkH wiuz2y 0f Mw Stanford" | awk '{ sub("23wU4X5M589","F.d_c_b. ") sub("Stanford","d0mA1n"); print $5, $2, $8; for (i=1; i<=1; i++) print "6\b"; print $9, $7, $6 }' | sed y/kqcbuHwM62z/gnotrzadqmC/ | tr 'q' ' ' | tr -d [:cntrl:] | tr -d 'ir' | tr y "\n"

%%%%%%%%%%%%%%%%%%%%%%%%%%%%%%%%%%%%%%%%%%%%%%



%%%%%%%%%%%%%%%%%%%%%%%%%%%%%%%%%%%%%%%%%%%
\chapter{Data Structures}
\label{chp:DataStructures}


% Check all books with the following keywords in my BibTeX database:
%	data structure analysis
%	data structures



%%%%%%%%%%%%%%%%%%%%%%%%%%%%%%%%%%%%%%%%%%%
\section{Basic Data Structures}
\label{sec:BasicDataStructures}

``[A list is a] container of variable length , and [a tuple is a] container [of] fixed length'' \cite[\S4.3 pp. 111]{Tate2010}.

A list (in {\it Prolog}) can be deconstructed into $[${\it Head} $|$ {\it Tail}$]$, where {\it Head} refers to the first element of the list and {\it Tail} refers to the rest of the list; on the other hand, tuples cannot be similarly deconstructed \cite[\S4.3 pp. 113]{Tate2010}.





%%%%%%%%%%%%%%%%%%%%%%%%%%%%%%%%%%%%%%%%%%%
\section{Graphs}
\label{sec:Graphs}

A graph $G$ is an ordered pair, $G = (V,E)$, of a vertex/node set and an edge set.

Types of finite graphs: \vspace{-0.3cm}
\begin{enumerate} \itemsep -4pt
\item undirected graph: \vspace{-0.3cm}
	\begin{enumerate} \itemsep -2pt
	\item simple graph: \vspace{-0.2cm}
		\begin{enumerate} \itemsep -2pt
		\item Does not allow multiple edges nor loops.
		\item Therefore, the edges of a simple graph form a set, as opposed to multigraphs that have multisets.
		\item An edge is a two-element subset of $V$; other graphs (i.e., multiple graphs) can have more than two nodes.
		\end{enumerate}
	\item hypergraph
	\end{enumerate}
\item directed graph: \vspace{-0.3cm}
	\begin{enumerate} \itemsep -2pt
	\item directed acyclic graphs (DAGs)
	\end{enumerate}
\end{enumerate}


%%%%%%%%%%%%%%%%%%%%%%%%%%%%%%%%%%%%%%%%%%%
\subsection{Graph Representations}
\label{ssec:GraphRepresentations}

Focus on sparse graph representations, which are common in modeling digital integrated circuits and neural networks (certain types), and dense graphs (e.g., neural networks). \\

For sparse graphs, use list or map -based graph representations for better memory efficiency. \\

For dense graphs, use matrix-based graph representation for faster access time at the expense of worse member efficiency. \\

Hence, there exists a trade-off between access time and member efficiency in graph representations.\\

The ways to represent graphs are listed as follows: \vspace{-0.3cm}
\begin{enumerate} \itemsep -4pt
\item adjacency matrix: \vspace{-0.3cm}
	\begin{enumerate} \itemsep -2pt
	\item 
	\end{enumerate}
\item adjacency list: \vspace{-0.3cm}
	\begin{enumerate} \itemsep -2pt
	\item 
	\end{enumerate}
\item adjacency map: \vspace{-0.3cm}
	\begin{enumerate} \itemsep -2pt
	\item 
	\end{enumerate}
\item edge list: \vspace{-0.3cm}
	\begin{enumerate} \itemsep -2pt
	\item Is this equivalent to the ``incidence list'' graph representation? {\Huge Cite this!!!}
	\item 
	\end{enumerate}
\end{enumerate}


Alternate graph representations that I am not exploring: \vspace{-0.3cm}
\begin{enumerate} \itemsep -4pt
\item distance matrix
\item incidence matrix
\end{enumerate}









%%%%%%%%%%%%%%%%%%%%%%%%%%%%%%%%%%%%%%%%%%%
\subsection{Directed Graphs}
\label{ssec:DirectedGraphs}



%%%%%%%%%%%%%%%%%%%%%%%%%%%%%%%%%%%%%%%%%%%
\subsubsection{Functions that need to be implemented}
\label{sssec:FunctionsThatNeedToBeImplemented}




Solvers for the following problems (or to perform the following functions) regarding: \vspace{-0.3cm}
\begin{enumerate} \itemsep -4pt
\item graph coloring: \vspace{-0.3cm}
	\begin{enumerate} \itemsep -2pt
	\item vertex coloring
	\item edge coloring
	\item References: \vspace{-0.2cm}
		\begin{enumerate} \itemsep -2pt
		\item 
		\end{enumerate}
	\end{enumerate}
\item routing problems: \vspace{-0.3cm}
	\begin{enumerate} \itemsep -2pt
	\item shortest path problem: \vspace{-0.2cm}
		\begin{enumerate} \itemsep -2pt
		\item 
		\end{enumerate}
	\item longest path problem: \vspace{-0.2cm}
		\begin{enumerate} \itemsep -2pt
		\item Note that the difficulty of the problem (in terms of computational time complexity) is different for different types of graphs: \vspace{-0.1cm}
			\begin{enumerate} \itemsep -1pt
			\item E.g., for undirected graphs, it is NP-hard, while linear time solutions exist for directed acyclic graphs (DAGs).
			\end{enumerate}
		\item \url{https://en.wikipedia.org/wiki/Longest_path_problem}
		\end{enumerate}
	\item minimum spanning tree: \vspace{-0.2cm}
		\begin{enumerate} \itemsep -2pt
		\item 
		\end{enumerate}
	\item Steiner tree: \vspace{-0.2cm}
		\begin{enumerate} \itemsep -2pt
		\item 
		\end{enumerate}
	\item traveling salesperson problem (NP-hard): \vspace{-0.2cm}
		\begin{enumerate} \itemsep -2pt
		\item 
		\end{enumerate}
	\item strongly connected components: \vspace{-0.2cm}
		\begin{enumerate} \itemsep -2pt
		\item \url{https://en.wikipedia.org/wiki/Strongly_connected_component}
		\end{enumerate}
	\end{enumerate}
\item network flow: \vspace{-0.3cm}
	\begin{enumerate} \itemsep -2pt
	\item max-flow min-cut theorem
	\item maximum flow problems
	\item References: \vspace{-0.2cm}
		\begin{enumerate} \itemsep -2pt
		\item \url{https://en.wikipedia.org/wiki/Maximum_flow_problem}
		\end{enumerate}
	\end{enumerate}
\item graph partitioning: \vspace{-0.3cm}
	\begin{enumerate} \itemsep -2pt
	\item force-directed graph partitioning
	\item min-cut graph partitioning
	\item References: \vspace{-0.2cm}
		\begin{enumerate} \itemsep -2pt
		\item \url{https://en.wikipedia.org/wiki/Connectivity_(graph_theory)}
		\end{enumerate}
	\end{enumerate}
\item graph-based floorplanning/placement: \vspace{-0.3cm}
	\begin{enumerate} \itemsep -2pt
	\item use constraint graphs for graph-based floorplanning/placement
	\item References: \vspace{-0.2cm}
		\begin{enumerate} \itemsep -2pt
		\item \url{https://en.wikipedia.org/wiki/Constraint_graph_(layout)}
		\end{enumerate}
	\end{enumerate}
\item Covering problems: \vspace{-0.3cm}
	\begin{enumerate} \itemsep -2pt
	\item 
	\item : \vspace{-0.2cm}
		\begin{enumerate} \itemsep -2pt
		\item 
		\end{enumerate}
	\end{enumerate}
\end{enumerate}





Solvers for the following problems (or to perform the following functions) regarding subgraphs, induced subgraphs, and minors: \vspace{-0.3cm}
\begin{enumerate} \itemsep -4pt
\item subgraph isomorphism problem: \vspace{-0.3cm}
	\begin{enumerate} \itemsep -2pt
	\item Find a fixed graph as a subgraph in a given graph: \vspace{-0.2cm}
		\begin{enumerate} \itemsep -2pt
		\item ``graph properties are hereditary for subgraphs''\dots\ ``A graph has a property if and only if all its subgraphs also have it''; see \url{https://en.wikipedia.org/wiki/Graph_theory}.
		\item Finding a specific type/kind of maximal subgraph is an NP-complete problem, such as the largest complete subgraph.
		\end{enumerate}
	\end{enumerate}
\item Finding induced subgraphs in a given graph: \vspace{-0.3cm}
	\begin{enumerate} \itemsep -2pt
	\item ``graph properties are hereditary'' for induced subgraphs\dots\ ``A graph has a property if and only if all its induced subgraphs also have it''; see \url{https://en.wikipedia.org/wiki/Graph_theory}.
	\item Finding a specific type/kind of maximal induced subgraph is an NP-complete problem: \vspace{-0.2cm}
		\begin{enumerate} \itemsep -2pt
		\item Independent set problem: Finding the largest edgeless induced subgraph (or independent set); see the following references: \vspace{-0.1cm}
			\begin{enumerate} \itemsep -1pt
			\item \url{https://en.wikipedia.org/wiki/Graph_theory}
			\end{enumerate}
		\end{enumerate}
	\end{enumerate}
\item minor containment problem: \vspace{-0.3cm}
	\begin{enumerate} \itemsep -2pt
	\item Find a fixed graph as a minor of a given graph.
	\item ``A minor or subcontraction of a graph is any graph obtained by taking a subgraph and contracting some (or no) edges''\dots\ ``A graph has a property if and only if all its minors [also] have it''
	\item ``[Minor containment] is related to graph properties such as planarity.'' See Wagner's Theorem about planar graphs.
	\item References: \vspace{-0.2cm}
		\begin{enumerate} \itemsep -2pt
		\item \url{https://en.wikipedia.org/wiki/Graph_theory}
		\end{enumerate}
	\end{enumerate}
\item subdivision containment problems: \vspace{-0.3cm}
	\begin{enumerate} \itemsep -2pt
	\item ``Find a fixed graph as a subdivision of a given graph'': \vspace{-0.2cm}
		\begin{enumerate} \itemsep -2pt
		\item ``A subdivision or homeomorphism of a graph is any graph obtained by subdividing some (or no) edges.''
		\item ``Subdivision containment is related to graph properties such as planarity.'' See Kuratowski's Theorem and the Kelmans-Seymour conjecture about planar graphs.
		\end{enumerate}
	\item References: \vspace{-0.2cm}
		\begin{enumerate} \itemsep -2pt
		\item \url{https://en.wikipedia.org/wiki/Graph_theory}
		\end{enumerate}
	\end{enumerate}
\end{enumerate}

%%%%%%%%%%%%%%%%%%%%%%%%%%%%%%%%%%%%%%%%%%%
\subsubsection{Binary Decision Diagrams (BDDs)}
\label{sssec:BinaryDecisionDiagramsBDDs}




%%%%%%%%%%%%%%%%%%%%%%%%%%%%%%%%%%%%%%%%%%%
\subsubsection{AND-Inverter Graphs (AIGs)}
\label{sssec:ANDInverterGraphsAIGs}













%%%%%%%%%%%%%%%%%%%%%%%%%%%%%%%%%%%%%%%%%%%
\subsection{Undirected Graphs}
\label{ssec:UndirectedGraphs}



















%%%%%%%%%%%%%%%%%%%%%%%%%%%%%%%%%%%%%%%%%
%	Optimization Techniques to be Implemented
%	This is written by Zhiyang Ong to document numerical methods that I have implemented for my C++ -based boilerplate code repository.

%	The MIT License (MIT)

%	Copyright (c) <2014> <Zhiyang Ong>

%	Permission is hereby granted, free of charge, to any person obtaining a copy of this software and associated documentation files (the "Software"), to deal in the Software without restriction, including without limitation the rights to use, copy, modify, merge, publish, distribute, sublicense, and/or sell copies of the Software, and to permit persons to whom the Software is furnished to do so, subject to the following conditions:

%	The above copyright notice and this permission notice shall be included in all copies or substantial portions of the Software.

%	THE SOFTWARE IS PROVIDED "AS IS", WITHOUT WARRANTY OF ANY KIND, EXPRESS OR IMPLIED, INCLUDING BUT NOT LIMITED TO THE WARRANTIES OF MERCHANTABILITY, FITNESS FOR A PARTICULAR PURPOSE AND NONINFRINGEMENT. IN NO EVENT SHALL THE AUTHORS OR COPYRIGHT HOLDERS BE LIABLE FOR ANY CLAIM, DAMAGES OR OTHER LIABILITY, WHETHER IN AN ACTION OF CONTRACT, TORT OR OTHERWISE, ARISING FROM, OUT OF OR IN CONNECTION WITH THE SOFTWARE OR THE USE OR OTHER DEALINGS IN THE SOFTWARE.

%	Email address: echo "cukj -wb- 23wU4X5M589 TROJANS cqkH wiuz2y 0f Mw Stanford" | awk '{ sub("23wU4X5M589","F.d_c_b. ") sub("Stanford","d0mA1n"); print $5, $2, $8; for (i=1; i<=1; i++) print "6\b"; print $9, $7, $6 }' | sed y/kqcbuHwM62z/gnotrzadqmC/ | tr 'q' ' ' | tr -d [:cntrl:] | tr -d 'ir' | tr y "\n"

%%%%%%%%%%%%%%%%%%%%%%%%%%%%%%%%%%%%%%%%%%%%%%



%%%%%%%%%%%%%%%%%%%%%%%%%%%%%%%%%%%%%%%%%%%
\chapter{Optimization}
\label{chp:Optimization}


%%%%%%%%%%%%%%%%%%%%%%%%%%%%%%%%%%%%%%%%%%%
\section{Benchmarks for Optimization}
\label{sec:BenchmarksForOptimization}

A collection of ``optimization solvers'' and benchmarks are available at \cite{Dongarra2016}. \\

Benchmarks for optimization problems: \vspace{-0.3cm}
\begin{enumerate} \itemsep -4pt
\item MIPLIB 2010 -- Mixed Integer Programming Library version 5 \cite{Koch2011a}. See \cite{Achterberg2015a} for publications associated with this set of benchmarks (or benchmark set).
\item 
\end{enumerate}





%%%%%%%%%%%%%%%%%%%%%%%%%%%%%%%%%%%%%%%%%%%
\section{Notes on Using Optimization Tools}
\label{sec:NotesonUsingOptimizationTools}


Optimization problems in EDA can be solved via optimization engines that I implement or external (i.e., third-party) optimization solvers. \\

Regarding external optimization solvers, some of them use {\it Algebraic Modeling Languages (AML)} \cite{WikipediaContributors2015i} to model the optimization problem computationally. These optimization solvers can solve optimization problems that are formulated as computational models in a specific AML representation. \\

{\it I am avoiding the use of external optimization solvers that require paid licenses. Hence, any external optimization solvers that I would use are either open-source software (or rather, free/libre/open-source software, FLOSS) or software that have free academic licenses.} \\

Solvers that use an AML, or several AMLs, in their software interface are: \vspace{-0.3cm}
\begin{enumerate} \itemsep -4pt
\item 
\end{enumerate}

{\bf For a list of optimization solvers/tools, see \S\ref{sec:OptimizationSolvers}.}














%%%%%%%%%%%%%%%%%%%%%%%%%%%%%%%%%%%%%%%%%%%
\section{Robust Linear Programming}
\label{sec:RobustLinearProgramming}


During the ``lab meeting'' on Friday, December 4, 2015, Prof. Jiang Hu told me that I can transform a robust linear programming into a standard/``standard'' linear programming problem. He told me to look at \cite{Bertsimas2004} and its references.




%%%%%%%%%%%%%%%%%%%%%%%%%%%%%%%%%%%%%%%%%%%
\section{Discrete Optimization}
\label{sec:DiscreteOptimization}

Discrete optimization is classified into the following categories \cite{WikipediaContributors2015h,Hammer1979,Lee2004c}: \vspace{-0.3cm}
\begin{enumerate} \itemsep -4pt
\item combinatorial optimization
\item integer programming
\end{enumerate}








%%%%%%%%%%%%%%%%%%%%%%%%%%%%%%%%%%%%%%%%%%%
%\section{Mathematical Programming Solvers}
%\label{sec:MathematicalProgrammingSolvers}
\section{Optimization Solvers}
\label{sec:OptimizationSolvers}

A (brief) description of optimization solvers (not restricted to solvers for mathematical programming), including linear programming solvers, is provided as follows in \S\ref{ssec:AccessibleOptimizationSolvers} and \S\ref{ssec:NotAccessibleOptimizationSolvers}. \\














%%%%%%%%%%%%%%%%%%%%%%%%%%%%%%%%%%%%%%%%%%%
\subsection{Accessible Optimization Solvers}
\label{ssec:AccessibleOptimizationSolvers}

External optimization solvers that are open-source software or provide free academic licenses: \vspace{-0.3cm}
\begin{enumerate} \itemsep -4pt
\item {\it LocalSolver} \cite{Innovation24Staff2015}: \vspace{-0.3cm}
	\begin{enumerate} \itemsep -2pt
	\item Hybrid solver for optimization problems
	\item Properties of the solver \cite[Product: Overview]{Innovation24Staff2015}: \vspace{-0.2cm}
		\begin{enumerate} \itemsep -2pt
		\item ``next-generation, hybrid mathematical programming solver''
		\item solve ``ultra-large real-life nonlinear problems''
		\item solve problems in a ``model-and-run fashion without any tuning''
		\item reliable and robust solver: {\bf Define reliability and robustness for solvers of optimization problems.}
		\item dynamically combines solutions from various optimization approaches and resolves them via a hybrid neighborhood search approach
		\item solver engines: \vspace{-0.1cm}
			\begin{enumerate} \itemsep -1pt
			\item ``local search techniques''
			\item ``constraint propagation techniques''
			\item ``inference techniques''
			\item linear programming solver/techniques
			\item mixed-integer programming solver/techniques, including mixed-integer linear programming (MILP) solver/techniques
			\item nonlinear programming solver/techniques
			\item combined pure and direct local search techniques
			\end{enumerate}
		\item is based on the {\it LocalSolver Programming language} (LSP) for mathematical modeling
		\item has lightweight object-oriented APIs
		\end{enumerate}
	\item From \cite[Support Center: Example tour]{Innovation24Staff2015}, {\it LocalSolver} can solve continuous and discrete/combinatorial optimization problems: \vspace{-0.2cm}
		\begin{enumerate} \itemsep -2pt
		\item continuous optimization problems: \vspace{-0.1cm}
			\begin{enumerate} \itemsep -1pt
			\item minimization of the Branin function: find the minimal point of the Branin function, within a specified domain
			\item optimal bucket design: minimization of a bucket encapsulating/covering the rod/cylinder
			\item Steel mill slab design: mathematical programming
			\end{enumerate}
		\item discrete optimization problems: \vspace{-0.1cm}
			\begin{enumerate} \itemsep -1pt
			\item car sequencing: scheduling problem, or assignment problem.
			\item Flowshop: scheduling problem
			\item knapsack problem
			\item max-cut problem
			\item Quadratic Assignment Problem (QAP)
			\item Steel mill slab design: integer programming
			\item Travelling salesman problem
			\item Vehicule routing problem
			\end{enumerate}
		\end{enumerate}
	\item Its technical documentation can be found at: \url{http://www.localsolver.com/documentation/index.html} \cite{Innovation24Staff2015a}. 
	\end{enumerate}
\item Stanford University: \vspace{-0.3cm}
	\begin{enumerate} \itemsep -2pt
	\item Systems Optimization Laboratory researchers, ``SOL Optimization Software,'' from {\it Stanford University: School of Engineering: Department of Management Science and Engineering: Systems Optimization Laboratory}, Stanford, CA, 2015. Available online at: \url{http://web.stanford.edu/group/SOL/download.html}; last accessed on December 14, 2015. \vspace{-0.2cm}
		\begin{enumerate} \itemsep -2pt
		\item ``Iterative solvers for sparse $Ax = b$: SYMMLQ, MINRES, MINRES-QLP, cgLanczos, CRAIG''
		\item ``Iterative solvers for sparse least-squares problems: LSQR, LSMR, CGLS, covLSQR, LSRN''
		\item ``Sparse and dense LU factorization (direct methods): LUSOL, LUMOD''
		\item ``Sparse optimization: ASP''
		\item ``Optimization with convex objective and linear constraints: PDCO (including sparse optimization)''
		\item ``Convex optimization in composite form: PNOPT''
		\item ``Fortran 90 quad-precision dotproduct of double-precision vectors: qdotdd''
		\end{enumerate}
%	\item Systems Optimization Laboratory researchers, ``SOL Optimization Software,'' from {\it Systems Optimization Laboratory, Department of Management Science and Engineering, School of Engineering, Stanford University}, Stanford, CA. Available online at: \url{http://web.stanford.edu/group/SOL/download.html}; last accessed on December 14, 2015.
	\end{enumerate}
\item 
\item 
\item 
\item 
\item 
\item 
\item 
\item 
\item 
\item 
\item 
\item 
\item 
\item 
\end{enumerate}

Additional notes: \vspace{-0.3cm}
\begin{enumerate} \itemsep -4pt
\item For mixed-integer programming, check performance comparisons on {MIPLIB} benchmarks (\url{http://www.localsolver.com/news.html?id=32})
\end{enumerate}
































%%%%%%%%%%%%%%%%%%%%%%%%%%%%%%%%%%%%%%%%%%%
\subsection{Not Accessible Optimization Solvers}
\label{ssec:NotAccessibleOptimizationSolvers} 

External optimization solvers that require paid licenses: \vspace{-0.3cm}
\begin{enumerate} \itemsep -4pt
\item Stanford University: \vspace{-0.3cm}
	\begin{enumerate} \itemsep -2pt
	\item Systems Optimization Laboratory researchers, ``SOL Optimization Software,'' from {\it Stanford University: School of Engineering: Department of Management Science and Engineering: Systems Optimization Laboratory}, Stanford, CA, 2015. Available online at: \url{http://web.stanford.edu/group/SOL/download.html}; last accessed on December 14, 2015. \vspace{-0.2cm}
		\begin{enumerate} \itemsep -2pt
		\item LSSOL
		\item MINOS
		\item NPSOL
		\item QPOPT
		\item SNOPT
		\item SQOPT
		\end{enumerate}
%	\item Systems Optimization Laboratory researchers, ``SOL Optimization Software,'' from {\it Systems Optimization Laboratory, Department of Management Science and Engineering, School of Engineering, Stanford University}, Stanford, CA. Available online at: \url{http://web.stanford.edu/group/SOL/download.html}; last accessed on December 14, 2015.
	\end{enumerate}
\item 
\item 
\item 
\item 
\item 
\item 
\item 
\item 
\item 
\item 
\item 
\item 
\item 
\item 
\end{enumerate}















%%%%%%%%%%%%%%%%%%%%%%%%%%%%%%%%%%%%%%%%%
%	Numerical Methods to be Implemented
%	This is written by Zhiyang Ong to document numerical methods that I have implemented for my C++ -based boilerplate code repository.

%	The MIT License (MIT)

%	Copyright (c) <2014> <Zhiyang Ong>

%	Permission is hereby granted, free of charge, to any person obtaining a copy of this software and associated documentation files (the "Software"), to deal in the Software without restriction, including without limitation the rights to use, copy, modify, merge, publish, distribute, sublicense, and/or sell copies of the Software, and to permit persons to whom the Software is furnished to do so, subject to the following conditions:

%	The above copyright notice and this permission notice shall be included in all copies or substantial portions of the Software.

%	THE SOFTWARE IS PROVIDED "AS IS", WITHOUT WARRANTY OF ANY KIND, EXPRESS OR IMPLIED, INCLUDING BUT NOT LIMITED TO THE WARRANTIES OF MERCHANTABILITY, FITNESS FOR A PARTICULAR PURPOSE AND NONINFRINGEMENT. IN NO EVENT SHALL THE AUTHORS OR COPYRIGHT HOLDERS BE LIABLE FOR ANY CLAIM, DAMAGES OR OTHER LIABILITY, WHETHER IN AN ACTION OF CONTRACT, TORT OR OTHERWISE, ARISING FROM, OUT OF OR IN CONNECTION WITH THE SOFTWARE OR THE USE OR OTHER DEALINGS IN THE SOFTWARE.

%	Email address: echo "cukj -wb- 23wU4X5M589 TROJANS cqkH wiuz2y 0f Mw Stanford" | awk '{ sub("23wU4X5M589","F.d_c_b. ") sub("Stanford","d0mA1n"); print $5, $2, $8; for (i=1; i<=1; i++) print "6\b"; print $9, $7, $6 }' | sed y/kqcbuHwM62z/gnotrzadqmC/ | tr 'q' ' ' | tr -d [:cntrl:] | tr -d 'ir' | tr y "\n"

%%%%%%%%%%%%%%%%%%%%%%%%%%%%%%%%%%%%%%%%%%%%%%



%%%%%%%%%%%%%%%%%%%%%%%%%%%%%%%%%%%%%%%%%%%
\chapter{Mathematics}
\label{chp:Mathematics}


%%%%%%%%%%%%%%%%%%%%%%%%%%%%%%%%%%%%%%%%%%%
%\section{Mathematics}
%\label{chp:Mathematics}


Math symbols that I use frequently: \vspace{-0.3cm}
\begin{enumerate} \itemsep -4pt
\item $\mathbb{N}$
\item $\displaystyle\sum^{i = 1}_{n}$
\item $f(x) = \displaystyle\lim_{n \rightarrow \infty} \frac{f(x)}{g(x)}$
\item $\varnothing$
\item $q$
\end{enumerate}

A $3 \times 3$ matrix:
$\left(
\begin{array}{ccc}
	11 & 12 & 13 \\
	21 & 22 & 23 \\
	31 & 32 & 33
\end{array}
\right)$
\ \\
\ \\

Here is an equation:
\begin{equation}
\label{eqn:myeqnexample}
\iint_{\Sigma} \nabla \times \mathbf{F} \cdot \mathrm{d}\mathbf{\Sigma} = \oint_{\partial\Sigma} \mathbf{F} \cdot \mathrm{d} \mathbf{r}.
\end{equation}
\ \\
\ \\

Here is an equation that is not numbered.
\begin{equation*}
\nabla \times \mathbf{E} = -\frac{\partial \mathbf{B}} {\partial t}
\end{equation*}



Here is the set of Maxwell's equations that is numbered.
\begin{gather}
	\nabla \cdot \mathbf{E} = \frac {\rho} {\varepsilon_0} \\
	\nabla \cdot \mathbf{B} = 0 \\
	\nabla \times \mathbf{E} = -\frac{\partial \mathbf{B}} {\partial t} \\
	\nabla \times \mathbf{B} = \mu_0\left(\mathbf{J} + \varepsilon_0 \frac{\partial \mathbf{E}} {\partial t} \right)
\end{gather}


\begin{gather*}
	{\rm minimize \displaystyle\sum^{c}_{i = 1} c_{i} \cdot x_{i}} \\	%	objective function defined mathematically	\\
	\underline{x} \in S \\
	{\rm subject\ to:} \\
	%	constraints	\\
	x_{1} + x_{4} = 0 \\
	x_{3} + 7 \cdot x_{4} + 2\cdot x_{9} = 0
\end{gather*}


\begin{equation}
\label{eqn:caseenv}
f(n) = 
	\begin{cases}
	case-1 &: \mathrm{n\ is\ odd} \\
	case-2 &: \mathrm{n\ is\ even} \\
	\end{cases}
\end{equation}

\begin{proof}
This is a proof for BLAH \dots
\end{proof}




\begin{theorem}{TITLE of theorem.}
My theorem is\dots
\end{theorem}



\begin{axiom}{TITLE of axiom.}
Blah\dots
\end{axiom}



Cases of putting a bracket/parenthesis on the right side of the equation.
\begin{gather*}
	\left.\begin{aligned}
	B'&=-\partial \times E,\\
	E'&=\partial \times B - 4\pi j,
	\end{aligned}
	\right\}
	\quad\text{Maxwell's equations}
\end{gather*}


%Cases of putting a bracket/parenthesis on the right side of the equation.\\
%$\begin{rcases*}
%	E = m c^2 & foo \\
%	\int x-3\, dx & barbaz
%\end{rcases*} y=f(x)$
\ \\
\ \\

Labeling an arrow: $\xrightarrow{ewq}$





%%%%%%%%%%%%%%%%%%%%%%%%%%%%%%%%%%%%%%%%%
%	Statistical Computation Methods to be Implemented
%	This is written by Zhiyang Ong to document methods for statistical computation that I have implemented for my C++ -based boilerplate code repository.

%	The MIT License (MIT)

%	Copyright (c) <2014> <Zhiyang Ong>

%	Permission is hereby granted, free of charge, to any person obtaining a copy of this software and associated documentation files (the "Software"), to deal in the Software without restriction, including without limitation the rights to use, copy, modify, merge, publish, distribute, sublicense, and/or sell copies of the Software, and to permit persons to whom the Software is furnished to do so, subject to the following conditions:

%	The above copyright notice and this permission notice shall be included in all copies or substantial portions of the Software.

%	THE SOFTWARE IS PROVIDED "AS IS", WITHOUT WARRANTY OF ANY KIND, EXPRESS OR IMPLIED, INCLUDING BUT NOT LIMITED TO THE WARRANTIES OF MERCHANTABILITY, FITNESS FOR A PARTICULAR PURPOSE AND NONINFRINGEMENT. IN NO EVENT SHALL THE AUTHORS OR COPYRIGHT HOLDERS BE LIABLE FOR ANY CLAIM, DAMAGES OR OTHER LIABILITY, WHETHER IN AN ACTION OF CONTRACT, TORT OR OTHERWISE, ARISING FROM, OUT OF OR IN CONNECTION WITH THE SOFTWARE OR THE USE OR OTHER DEALINGS IN THE SOFTWARE.

%	Email address: echo "cukj -wb- 23wU4X5M589 TROJANS cqkH wiuz2y 0f Mw Stanford" | awk '{ sub("23wU4X5M589","F.d_c_b. ") sub("Stanford","d0mA1n"); print $5, $2, $8; for (i=1; i<=1; i++) print "6\b"; print $9, $7, $6 }' | sed y/kqcbuHwM62z/gnotrzadqmC/ | tr 'q' ' ' | tr -d [:cntrl:] | tr -d 'ir' | tr y "\n"

%%%%%%%%%%%%%%%%%%%%%%%%%%%%%%%%%%%%%%%%%%%%%%



%%%%%%%%%%%%%%%%%%%%%%%%%%%%%%%%%%%%%%%%%%%
\chapter{Statistics}
\label{chp:Statistics}


%%%%%%%%%%%%%%%%%%%%%%%%%%%%%%%%%%%%%%%%%%%
%\section{Mathematics}
%\label{chp:Mathematics}







%%%%%%%%%%%%%%%%%%%%%%%%%%%%%%%%%%%%%%%%%
%	C++ and C++ STL Notes and Resources
%	This is written by Zhiyang Ong to document usage of C++ STL features that I can use for my C++ -based software, particularly those in electronic design automation.

%	The MIT License (MIT)

%	Copyright (c) <2014> <Zhiyang Ong>

%	Permission is hereby granted, free of charge, to any person obtaining a copy of this software and associated documentation files (the "Software"), to deal in the Software without restriction, including without limitation the rights to use, copy, modify, merge, publish, distribute, sublicense, and/or sell copies of the Software, and to permit persons to whom the Software is furnished to do so, subject to the following conditions:

%	The above copyright notice and this permission notice shall be included in all copies or substantial portions of the Software.

%	THE SOFTWARE IS PROVIDED "AS IS", WITHOUT WARRANTY OF ANY KIND, EXPRESS OR IMPLIED, INCLUDING BUT NOT LIMITED TO THE WARRANTIES OF MERCHANTABILITY, FITNESS FOR A PARTICULAR PURPOSE AND NONINFRINGEMENT. IN NO EVENT SHALL THE AUTHORS OR COPYRIGHT HOLDERS BE LIABLE FOR ANY CLAIM, DAMAGES OR OTHER LIABILITY, WHETHER IN AN ACTION OF CONTRACT, TORT OR OTHERWISE, ARISING FROM, OUT OF OR IN CONNECTION WITH THE SOFTWARE OR THE USE OR OTHER DEALINGS IN THE SOFTWARE.

%	Email address: echo "cukj -wb- 23wU4X5M589 TROJANS cqkH wiuz2y 0f Mw Stanford" | awk '{ sub("23wU4X5M589","F.d_c_b. ") sub("Stanford","d0mA1n"); print $5, $2, $8; for (i=1; i<=1; i++) print "6\b"; print $9, $7, $6 }' | sed y/kqcbuHwM62z/gnotrzadqmC/ | tr 'q' ' ' | tr -d [:cntrl:] | tr -d 'ir' | tr y "\n"

%%%%%%%%%%%%%%%%%%%%%%%%%%%%%%%%%%%%%%%%%%%%%%



%%%%%%%%%%%%%%%%%%%%%%%%%%%%%%%%%%%%%%%%%%%
\chapter{C++ Resources}
\label{chp:CppResources}

	Quick advice: {\tt Learn how to use {\it C++1y} features, including those of {C++11}, {C++14}, and {C++17}.} Older {\it C++} versions include {\it C++98} and {\it C++03}. \\

	Reference: Free Software Foundation contributors, ``{C++1y/C++14} Support in {GCC},'' from {\it {GCC}, the {GNU} Compiler Collection: {GCC} Projects}, Free Software Foundation, Boston, MA, November 14, 2015. Available online at: \url{https://gcc.gnu.org/projects/cxx1y.html}; last accessed on January 25, 2016. \\
	
	{\Huge Add references for this chapter!} \\
\ \\

Books/references that cover the following modern {\it C++} standards: \vspace{-0.3cm}
\begin{enumerate} \itemsep -4pt
\item {\it C++11}: \vspace{-0.3cm}
	\begin{itemize} \itemsep -2pt
	\item \cite{Wikibookscontributors2014}
	\item \cite{Stroustrup2014}
	\item \cite{Stroustrup2014a}
	\item \cite{Bronson2013}
	\item \cite{Carrano2013}
	\item \cite{Langr2013}
	\item \cite{Lippman2013}
	\item \cite{Lischner2013}
	\item \cite{Malik2013}
	\item \cite{Malik2013a}
	\item \cite{Olsson2013}
	\item \cite{Overland2013}
	\item \cite{Savitch2013}
	\item \cite{Seacord2013}
	\item \cite{Stroustrup2013}
	\item \cite{Zak2013}
	\item \cite{ISOIECJTC1SC22WG21Members2012}
	\item \cite{Allain2012}
	\item \cite{Bronson2012}
	\item \cite{Deitel2012}
	\item \cite{Gaddis2012}
	\item \cite{Green2012}
	\item \cite{Groff2012}
	\item \cite{Horstmann2012}
	\item \cite{Josuttis2012}
	\item \cite{McLaughlin2012}
	\item \cite{Prata2012}
	\item \cite{Savitch2012}
	\item \cite{VibrantPublishers2012a}
	\item \cite{Becker2011}
	\item \cite{Gregoire2011}
	\end{itemize}
\item {\it C++14}: \vspace{-0.3cm}
	\begin{itemize} \itemsep -2pt
	\item \cite{Gottschling2016}
	\item \cite{Guntheroth2016}
	\item \cite{ISOIECJTC1SC22WG21Members2016}
	\item \cite{Matloff2016}
	\item \cite{Mohanty2016}
	\item \cite{DiGennaro2015}
	\item \cite{Malik2015}
	\item \cite{Malik2015a}
	\item \cite{Meyers2015} -- Important.
	\item \cite{Sutherland2015}
	\item \cite{Dale2014}
	\item \cite{Deitel2014}
	\item \cite{Fog2014c}
	\item \cite{Gregoire2014}
	\item \cite{Horton2014}
	\item \cite{Sutherland2014}
	\item \cite{Sutherland2014a}
	\item \cite{DuToit2013}
	\end{itemize}
\item {\it C++17}: \vspace{-0.3cm}
	\begin{itemize} \itemsep -2pt
	\item \cite{Malik2018}
	\item \cite{Deitel2017}
	\item \cite{ISOIECJTC1SC22WG21Members2017}
	\item \cite{Mohanty2017}
	\item \cite{Rao2017}
	\item \cite{Roth2017}
	\item \cite{Smith2017b}
	\end{itemize}
\item Other books/references that may over modern {\it C++}: \vspace{-0.3cm}
	\begin{itemize} \itemsep -2pt
	\item \cite{Barlas2015}
	\item \cite{Oliveira2015}
	\item \cite{Dawson2014}
	\item \cite{Pena2014}
	\item \cite{Stevanovic2014}
	\item \cite{Weiss2014}
	\item \cite{Basalaj2013}
	\item \cite{Dale2013}
	\item \cite{Drozdek2013}
	\item \cite{ProgrammingResearchLtdStaff2013}
	\item \cite{Downey2012}
	\item \cite{Meyers2012}
	\item \cite{PittFrancis2012}
	\item \cite{Shapira2012}
	\item \cite{Williams2012a}
	\item \cite{Dawson2011}
	\item \cite{DiGennaro2011}
	\item \cite{EdisonDesignGroup2011}
	\item \cite{Gaddis2011}
	\item \cite{Main2011}
	\item \cite{Malik2011}
	\item \cite{Malik2011a}
	\item \cite{Overland2011}
	\end{itemize}
\item Other helpful resources for {\it C++}: \vspace{-0.3cm}
	\begin{itemize} \itemsep -2pt
	\item \cite{Eddelbuettel2013} used with {\it R}
	\item \cite{Laakmann2009}
	\item \cite{Mintz2006}
	\end{itemize}
\end{enumerate}








%%%%%%%%%%%%%%%%%%%%%%%%%%%%%%%%%%%%%%%%%%%
\section{Resources for C++ and Notes About C++}
\label{sec:ResourcesAndNotesAboutCpp}



Some {\it C++} resources are: \vspace{-0.3cm}
\begin{enumerate} \itemsep -4pt
\item Online {\it C++} tutorial reference: \cite{CplusplusCom2015a,CplusplusCom2015b,CplusplusCom2014,Soulie2007}: \url{http://www.cplusplus.com/reference/stl/}
%	\item \cite{CplusplusCom2014}, \cite{CplusplusCom2015}, \cite{CplusplusCom2015a}, \cite{CplusplusCom2015b}, and \cite{Soulie2007}: \url{http://www.cplusplus.com/reference/stl/}
\item Pointers to functions \cite{CplusplusCom2015b}: \url{http://www.cplusplus.com/doc/tutorial/pointers/}
\item An online {\it C++} reference: \cite{cppreference2015}.
\item From an online {\it Marshall Cline} resource: \cite{Cline2011,Cline2003,Cline2000}
\item Books on {\it C++}: \cite{Pozrikidis2007,Koenig2000}; processed for {\it C++} templates
\item Embedded {\it C++}: \cite{Katupitiya2006}
\item GUI and other graphics with {\it C++}: \cite{Pozrikidis2007}
\item Summaries of {\it C++}: \cite{Josuttis1999} 
\end{enumerate}


Some {\it C++} STL resources are: \vspace{-0.3cm}
\begin{enumerate} \itemsep -4pt
\item Online {\it C++} STL reference: \cite{CplusplusCom2015}: \url{http://www.cplusplus.com/reference/stl/}
\item \cite{cppreference2015}: \url{http://en.cppreference.com/w/cpp/container}
\item From an online {\it C++} STL resource from SGI (formerly, Silicon Graphics, Inc.): \cite{HewlettPackardCompanyStaff2014,HewlettPackardCompanyStaff1994}.
\item Another online {C++} STL reference: \cite{Mohtashim2015a}: \url{http://www.tutorialspoint.com/cplusplus/cpp_stl_tutorial.htm}
\item Other online {\it C++} STL references: \cite{Riesbeck2009,Riesbeck2009a}.
\item \url{http://www.cs.wustl.edu/~schmidt/PDF/stl4.pdf}
\end{enumerate}

Books to classify: \vspace{-0.3cm}
\begin{enumerate} \itemsep -4pt
\item C++ programming: \cite{Horstmann2012,Savitch2009,Prata2005,Romanik2003,Schildt2003a,Schildt1998a}
%	Finished going through: Horstmann2012,Savitch2009,Prata2005,Romanik2003,Schildt2003a,Schildt1998a
\end{enumerate}



References for {\it C++} libraries of interest: \vspace{-0.3cm}
\begin{enumerate} \itemsep -4pt
\item {\it Boost C++}: \cite{Mukherjee2015,Polukhin2013,Schaling2012,Karlsson2006a}
\item {\it Other libraries that can be considered include:  cpp-netlib: The C++ Network Library; nana; FLTK; gtkmm; Qt; evince; glibmm; cairomm; opencv; Ogre3D; OpenGL; CGAL; QuantLib; Google Test; and cppunit.}
\item E.g., see \url{http://en.cppreference.com/w/cpp/links/libs} \cite[Useful resources: List of C++ libraries -- A list of open source C++ libraries]{cppreference2015}.
\end{enumerate}



C++ topics: \vspace{-0.3cm}
\begin{enumerate} \itemsep -4pt
\item Function objects: \vspace{-0.3cm}
	\begin{enumerate} \itemsep -2pt
	\item \url{https://en.wikipedia.org/wiki/Functional_(C%2B%2B)}
	\item \url{http://stackoverflow.com/questions/356950/c-functors-and-their-uses}
	\item \url{http://www.cprogramming.com/tutorial/functors-function-objects-in-c++.html}
	\item \cite[pp. 233--243]{Josuttis2012}
	\item \cite[pp. 327--332, 885, 922--931, 947]{Prata2005}
	\item \cite[pp. 126--129]{Schildt1998a}. Function pointers are pointers to functions; note that these functions are not variables.
%	Finding references for function objects is not a top priority.
	\end{enumerate}
%%%%%%%%%%%%%%%%%%%%%%%%%%%%%%%%%%%%%%%%%%%%%
\item Strings: \vspace{-0.3cm}
	\begin{enumerate} \itemsep -2pt
	\item \cite{Stroustrup2014}, Chp 23
	\item \cite{Stroustrup2009}, Chp 23
	%%%%%%%%%%%%%%%%%%%%%%%%%%%%%%%%%
	\item \cite{Gregoire2014}, Chp 18
	\item \cite{Allain2012}, Chp 19
	\item \cite[pp. 56--60, string data types, variable vs. literal; strings and string class, 13, 82--87, 320--324, 363--365, 496]{Horstmann2012}
	\item \cite[655--716]{Josuttis2012}
	\item \cite[pp. 64--67, 320--325, 465--482]{Savitch2009}
	\item \cite[\S14.2]{Scheinerman2006}
	\item \cite[pp. 114--131]{Prata2005}
	\item \cite{Eckel2003}, Chp 1
	\item \cite{Heller2003}: \vspace{-0.2cm}
		\begin{enumerate} \itemsep -2pt
		\item The {\tt put} pointer points ``to the next free byte in the {\tt stringstream}.'' That is, it ``holds the address of the next byte in the output area of the'' {\tt stringstream}. When the {\tt stringstream} is empty, the {\tt put} pointer points to the beginning of the {\tt stringstream} buffer. \cite[\S9.8]{Heller2003}.
		\item ``The type of the {\tt put} pointer'' does not matter to the software developer(s), since they ``cannot access it directly'' \cite[\S9.8]{Heller2003}.
		\item ``The {\tt get} pointer holds the address of the next byte in the input area of the stream, or the next byte we get if we use $>>$ to read data from the {\tt stringstream}'' \cite[\S9.9]{Heller2003}.
		\item ``The {\tt end} pointer indicates the end of the {\tt stringstream}. Attempting to read anything at or after this position will cause the read to fail because there is nothing else to read'' \cite[\S9.9]{Heller2003}.
		\item Developers only have to know about how {\tt put}, {\tt get}, and {\tt end} pointers work. They do not have to know the actual representation of these pointers \cite[\S9.9]{Heller2003}.
		\item The {\tt stringstream} object acts as a buffer, and is ``an area of allocated memory'' (``by the stringstream member functions'') \cite[\S9.9]{Heller2003}.
		\end{enumerate}
	\item \cite{McMahon20XY}: \vspace{-0.2cm}
		\begin{enumerate} \itemsep -2pt
		\item C strings (or C-strings, or C-style strings) are null-terminated strings (arrays of characters that each end with a terminating ``null character'' with ASCII value 0) and are arrays of characters; the ``null character'' is usually represented by the literal character '$\backslash$0'. ``However, an array of {\tt char} is NOT by itself a C string.''
		\item ``Since {\tt char} is a built-in data type, no header file is required to create a C string. The C library header file $<${\tt cstring}$>$ contains a number of utility functions that operate on C strings.''
		\item ``It is also possible to declare a C string as a pointer to a {\tt char}: char$^{\ast}$ s3 = ``hello''; '' It creates a character array with just enough memory space (in the heap) to store the null-terminated string. The address of the string's first character is placed in the {\tt char} pointer {\it s3}. When this improperly used, it can corrupt program memory or cause run-time errors.
		\item ``[Use] the {\tt C} library function {\tt strlen()}'' to determine ``the length of a {\tt C} string.'' It returns an unsigned integer representing the number of characters in the string, excluding the terminating null character.
		\item Relational operators (such as $==, !=, >, <, >=, <=$) compare the addresses of the first characters in the two string operands (as the array names are treated as pointers), instead of the contents of these strings.
		\item ``Use the {\tt C} library function {\tt strcmp()}'' ``to compare the contents of two {\tt C} strings.'' The input arguments of this function are two pointers to {\tt C} strings.
		\item ``Use the {\tt C} library function {\tt strcpy()}'' to assign a string to a C string or change its contents. The {\tt strcpy()} function accepts a pointer to the C string as the first input argument, and a pointer to the contents of a valid C string or string literal (i.e., a character) as the second input argument. The {\tt C} library function {\tt strcat()} has the same input arguments as {\tt strcpy()}, and is used for concatenating two strings.
		\item C strings can be used as input parameters or the return type. They are specified as {\tt char[]} or {\tt char$^{\ast}$}.
		\item ``A {\tt C++} string is an object of the class string, which is defined in the header file $<$string$>$ and which is in the standard namespace.'' The variable name of a {\tt C++} string is a pointer to the first character of the string; the variable name contains the address of the string's first character. The {\tt C++} string is a dynamically-allocated array of characters.
		\item ``[Use] the string class methods {\tt length()} or {\tt size()}'' to determine ``the length of the {\tt C++} strings.''
		\item To improve memory efficiency and reduce memory usage, explicitly {\it pass a string object}. Else, the {\tt C++} string objects are pass and returned by value, which involves making a copy of the string object.
		\item Concatenate C++ strings, C strings, and string literals in any order using the ``+'' operator.
		\item Convert a C++ string into a C string via the {\tt c\_str()} function of the {\tt string} class. The {\tt c\_str()} function returns a pointer to the array of characters representing the string. If the C++ string is not null-terminated, a null character is appended to the new C string. The returned C string ``can be used, printed, copied, etc.'' but not be modified.
		\item Since programming with arrays can enbug the code more easily, the use of C++ {string} is (strongly) recommended for use. This is because the properties of a2
		\item When a C string is required by a function, convert the C++ string into a C string (as aforementioned). Instances in which a C string have to be converted into a C++ string are: \vspace{-0.1cm}
			\begin{enumerate} \itemsep -1pt
			\item Strings passed into {\tt main()} as C strings from the command line argument.
			\item Functions for file input/output operations require filenames to be specified as C strings.
			\item The C++ string class does not have the equivalent functions of certain C string library functions.
			\item Unlike C++ strings, C strings can be serialized in binary format without requiring a bunch of extra code to be written.
			\end{enumerate}
		\item The function {\tt atoi} converts a string to an integer. Similar functions for converting strings into numbers are: {\tt atol} and {\tt atof}. The C++ STL does not have a {\tt itoa} function to convert a number to an integer. However, some compilers supports this function in the {\it C Standard General Utilities Library}.
		\end{enumerate}
	%%%%%%%%%%%%%%%%%%%%%%%%%%%%%%%%%%%%%%%%%
	\item The function {\tt strtol} converts a string into a long integer: \vspace{-0.2cm}
		\begin{enumerate} \itemsep -2pt
		\item See \url{http://www.cplusplus.com/reference/cstdlib/strtol/}.
		\item \cite[$<$cstdlib$>$ (stdlib.h) -- C Standard General Utilities Library: {\tt strtol} function]{CplusplusCom2015}
		\end{enumerate}
	\item Danny Kalev, ``String Streams,'' in {\it {InformIT}: The Trusted Technology Learning Source: Articles: Programming: {C/C++} Articles: {InformIT C++} Reference Guide}, Pearson Education, Indianapolis, IN, January 1, 2003. Available online at: \url{http://www.informit.com/guides/content.aspx?g=cplusplus&seqNum=72}; last accessed on November 13, 2015. \vspace{-0.2cm}
		\begin{enumerate} \itemsep -2pt
		\item Static buffers (via {\it atoi()}, {\it sprintf()}, or {\it sscanf()} from the {\it $<$stdio.h$>$}) for type conversions can cause buffer overflow and do not provide adequate type safety (i.e., adequate type checking mechanism). This can be mitigated via {\it stringstreams}.
		\end{enumerate}
	\end{enumerate}
%%%%%%%%%%%%%%%%%%%%%%%%%%%%%%%%%%%%%%%%%%%%%
\item IO Streams: \vspace{-0.3cm}
	\begin{enumerate} \itemsep -2pt
	\item \cite{Gregoire2014}, Chp 12
	\item \cite{Stroustrup2014}, Chp 10-11
	\item \cite{Lippman2013}, Chp 8
	\item \cite{Allain2012}, Chp 28
	\item \cite[chp. 12]{Gaddis2012}.
	\item \cite[Chp. 8, pp. 351--388; pp. 4, 12--15, 49--51, 150--154, 352, 361--365, 497--498]{Horstmann2012}
	\item \cite[pp. 743--847]{Josuttis2012}
	\item \cite{Prata2012}, Chp 17q
	\item \cite[Chp. 13]{Gaddis2011}.
	\item \cite{Gaddis2010}, Chp 12.
	\item \cite[\S2.2; Chp. 6]{Savitch2009}
	\item \cite{Stroustrup2009}, Chp 10-11
	\item \cite[\S14.3-14.8]{Scheinerman2006}
	\item \cite[pp. 262--273]{Prata2005}
	\item \cite{Eckel2003}, Chp 2
	\item \cite{Oualline2003}, Chp 16
	\item \cite{Schildt2003}, Chp 21
	\item \cite[Chp. 18, pp. 417--450; Appendix A, pp. 563--580]{Schildt2003a}
	\item \cite{Vermeir2001}, Chp 10
	\item \cite[Chp. 8--9, pp. 187--235; Chp. 20--21, pp. 511--568]{Schildt1998a}
	\end{enumerate}
%%%%%%%%%%%%%%%%%%%%%%%%%%%%%%%%%%%%%%%%%%%%%
\item Templates: \vspace{-0.3cm}
	\begin{enumerate} \itemsep -2pt
	\item \cite{Gregoire2014}, Chp 11,21
	\item \cite{Stroustrup2014}, Chp 19
	\item \cite{Lippman2013}, Chp 16
	\item \cite{Allain2012}, Chp 29
	\item \cite[\S16.2--\S16.4]{Gaddis2012}
	\item \cite[pp. 13, 26--27, 33--34, 36, 62, 68, 1024]{Josuttis2012}
	\item \cite[\S16.2--\S16.4]{Gaddis2011}
	\item \cite[Chp. 17]{Savitch2009}
	\item \cite{Stroustrup2009}, Chp 19
	\item \cite[\S6.16, pp. 193--195]{Pozrikidis2007}
	\item \cite{Abrahams2005}, book
	\item \cite[pp. --]{Prata2005}
	\item \cite{Eckel2003}, Chp 3
	\item \cite{Oualline2003}, Chp 24
	\item \cite{Schildt2003}, Chp 18
	\item \cite[Chp. 16, pp. 375--394]{Schildt2003a}
	\item \cite{Vandevoorde2003}, book
	\item \cite{Alexandrescu2001}, book; typelist - Chp 3
	\item \cite{Vermeir2001}, Chp 6
	\item \cite{Eckel2000}, Chp 16
	\item \cite[Chp. 18, pp. 461--487]{Schildt1998a}
	\item References: \vspace{-0.2cm}
		\begin{enumerate} \itemsep -2pt
		\item \cite{Moses2016} recommends including the implementation file ({\tt .cpp}), instead of the header file ({\tt .hpp}) in any {\it C++} file that uses the {\it C++} template.
		\item \cite{Ben2013} argues for separating the {\it C++} template definitions in the header file ({\tt .hpp}) from the {\it C++} template implementations in the implementation file ({\tt .cpp}).
		\item ``Although standard C++ has no such requirement, some compilers require that all function and class templates need to be made available in every translation unit they are used. In effect, for those compilers, the bodies of template functions must be made available in a header file. To repeat: that means those compilers won't allow them to be defined in non-header files such as .cpp files.\dots\ There is an export keyword which is supposed to mitigate this problem, but it's nowhere close to being portable.'' \cite{Gogolev2009}
		\item \cite{Benoit2009} argues that you can merge the definition and implementation in a single file, which ends with the {\tt .hpp} file extension.
		\item \cite{DevSolar2009,Diago2013} argue that both the definition and the implementation of {\it C++} templates should be placed together in the {\it C++} header file, with white space 
		\item This is a list of different implementations of a C++ template: \vspace{-0.1cm}
			\begin{enumerate} \itemsep -1pt
			\item Implemented with {\it C++} template definition and implementation in one file, where the definition and implementation are clearly separated/distinguished/demarcated. The implementation is appended to the end of the definition, with some white space to separate them \cite{Gogolev2009,DevSolar2009,Diago2013}.
			\item Implemented with merged {\it C++} template definition and implementation in one file \cite{Benoit2009}.
			\item Implemented with {\it C++} template definition and implementation separately in different files: a {\it C++} header file and a {\it C++} implementation file. Any file that uses this {\it C++} template will have to import the {\it C++} implementation file, instead of the {\it C++} header file \cite{Moses2016,Ben2013}. {\color{red}\tt I shall use this method. This clearly decouples the {\it C++} template definition and implementation separately into different files, and keeps things modular in my software architecture. It prevents templates from being huge monoliths.}
			\end{enumerate}
		\end{enumerate}
	\end{enumerate}
%%%%%%%%%%%%%%%%%%%%%%%%%%%%%%%%%%%%%%%%%%%%%
\item Debugging: \vspace{-0.3cm}
	\begin{enumerate} \itemsep -2pt
	\item \cite{Eckel2003}, Chp 11 (especially memory management problems, pp. 533)
	\end{enumerate}
%%%%%%%%%%%%%%%%%%%%%%%%%%%%%%%%%%%%%%%%%%%%%
\item STL, books on {\it C++} STL: \vspace{-0.3cm}
	\begin{enumerate} \itemsep -2pt
	\item \cite[\S16.5, 983--996]{Gaddis2012}
	\item \cite{Josuttis2012}
	\item \cite[Chp. 18, pp. 943--998]{Savitch2009}
	\item \cite{Reese2006a}
	\item \cite[Chp. 16, pp. 877--922, 930--940]{Prata2005}
	\item \cite[Chp. 21, 499--545]{Schildt2003a}
	\item \cite{Robson2000}
	\item \cite{Josuttis1999a}
	\end{enumerate}
%%%%%%%%%%%%%%%%%%%%%%%%%%%%%%%%%%%%%%%%%%%%%
\item STL containers: \vspace{-0.3cm}
	\begin{enumerate} \itemsep -2pt
	\item \cite{Gregoire2014}, Chp 15-16
	\item \cite{Lippman2013}, Chp 9,11
	\item \cite{Allain2012}, Chp 18
	\item \cite{Prata2012}, Chp 16
	\item \cite{EliteHussar2010}: \vspace{-0.2cm}
		\begin{enumerate} \itemsep -2pt
		\item {\tt vector$<$int$>$ v(10);} \hspace{0.2in} //{\it\ Create an int vector of size 10.}
		\item {\tt v[5] = 10;} //{\it\ Target of this assignment is the return value of operator[].}
		\item Determine how to operate with a pointer to a vector of pointers: \vspace{-0.1cm}
			\begin{enumerate} \itemsep -1pt
			\item vector$<$object$^{\ast}>$ $^{\ast}$connections;
			\item object$^{\ast}$ oneObject = new object;
			\item connections--$>$push\_back(oneObject);
			\item ($^{\ast}$connections)[0]--$>$Initialize(index);
			\end{enumerate}
		\end{enumerate}
	\item \cite[\S18.2, pp. 960--977]{Savitch2009}
	\item \cite{Reese2006a}, book
	\item \cite[Chp. 8]{Scheinerman2006}
	\item \cite{Schildt2004a}, Chp 8
	\item \cite{Eckel2003}, Chp 4
	\item \cite{Oualline2003}, Chp 25
	\item \cite{Vermeir2001}, Chp 7
	\item \cite[Chp. 24, pp. 625--691; Chp. 25--38, pp. 695--927]{Schildt1998a}
	\end{enumerate}
%%%%%%%%%%%%%%%%%%%%%%%%%%%%%%%%%%%%%%%%%%%%%
\item STL algorithms: \vspace{-0.3cm}
	\begin{enumerate} \itemsep -2pt
	\item \cite{Gregoire2014}, Chp 15,17
	\item \cite{Lippman2013}, Chp 10
	\item \cite{Allain2012}, Chp 18
	\item \cite{Prata2012}, Chp 16
	\item \cite[\S18.3, pp. 977-991]{Savitch2009}
	\item \cite{Reese2006a}, book
	\item \cite{Eckel2003}, Chp 5
	\item \cite{Oualline2003}, Chp 25
	\item \cite{Vermeir2001}, Chp 7
	\end{enumerate}
%%%%%%%%%%%%%%%%%%%%%%%%%%%%%%%%%%%%%%%%%%%%%
\item Function addresses: \vspace{-0.3cm}
	\begin{enumerate} \itemsep -2pt
	\item \cite{Stroustrup2014}, Chp 8
	\item \cite{Stroustrup2009}, Chp 8
	\item \cite[pp. 330--331]{Prata2005}
	\item \cite{Eckel2000}, Chp 3, pp. 213
	\end{enumerate}
%%%%%%%%%%%%%%%%%%%%%%%%%%%%%%%%%%%%%%%%%%%%%
\item Dynamic memory management problems: \vspace{-0.3cm}
	\begin{enumerate} \itemsep -2pt
	\item \cite{Gregoire2014}, Chp 10,22
	\item \cite{Lippman2013}, Chp 12,13
	\item \cite{Allain2012}, Chp 14
	\item \cite[\S13.9, 750--754]{Gaddis2012}.
	\item \cite{Prata2012}, Chp 9,12
%	\item \cite{Gaddis2011}.
	\item \cite{Gaddis2010}, Chp 13.
	\item \cite{Meyers2005}, Chp 2-4
	\item \cite[Chp. 9, pp. 393--423; Chp. 12, pp. 562--606; Chp. 13, pp. 677--685]{Prata2005}
	\item \cite[\S3.1; \S8.1]{Romanik2003}
	\item \cite{Schildt2003}, Chp 29
	\item \cite{Eckel2000}, Chp 6,13
	\item \cite[pp. 349--359]{Schildt1998a}
	\end{enumerate}
%%%%%%%%%%%%%%%%%%%%%%%%%%%%%%%%%%%%%%%%%%%%%
\item Function overloading: \vspace{-0.3cm}
	\begin{enumerate} \itemsep -2pt
	\item \cite{Stroustrup2014}, Chp 8
	\item \cite[\S6.14, pp. 356--360]{Gaddis2012}.
	\item \cite[\S6.14, pp. 359--363]{Gaddis2011}.
	\item \cite{Gaddis2010}, Chp 6.
	\item \cite[\S4.4, pp. 230--243]{Savitch2009}
	\item \cite{Stroustrup2009}, Chp 8
	\item \cite[pp. 216--217, 365--370]{Prata2005}. Function overloading is also known as function polymorphism \cite[pp. 388]{Prata2005}.
	\item \cite{Schildt2003}, Chp 14
	\item \cite{Eckel2000}, Chp 7
	\item \cite[Chp. 14, pp. 361--384]{Schildt1998a}
	\end{enumerate}
%%%%%%%%%%%%%%%%%%%%%%%%%%%%%%%%%%%%%%%%%%%%%
\item Operator overloading: \vspace{-0.3cm}
	\begin{enumerate} \itemsep -2pt
	\item \cite{Lippman2013}, Chp 14
	\item \cite[\S11.2, pp. 633--651]{Savitch2009}
	\item \cite[pp. 502--515, 524--537]{Prata2005}
	\item \cite{Oualline2003}, Chp 18
	\item \cite{Schildt2003}, Chp 15
	\item \cite[Chp. 13, pp. 299--330]{Schildt2003a}
	\item \cite{Eckel2000}, Chp 12
	\item \cite[Chp. 15, pp. 385--418]{Schildt1998a}
	\end{enumerate}
%%%%%%%%%%%%%%%%%%%%%%%%%%%%%%%%%%%%%%%%%%%%%
\item Constants: \vspace{-0.3cm}
	\begin{enumerate} \itemsep -2pt
	\item \cite{Eckel2000}, Chp 8
	\end{enumerate}
%%%%%%%%%%%%%%%%%%%%%%%%%%%%%%%%%%%%%%%%%%%%%
\item Functions and pointers: \vspace{-0.3cm}
	\begin{enumerate} \itemsep -2pt
	%%%%%%%%%%%%%%%%%%%%%%%%%%%%%%%%%%%%%%%%%%%%%
	\item See \cite[\S7.2-7.4]{Pomeranz2014} regarding passing arguments by value, reference, and by address.
	\item \cite{Stroustrup2014}, Chp 8: \vspace{-0.2cm}
		\begin{enumerate} \itemsep -2pt
		\item Pass-by-reference: \vspace{-0.1cm}
			\begin{enumerate} \itemsep -1pt
			\item e.g., void init(vector$<$double$>$ \&v)
			\item ``It is not possible to refer directly to a reference variable after it is defined; any occurrence of its name refers directly to the variable it references. ''
			\item ``Once a reference is created, it cannot be later made to reference another variable. This is something that is often done with pointers.''
			\item ``References cannot be null, whereas pointers can; every reference refers to some variable, although it may or may not be valid.''
			\item ``References cannot be uninitialized. Because it is impossible to reinitialize a reference, they must be initialized as soon as they are created. In particular, local and global variables must be initialized where they are defined, and references which are data members of a class must be initialized in the initializer list of the class's constructor.''
			\item Avoid mixing references and pointers in a block of code to avoid confusion, and make it easier for the {\it C++} code to be read and debug.
			\item The required syntax for pointers make them prominent in comparison to that of references.
			\item The number of operations on references is less than that on pointers. Hence, usage of references is easier to understand than that of pointers. Consequently, it is easier to use references than pointers without enbugging the code.
			\item Pointers can be invalidated as follows: \vspace{-0.1cm}
				\begin{itemize} \itemsep -1pt
				\item ``Carrying a null value''
				\item ``Out-of-bounds [pointer] arithmetic''
				\item Illegal casts on pointers
				\item Produce pointers from random integers
				\end{itemize}
			\item References can be invalidated as follows: \vspace{-0.1cm}
				\begin{itemize} \itemsep -1pt
				\item ``[Refer] to a variable with automatic allocation which goes out of scope''
				\item ``[Refer] to an object inside a block of dynamic memory which has been freed''
				\end{itemize}
			\item ``Arrays are always passed by address. This includes C strings.''
			\item ``Dynamic storage is allocated using pointers.''
			\item Reference: Kurt McMahon, ``Passing Variables by Address,'' in {\it Northern Illinois University: College of Engineering and Engineering Technology: Department of Computer Science: CSCI 241 Intermediate Programming in C++ (Fall 2015): Notes}, Northern Illinois University, DeKalb, IL, October 28, 2015. Available online at: \url{http://faculty.cs.niu.edu/~mcmahon/CS241/Notes/pass_by_address.html}; last accessed on November 3, 2015.
			\end{enumerate}
		\item Pass-by-const-reference: e.g., void print(const vector$<$double$>$ \&v)
		\item Pass-by-value: e.g., void fn(int x)
		\item Pass-by-address: e.g., void print(int * ptr) \vspace{-0.1cm}
			\begin{enumerate} \itemsep -1pt
			\item Reference: Kurt McMahon, ``Passing Variables by Address,'' in {\it Northern Illinois University: College of Engineering and Engineering Technology: Department of Computer Science: CSCI 241 Intermediate Programming in C++ (Fall 2015): Notes}, Northern Illinois University, DeKalb, IL, October 28, 2015. Available online at: \url{http://faculty.cs.niu.edu/~mcmahon/CS241/Notes/pass_by_address.html}; last accessed on November 3, 2015.
			\end{enumerate}
		\end{enumerate}
	%%%%%%%%%%%%%%%%%%%%%%%%%%%%%%%%%%%%%%%%%%%%%
	\item \cite{Lippman2013}, Chp 6
	\item \cite{Allain2012}, Chp 12-13
	\item \cite[Chp. 5, pp. 193--248; Chp. 7, pp. 307--349]{Horstmann2012}
	\item \cite{Prata2012}, Chp 7-8
	\item \cite[Chp. 4--5; \S11.1; and Chp. 14]{Savitch2009}
	\item \cite{Stroustrup2009}, Chp 8
	\item \cite[Chp. 7--8, pp. 279--391]{Prata2005}.: \vspace{-0.2cm}
		\begin{enumerate} \itemsep -2pt
		\item \cite[pp. 382--383]{Prata2005} provides information on which function version (from function overloading, function templates, and function template overloading) is chosen by the compiler, during compilation.
		\item {\tt const} member functions (put the {\tt const} after the function parentheses) guarantees that the invoking object would not be modified.
		\end{enumerate}
	\item \cite{Oualline2003}, Chp 15,20
	\item \cite[Chp. 6--8, pp. 105--179]{Schildt2003a}
	\item \cite{Eckel2000}, Chp 11: \vspace{-0.2cm}
		\begin{enumerate} \itemsep -2pt
		\item use const at the end of accessor functions
		\item Do not use pointers as instance variables
		\end{enumerate}
	\item \cite[Chp. 5--6; pp. 113--160]{Schildt1998a}
	%%%%%%%%%%%%%%%%%%%%%%%%%%%%%%%%%%%%%%%%%%
	\item Elsewhere: \vspace{-0.2cm}
		\begin{enumerate} \itemsep -2pt
		\item \url{http://stackoverflow.com/questions/1143262/what-is-the-difference-between-const-int-const-int-const-and-int-const} \cite{Mortensen2015}: \vspace{-0.1cm}
			\begin{enumerate} \itemsep -1pt
			\item Read it backwards; the first {\it const} can be on either side of the type.
			\item ``Read pointer declarations right-to-left.''
			\item From the answer of Ted Dennison, July 17, 2009. {\bf Rule: The ``const'' goes after the thing it applies to. Putting const at the very front (e.g., const int $^{\ast}$) is an exception to the rule.}
			\item int$^{\ast}$ -- pointer to int
			\item int const $^{\ast}$ == const int $^{\ast}$ -- pointer to const int
			\item int $^{\ast}$ const -- const pointer to int
			\item int const $^{\ast}$ const == const int $^{\ast}$ const -- const pointer to const int
			\item int $^{\ast}$$^{\ast}$ -- pointer to pointer to int
			\item int $^{\ast}$$^{\ast}$ const -- A const pointer to a pointer to an int
			\item int $^{\ast}$ const $^{\ast}$ -- A pointer to a const pointer to an int
			\item int const $^{\ast}$$^{\ast}$ -- A pointer to a pointer to a const int
			\item int $^{\ast}$ const $^{\ast}$ const -- A const pointer to a const pointer to an int
			\end{enumerate}
		\item For the following \cite{Mortensen2015}, let: {\it int var0 = 0;} \vspace{-0.1cm}
			\begin{enumerate} \itemsep -1pt
			\item {\it const int {\rm \&}ptr1 = var0;} // Constant reference
			\item {\it int $^{\ast}$ const ptr2 = {\rm \&}var0;} // Constant pointer
			\item {\it int const $^{\ast}$ ptr3 = {\rm \&}var0;} // Pointer to const
			\item {\it const int $^{\ast}$ const ptr4 = {\rm \&}var0;} // Const pointer to a const
			\end{enumerate}
		\item \cite{Ozcan2013}: \vspace{-0.1cm}
			\begin{enumerate} \itemsep -1pt
			\item ``A reference is a variable that refers to something else and can be used as an alias for that something else. A pointer is a variable that stores a memory address, for the purpose of acting as an alias to what is stored at that address. So, a pointer is a reference, but a reference is not necessarily a pointer. Pointers are a particular implementation of the concept of a reference, and the term tends to be used only for languages that give you direct access to the memory address. References can be implemented internally in a language using pointers, or using some other mechanism.'' Answer from dan1111.
			\item ``Passing an object by value means making a copy of it. You can modify that copy without affecting the original. Making that copy can cost a lot of memory access though. Passing an object by reference means passing a handle to that object. This is cheaper because you don't need to make a copy. It also means that any changes you make will affect the original.'' Answer from Steve Rowe.
			\item ``There is no such thing as a null reference. A reference must always refer to some object. As a result, if you have a variable whose purpose is to refer to another object, but it is possible that there might not be an object to refer to, you should make the variable a pointer, because then you can set it to null. On the other hand, if the variable must always refer to an object, i.e., if your design does not allow for the possibility that the variable is null, you should probably make the variable a reference.'' Answer from Harssh S. Shrivastava.
			\end{enumerate}
		\item A pointer is dereferenced via the explicit $^{\ast}$ operator. The $^{\ast}$ operator should not be used to dereference a reference (variable) \cite{Saks2001}.
		\item \cite{Saks2001}: \vspace{-0.1cm}
			\begin{enumerate} \itemsep -1pt
			\item int $^{\ast}$pi = {\&}i; // Indirect expression to dereference $pi$ to $i$. ``Declare $pi$ as an object of type `pointer to int' whose initial value is the address of object $i$'' \cite{Saks2001a}.
			\item int {\&}ri = i; // $ri$ is dereferenced to refer to $i$. ``Declares $ri$ as an object of type `reference to int' referring to $i$'' \cite{Saks2001a}.
			\item The {\it C++} standard does not dictate how compilers shall implement references. However, popular compilers tend to implement references as pointers. Therefore, there are no significant advantages of using references or pointers.
			\end{enumerate}
		\item \cite{Saks2001a}: \vspace{-0.1cm}
			\begin{enumerate} \itemsep -1pt
			\item ``A valid reference must refer to an object; a pointer need not. A pointer, even a const pointer, can have a null value. A null pointer doesn't point to anything.''
			\item I can bind a reference to a null pointer, but I cannot dereference a null pointer since it can ``produce undefined behavior''.
			\end{enumerate}
		\item You cannot call a non-const method from a const method. That would 'discard' the const qualifier.: \vspace{-0.1cm}
			\begin{enumerate} \itemsep -1pt
			\item \url{http://stackoverflow.com/questions/2382834/discards-qualifiers-error}
			\end{enumerate}
		\item Pointer to constant data: {\it const type$^{\ast}$ variable;} and {\it type const $^{\ast}$ variable;} \vspace{-0.1cm}
			\begin{enumerate} \itemsep -1pt
			\item \url{http://www.cprogramming.com/reference/pointers/const_pointers.html}
			\end{enumerate}
		\item Pointer with constant memory address: {\it type $^{\ast}$ const variable = some-memory-address;} \vspace{-0.1cm}
			\begin{enumerate} \itemsep -1pt
			\item \url{http://www.cprogramming.com/reference/pointers/const_pointers.html}
			\end{enumerate}
		\item Constant data with a constant pointer: {\it const type $^{\ast}$ const variable = some-memory-address;} and {\it type const $^{\ast}$ const variable = some-memory-address;} \vspace{-0.1cm}
			\begin{enumerate} \itemsep -1pt
			\item \url{http://www.cprogramming.com/reference/pointers/const_pointers.html}
			\end{enumerate}
		\end{enumerate}
	\item With shallow copying, I would only copy the memory references or pointers. The copy and the original reference the same object. On the other hand, with deep copying, I would copy the values; this is also known as cloning. The copy and the original reference do not share objects; each of them references its own object. The default copy constructor carries out shallow copy.
	\end{enumerate}
%%%%%%%%%%%%%%%%%%%%%%%%%%%%%%%%%%%%%%%%%%%%%
\item Extern function: \vspace{-0.3cm}
	\begin{enumerate} \itemsep -2pt
	\item : \vspace{-0.2cm}
		\begin{enumerate} \itemsep -2pt
		\item 
		\end{enumerate}
	\item : \vspace{-0.2cm}
		\begin{enumerate} \itemsep -2pt
		\item 
		\end{enumerate}
	\item : \vspace{-0.2cm}
		\begin{enumerate} \itemsep -2pt
		\item 
		\end{enumerate}
	\item : \vspace{-0.2cm}
		\begin{enumerate} \itemsep -2pt
		\item 
		\end{enumerate}
	\item : \vspace{-0.2cm}
		\begin{enumerate} \itemsep -2pt
		\item 
		\end{enumerate}
	\item : \vspace{-0.2cm}
		\begin{enumerate} \itemsep -2pt
		\item 
		\end{enumerate}
	\item : \vspace{-0.2cm}
		\begin{enumerate} \itemsep -2pt
		\item 
		\end{enumerate}
	\end{enumerate}
%%%%%%%%%%%%%%%%%%%%%%%%%%%%%%%%%%%%%%%%%%%%%
\item {\tt typename}: \vspace{-0.3cm}
	\begin{enumerate} \itemsep -2pt
	\item Evan Driscoll, ``A Description of the C++ {\tt typename} keyword,'' from the Department of Computer Sciences, University of Wisconsin-Madison College of Engineering, University of Wisconsin-Madison, Madison, WI. Available online at: \url{http://pages.cs.wisc.edu/~driscoll/typename.html}; last accessed on February 15, 2016.
	\item From \cite[API documentation for Eigen3: The template and typename keywords in C++]{Avery2016}, or \url{http://eigen.tuxfamily.org/dox/TopicTemplateKeyword.html}.
	\item Wikipedia contributors, ``typename,'' in {\it Wikipedia, The Free Encyclopedia: C++}, Wikimedia Foundation, San Francisco, CA, April 13, 2015. Available online at: \url{https://en.wikipedia.org/wiki/Typename}; last accessed on February 15, 2016.: \vspace{-0.2cm}
		\begin{enumerate} \itemsep -2pt
		\item Usage \#1: ``A synonym for `class' in template parameters''
		\item Usage \#2: ``A method for indicating that a dependent name is a type''
		\end{enumerate}
	\item \cite[pp. 916]{Savitch2009}
	\item 
	\item 
	\item 
	\item 
	\item 
	\item 
	\item 
	\end{enumerate}
%%%%%%%%%%%%%%%%%%%%%%%%%%%%%%%%%%%%%%%%%%%%%
\item OOD and inheritance: \vspace{-0.3cm}
	\begin{enumerate} \itemsep -2pt
	\item \cite{Gregoire2014}, Chp 4-9
	\item \cite{Stroustrup2014}, Chp 9
	\item \cite{Lippman2013}, Chp 7,15,18,19
	\item \cite{Allain2012}, Chp 24-26
	\item \cite[Chp. 13--15; Appendices E and J]{Gaddis2012}.
	\item \cite[Chp. 9--10, 389--479]{Horstmann2012}
	\item \cite{Prata2012}, Chp 10-11,13,14,15
	\item \cite[Chp. 7, 11, 15; Appendices A, D, J, and K]{Gaddis2011}.
	\item \cite{Gaddis2010}, Chp 13,14,15.
	\item \cite[Chp. 10; \S12.1, 696--711; Chp. 15; Chp. 17]{Savitch2009}
	\item \cite{Stroustrup2009}, Chp 9
	\item \cite[Chp. 10--14, pp. 445--786]{Prata2005}
	\item \cite{Oualline2003}, Chp 13-14,21
	\item \cite[pp. 6--8; Chp. 11--12, pp. 245--298; Chp. 14--15, pp. 331--373]{Schildt2003a}
	\item \cite{Vermeir2001}, Chp 3-4,8
	\item \cite{Eckel2000}, Chp 14,15
	\item \cite[Chp. 11--12, pp. 255--325; Chp. 16--17, pp. 419--460]{Schildt1998a}
	\end{enumerate}
%%%%%%%%%%%%%%%%%%%%%%%%%%%%%%%%%%%%%%%%%%%%%
\item SW engineering issues: \vspace{-0.3cm}
	\begin{enumerate} \itemsep -2pt
	\item \cite{Gregoire2014}, Chp 24-26
	\item \cite{Allain2012}, Chp 21
	\item \cite[Chp. 4 and 7]{Romanik2003}
	%%%%%%%%%%%%%%%%%%%%%%%%%%%%%%%%%%%%%%%%%%
	\item Debugging: \vspace{-0.2cm}
		\begin{enumerate} \itemsep -2pt
		\item \cite[\S14.2]{Scheinerman2006}
		\end{enumerate}
	\end{enumerate}
%%%%%%%%%%%%%%%%%%%%%%%%%%%%%%%%%%%%%%%%%%%%%
\item multi-threading: \vspace{-0.3cm}
	\begin{enumerate} \itemsep -2pt
	\item \cite{Schildt2004a}, Chp 3
	\item \cite[\S5.1]{Romanik2003}
	\end{enumerate}
%%%%%%%%%%%%%%%%%%%%%%%%%%%%%%%%%%%%%%%%%%%%%
\item graphs: \vspace{-0.3cm}
	\begin{enumerate} \itemsep -2pt
	\item \cite{Schildt2004a}, Chp 7
	\end{enumerate}
%%%%%%%%%%%%%%%%%%%%%%%%%%%%%%%%%%%%%%%%%%%%%
\item typedef: \vspace{-0.3cm}
	\begin{enumerate} \itemsep -2pt
	\item In the sandbox, use the {\it Make} target {\it make typedef} to study an example of how {\it typedef} can be used. When the {\it header file} defines/specifies the {\it typedef}, and is included in the {\it C++ implementation file} and other {\it C++ implementation file}s that instantiates those objects, it can be used subsequently without additional definition/specification. October 6, 2015.
	\item \cite[pp. 510-512]{Savitch2009}
	\end{enumerate}
\end{enumerate}

	I am skipping topics, such as: run-time type ID and casting operators \cite[Chp. 19, pp. 451--470]{Schildt2003a}; namespaces \cite[pp. 472--480]{Schildt2003a}; \#error {\it C++} preprocessor \cite[pp. 552]{Schildt2003a}; \#line {\it C++} preprocessor \cite[pp. 558]{Schildt2003a}; \#pragma {\it C++} preprocessor \cite[pp. 559]{Schildt2003a}; \#\# {\it C++} preprocessor operator \cite[pp. 559--560]{Schildt2003a}; and {\it C++} predefined macro names \cite[pp. 560--561]{Schildt2003a}.





%%%%%%%%%%%%%%%%%%%%%%%%%%%%%%%%%%%%%%%%%%%
\section{Computational Complexity of C++ Containers}
\label{sec:ComputationalComplexityofCppContainers}


	Table \ref{tab:ComputationalComplexityofCppContainers} shows a tabulated summary of containers in the {\it C++} Standard Template Library (STL) and the computational complexity for each of their common operations: add(element e), remove(element e), search(element e), size(), empty(), begin(), and end(). \\

%\begin{table}[htdp]
\begin{table}[htp]
\caption{Computational Complexity of Basic Operations of Containers from the {\it C++ STL}.}	\vspace{-0.2in}
\label{tab:ComputationalComplexityofCppContainers}
	\begin{center}
		\begin{tabular}{|c|c|c|c|c|c|c|c|}
		\hline
		Container $\backslash$ Complexity & add & remove & search & size & empty & begin & end \\
		\hline
		vector & O(1) & O(n) & O(n) & O(1) & O(1) & O(1) & O(1) \\
		\hline
		list & O(1) & O(n) & O(n) & O(1) & O(1) & O(1) & O(1) \\
		\hline
		queue & O(1) amortized & O(1) & O(n) & O(1) & O(1) & O(1) & O(1) \\
		\hline
		priority queue & O(log n) & O(log n) & O(log n)???, or O(n) & O(1) & O(1) & O(1) & ??? \\
		\hline
		set & O(log n) & O(log n) & O(log n) & O(1) & O(1) & O(1) & O(1) \\
		\hline
		multi-set & O(log n) & ??? & O(log n) & O(1) & O(1) & O(1) & O(1) \\
		\hline
		map & O(log n) & O(log n) & O(log n) & O(1) & O(1) & O(1) & O(1) \\
		\hline
		multi-map & O(log n) & ??? & O(log n) & O(1) & O(1) & O(1) & O(1) \\
		\hline
		stack & O(1) & O(1) & O(n) & O(1) & O(1) & O(1) & O(1) \\
		\hline
		\end{tabular}
	\end{center}
\end{table}


To conclude, we can get some facts about each data structure: \vspace{-0.3cm}
\begin{enumerate} \itemsep -4pt
\item {\tt std::list} is very very slow to iterate through the collection due to its very poor spatial locality.
\item {\tt std::vector} and {\tt std::deque} perform always faster than {\tt std::list} with very small data
\item {\tt std::list} handles very well large elements
\item {\tt std::deque} performs better than a {\tt std::vector} for inserting at random positions (especially at the front, which is constant time)
\item {\tt std::deque} and {\tt std::vector} do not support very well data types with high cost of copy/assignment
\end{enumerate}



This draws simple conclusions on the usage of each data structure \cite{Bulka2000,Josuttis1999a}: \vspace{-0.3cm}
\begin{enumerate} \itemsep -4pt
\item Number crunching: use std::vector or std::deque
\item Linear search: use std::vector or std::deque
\item Random Insert/Remove:
\item Small data size: use std::vector
\item Large element size: use std::list (unless if intended principally for searching)
\item Non-trivial data type: use std::list unless you need the container especially for searching. But for multiple modifications of the container, it will be very slow.
\item Push to front: use std::deque or std::list
\end{enumerate}

Notes about asymptotic notations: \vspace{-0.3cm}
\begin{enumerate} \itemsep -4pt
\item Comparison of big O notations, and other asymptotic notations, in general -- based on ``running time (T(n))'' $[$Wikipedia 2015a$] [$Wikipedia 2015$]$: \vspace{-0.3cm}
	\begin{enumerate} \itemsep -2pt
	\item $O(1)$: constant time
	\item $O(\log^{\ast} n)$, log star: iterated logarithmic time.

	Log star n is a recursive function; $\log^{\ast} n \coloneqq
		\begin{cases}
		0 &: \mathrm{if}\ n \leq 1 \\
		1 + \log^{\ast}(\log n) &: \mathrm{if}\ n > 1 \\
		\end{cases}$\ $[$Wikipedia 2015b$]$
%	\usepackage{mathtools}
%	Use the above/aforementioned package to use the symbol: \coloneqq \cite{Pakin2008}
	\item $O(\log \log n)$: log-logarithmic, double logarithmic
	\item $O(\log n)$: logarithmic time, computational time complexity class DLOGTIME. E.g., $\log n^{2}$.
	\item poly$(\log n)$ or $O((\log n)^{c}), c > 1$: polylogarithmic time. E.g., $(\log n)^{2}$.
	\item $O(n^{c})$, where $0 < c < 1$: fractional power. E.g., $n^{\frac{2}{3}}$.
	\item $o(n)$: sub-linear time (or sublinear time)
	\item $O(n)$: linear time
	\item $O(n \log^{\ast} n)$: ``n log star n'' time, or ``n log-star n''
	\item $O(n \log n) = O(\log n!)$: linearithmic time, including $\log n!$. Or, loglinear, or quasilinear.
	\item $O(n^{2})$: quadratic time
	\item $O(n^{3})$: cubic time
	\item poly$(n)$, or $2^{O(\log n)}$. Or, $O(n^{c}), c > 1$: polynomial time, including $n, n \log n, n^{10}$. Computational time complexity class P. Or, algebraic.
	\item $2^{{\rm poly}(\log n)}$: quasi-polynomial time, including $n^{\log \log n}, n^{\log n}$. Computational time complexity class QP.
	\item $O(2^{n^{\varepsilon}}), \forall \varepsilon > 0$: sub-exponential time, including $O(2^{\log n^{\log \log n}})$. Computational time complexity class SUBEXP.
	\item $2^{o(n)}$: sub-exponential time, including $2^{n^{\frac{1}{3}}}$. Computational time complexity class SUBEXP. Or, L-notation.
	\item $2^{O(n)}$: exponential time (with linear exponent), including $1.1^{n}, 10^{n}$. Computational time complexity class E.
	\item $2^{{\rm poly}(n)}$. Or, $O(c^{n}), c > 1$: exponential time, including $2^{n}, 2^{n^{2}}$. Computational time complexity class EXPTIME.
	\item $O(n!)$: factorial time, including $n!$.
	\item $2^{2^{{\rm poly}(n)}}$: double exponential time, including $2^{2^{n}}$. Computational time complexity class 2-EXPTIME.
	\item $n! > n^{n}$
	\end{enumerate}
\item Types of asymptotic notations $[$Wikipedia 2015$]$: \vspace{-0.3cm}
	\begin{enumerate} \itemsep -2pt
	\item $f(n) = O(g(n))$: Big O notation, or Big Oh notation
	\item $f(n) = \Omega(g(n))$: Big Omega notation
	\item $f(n) = \Theta(g(n))$: Big Theta notation
	\item $f(n) = o(g(n))$: Small O notation, or Small Oh notation
	\item $f(n) = \omega(g(n))$: Small Omega notation
	\item $f(n) \sim g(n)$: ``On the order of''
	\end{enumerate}
\item References: \vspace{-0.3cm}
	\begin{enumerate} \itemsep -2pt
	\item $[$Wikipedia 2015$]$ Wikipedia contributors, ``Big O notation,'' sections {\it Orders of common functions} and {\it Related asymptotic notations: Family of Bachmann?Landau notations}, in {\it Wikipedia, The Free Encyclopedia: Analysis of algorithms, or Asymptotic analysis}, Wikimedia Foundation, San Francisco, CA, November 29, 2015. Available online at: \url{https://en.wikipedia.org/wiki/Big_O_notation#Orders_of_common_functions}; last accessed on December 1, 2015.
	\item $[$Wikipedia 2015a$]$ Wikipedia contributors, ``Time complexity,'' section {\it Table of common time complexities}, in {\it Wikipedia, The Free Encyclopedia: Computational complexity theory}, Wikimedia Foundation, San Francisco, CA, November 16, 2015. Available online at: \url{https://en.wikipedia.org/wiki/Time_complexity#Table_of_common_time_complexities}; last accessed on December 1, 2015.
	\item $[$Wikipedia 2015b$]$ Wikipedia contributors, ``Iterated logarithm,'' in {\it Wikipedia, The Free Encyclopedia: Asymptotic analysis}, Wikimedia Foundation, San Francisco, CA, November 6, 2015. Available online at: \url{https://en.wikipedia.org/wiki/Iterated_logarithm}; last accessed on December 1, 2015.
	\end{enumerate}
\item Note that I denote ``is defined as'' as: $\equiv, \triangleq, \stackrel{\text{\tiny def}}{=}, \coloneqq$
\item Note that $\log n$ is faster than $(\log n)^{2}$, although initially the latter is slightly faster than the former (for negligibly small $n$).
\end{enumerate}


Books on computational complexity: \vspace{-0.3cm}
\begin{enumerate} \itemsep -4pt
\item \cite[pp.10--11 ]{Josuttis2012}
\end{enumerate}


%%%%%%%%%%%%%%%%%%%%%%%%%%%%%%%%%%%%%%%%%%%
\section{Additional Notes About C++}
\label{sec:AdditionalNotesAboutCpp}


Static variables: \vspace{-0.3cm}
\begin{enumerate} \itemsep -4pt
%%%%%%%%%%%%%%%%%%%%%%%%%%%%%%%%%%%%
%	\item K. Hong, ``C++ Tutorial
%	Private Inheritance - 2015,'' San Francisco, CA. Available online from {\it Open Source \dots: Java/C++/Python/Android/Design Patterns: C++ Tutorial Home -- 2015} at: \url{}; last accessed on October 23, 2015.
\item K. Hong, ``Static Variables and Static Class Members - 2015,'' San Francisco, CA. Available online from {\it Open Source \dots: Java/C++/Python/Android/Design Patterns: C++ Tutorial Home -- 2015} at: \url{http://www.bogotobogo.com/cplusplus/statics.php}; last accessed on October 23, 2015.
\end{enumerate}




Formatting data: \vspace{-0.3cm}
\begin{enumerate} \itemsep -4pt
\item Synesis Software Pty Ltd staff, ``Synesis Software Training Courses: FastFormat, Beginner's (part 1 of 2),'' Synesis Software Pty Ltd, Sydney, Australia, 2015.  Available online at: \url{http://www.synesis.com.au/training-beginners-fastformat.html}; December 1, 2015 was the last accessed date. \vspace{-0.3cm}
	\begin{enumerate} \itemsep -2pt
	\item ``Formatting APIs'': \vspace{-0.2cm}
		\begin{enumerate} \itemsep -2pt
		\item ``Replacement-based APIs'': \vspace{-0.1cm}
			\begin{enumerate} \itemsep -1pt
			\item ``Streams (printf()-family)''
			\item ``Boost.Format''
			\item ``FastFormat.Format''
			\end{enumerate}
		\item ``Concatenation-based APIs'': \vspace{-0.1cm}
			\begin{enumerate} \itemsep -1pt
			\item ``IOStreams''
			\item ``Loki.SafeFormat''
			\item ``FastFormat.Write''
			\end{enumerate}
		\end{enumerate}
	\item ``struct tm''
	\item ``struct in\_addr''	% struct in_addr
	\item ``ATL types''
	\item ``ACE types''
	\end{enumerate}
\item Synesis Software Pty Ltd staff, ``Synesis Software Training Courses: FastFormat, Advanced (part 2 of 2),'' Synesis Software Pty Ltd, Sydney, Australia, 2015.  Available online at: \url{http://www.synesis.com.au/training-advanced-fastformat.html}; December 1, 2015 was the last accessed date. \vspace{-0.3cm}
	\begin{enumerate} \itemsep -2pt
	\item ``Format-specification Defect Handling'': ``Scoping'' and ``Disgnostic Logging''
	\end{enumerate}
\end{enumerate}



%%%%%%%%%%%%%%%%%%%%%%%%%%%%%%%%%%%%%%%%%%%
\subsection{Alternate Computer Number System for Representing Fractions in C++}
\label{ssec:AlternateComputerNumberSystemforRepresentingFractionsInCpp}

	An alternate computer number system for representing fractions in C++ is the fixed-point number system. For a detailed classification of computer number systems, see \cite{Lu2004}. \\

	In {\it C++}, the numerical data types are based on cardinal numbers (e.g., one, two, three, \dots), instead of ordinal numbers/integers (e.g., first, second, third, \dots); see \cite{DictionaryDotComStaff2016,WolframResearchStaff2016,Pierce2016} for the definitions of ``cardinal number'' and ``ordinal number.'' From \cite{Pierce2016}, ``a Nominal Number is a number used only as a name, or to identify something (not as an actual value or position).'' E.g., ``the number on the back of a footballer (``8''),'' ``a postal code (``91210''),'' and ``a model number (``380'').'' \\




Resources to help me implement the fixed-point ``data type'' as a class in {\it C++}, and fixed-point arithmetic: \vspace{-0.3cm}
\begin{enumerate} \itemsep -4pt
\item 
\end{enumerate}



%%%%%%%%%%%%%%%%%%%%%%%%%%%%%%%%%%%%%%%%%%%
\section{Software Development in C++}
\label{sec:SoftwareDevelopmentInCpp}


Notes about software development in {\it C++}: \vspace{-0.3cm}
\begin{enumerate} \itemsep -4pt
\item Notes from Synesis Software Pty Ltd: \vspace{-0.3cm}
	\begin{enumerate} \itemsep -2pt
	\item Synesis Software Pty Ltd staff, ``Synesis Software Training Courses,'' Synesis Software Pty Ltd, Sydney, Australia, 2015.  Available online at: \url{http://www.synesis.com.au/training.html}; December 1, 2015 was the last accessed date. \vspace{-0.2cm}
		\begin{enumerate} \itemsep -2pt
		\item Use {\tt FastFormat} as a ``C++ diagnostic logging API library''
		\item {\tt STLSoft libraries}. ``Apply the concepts, principles and techniques of Extended STL to enhance the expressiveness, flexibility, and performance of your C++ software.'' See \cite{Wilson2007} for more details.
		\item ``Building Bullet-Proof Software in C++ - no system built by Synesis Software has ever failed in production. This course takes you through the principles and practices of how we develop software, providing you with practical, applicable strategies and tactics for achieving the same outcome in your software developments.''
		\item ``Guerilla Testing C++ - or, {\bf \large `How to discover the Gold Nuggets in your Big Ball of Mud'}. No matter how badly a C++ codebase is enmeshed, you can get it under test if you know how to master its coupling.''
		\end{enumerate}
	\item Synesis Software Pty Ltd staff, ``Resources,'' Synesis Software Pty Ltd, Sydney, Australia.  Available online at: \url{http://www.synesis.com.au/resources.html}; December 1, 2015 was the last accessed date. \vspace{-0.2cm}
		\begin{enumerate} \itemsep -2pt
		\item 100\% type-safe {\it C++} API
		\item C++ diagnostic logging API library (or, diagnostic logging libraries): \vspace{-0.1cm}
			\begin{enumerate} \itemsep -1pt
			\item Pantheios: \url{http://pantheios.org/}
			\item ACE
			\item log4cxx
			\end{enumerate}
		\item C++ formatting library: FastFormat \url{http://fastformat.org/}
		\item ``The STLSoft libraries provide STL extensions and facades over operating-system and third-party-library APIs. The libraries are 100\% header-only.'' See \url{http://stlsoft.org/}.
		\item ``UNIXem is a simple library that emulates a useful subset of the UNIX system APIs on Windows\dots\ UNIXem is the only library provided by Synesis Software that is not production-quality. It is appropriate for research, such as when developing tests for cross-platform software.'' See \url{http://synesis.com.au/software/unixem.html}.
		\end{enumerate}
	\item Synesis Software Pty Ltd staff, ``Guerilla Testing C++ or, `How to discover the Gold Nuggets in your Big Ball of Mud','' Synesis Software Pty Ltd, Sydney, Australia.  Available online at: \url{http://www.synesis.com.au/training-guerilla-testing-cplusplus.html}; December 1, 2015 was the last accessed date.: \vspace{-0.2cm}
		\begin{enumerate} \itemsep -2pt
		\item ``Change is the most expensive part of the cost of a software project. The biggest impediments to change are lack of clarity on what to alter to effect the change, and uncertainty about unintended side-effects of the change.''
		\item ``No matter how badly a C++ codebase is enmeshed, you can get it under test if you know how to master its coupling.''
		\item ``Many long-lived codebases have evolved to a point where some, perhaps most, aspects of its functionality are no longer precisely known / codified / automatically tested. This course will teach you, using practical examples, how to wrest control from any codebase, no matter how badly enmeshed, isolate known pieces of good functionality, get them under test, and eventually to isolate and separate them into a new context, while, where required, maintaining compatibility with their original context.''
		\item ``This course will teach you how to refactor any codebase with confidence, rather than poking at the edges of its functionality in fear.''
		\item ``Release costs'' serve as an indicator to the existence of ``a Big Ball of Mud.''
		\item ``Factors that inhibit testing'': \vspace{-0.1cm}
			\begin{enumerate} \itemsep -1pt
			\item ``Coupling, coupling, coupling''
			\item ``The inconstant environment''
			\item ``Trust''
			\item ``Defensive code''
			\item ``Fuzzy (or no!) abstraction borders''
			\end{enumerate}
		\item ``Key characteristics'' identified in situ: ``diagnostics, contracts, code coverage, and testing.''
		\item Remember the following ``when testing mud-balls'': ``automation; minimalism, incrementality, unit testing vs component testing; coverage (in realistic time); only change what you can test (and are testing!) -- [there are] exceptions to this rule; beyond salvation -- sometimes it's just mud.''
		\item Islands of ``known Functionality'' are created as follows: \vspace{-0.1cm}
			\begin{enumerate} \itemsep -1pt
			\item ``Decomposition -- Identifying Units, Identifying Components, and Identifying Modules''
			\item ``Triage''
			\item ``Isolation''
			\item ``Striding two worlds''
			\item ``Transplantation''
			\item ``Separation''
			\item ``Versioning -- Static and Dynamic''
			\item ``When to `throw it out'.''
			\end{enumerate}
		\item ``Inconstant Environment'' handling: \vspace{-0.1cm}
			\begin{enumerate} \itemsep -1pt
			\item ``File system''
			\item ``Memory''
			\item ``User-interface''
			\item ``Time''
			\item ``Data storage''
			\end{enumerate}
		\item Techniques to address/mitigate coupling: \vspace{-0.1cm}
			\begin{enumerate} \itemsep -1pt
			\item ``Pre-processor'': \vspace{-0.1cm}
				\begin{itemize} \itemsep -1pt
				\item \#ifdef
				\item \#define
				\item \#include
				\item ``I would recommend putting your include guards above your includes. That way the includes don't get parsed twice for the same file.'' {\color{red} That is, the pre-processors {\tt \#ifdef} and {\tt \#define} should be placed above the {\tt \#include} statements. This would avoid repetitive parsing of (header) files that are included.} Reference: sdsmith, comment to the question ``Why is the discrepancy in these two cases of using C++ templates?,'' Stack Exchange Inc., New York, NY, March 30, 2016. Available online from {\it Stack Exchange Inc.: Stack Overflow: Questions} at: \url{http://stackoverflow.com/q/36319028/1531728}; March 30, 2016 was the last accessed date \cite{sdsmith2016}.
				\end{itemize}
			\item ``linkage'': \vspace{-0.1cm}
				\begin{itemize} \itemsep -1pt
				\item ``interpositioning''
				\item ``dynamic library redirection''
				\end{itemize}
			\item ``object-oriented techniques'': \vspace{-0.1cm}
				\begin{itemize} \itemsep -1pt
				\item ``overloading''
				\item ``overriding''
				\item ``inheritance''
				\item ``interfaces''
				\end{itemize}
			\item ``patterns'': \vspace{-0.1cm}
				\begin{itemize} \itemsep -1pt
				\item ``class adaptor''
				\item ``instance adaptor''
				\item ``decorator''
				\item ``visitor''
				\end{itemize}
			\item ``generic programming'': \vspace{-0.1cm}
				\begin{itemize} \itemsep -1pt
				\item ``policies''
				\item ``shims''
				\item ``traits''
				\end{itemize}
			\item ``Testing'': \vspace{-0.1cm}
				\begin{itemize} \itemsep -1pt
				\item ``Stubbing''
				\item ``Mocking''
				\item ``Versioned testing''
				\end{itemize}
			\end{enumerate}
		\item 
		\item 
		\item 
		\item 
		\item 
		\item 
		\item 
		\item 
		\end{enumerate}
	\end{enumerate}
\item To obtain the {\tt x86} assembly output to a {\it C++} program: \vspace{-0.3cm}
	\begin{enumerate} \itemsep -2pt
	\item Reference: Andrew Edgecombe, answer to ``How do you get assembler output from C/C++ source in gcc?,'' Stack Exchange Inc., New York, NY, September 26, 2008 (edited by Prashant Kumar on May 1, 2012). Available online from {\it Stack Exchange Inc.: Stack Overflow: Questions} at: \url{http://stackoverflow.com/a/137074}; March 15, 2016 was the last accessed date.: \vspace{-0.2cm}
		\begin{enumerate} \itemsep -2pt
		\item To view the source file: {\tt cat {\it [source\_file]}}
		\item To remove all symbol table and relocation information from the executable, the executable for that source file becomes: {\tt gcc -s {\it [source\_file.c]}}: \vspace{-0.1cm}
			\begin{enumerate} \itemsep -1pt
			\item Or, try: {\tt gcc -S {\it [source\_file.c]}}
			\item Alternatively, try: {\tt strip {\it [source\_file]}}
			\item {\tt gcc -g {\it [source\_file.c]}} adds debugging information to the executable.
			\end{enumerate}
		\item If access to source files is unavailable, but access to the object file is available, use: {\tt objdump -S --disassemble {\it [object\_file]} $>$ {\it [object\_file.dump]}}. Use {\tt file {\it [object\_file]}} to determine if I can get enough information from the object file.: \vspace{-0.1cm}
			\begin{enumerate} \itemsep -1pt
			\item {\tt gcc -S -o {\it [unassembled\_file.s]} {\it [source\_file.c]}}
			\item To view the output file from {\tt gcc -S}: {\tt cat {\it [source\_file]}}
			\end{enumerate}
		\end{enumerate}
	\item References: \cite[pp. 3, or pp. 15 in the PDF]{Arndt2010} and Kenneth Finnegan, answer to ``How do you get assembler output from C/C++ source in gcc?,'' Stack Exchange Inc., New York, NY, September 26, 2008. Available online from {\it Stack Exchange Inc.: Stack Overflow: Questions} at: \url{http://stackoverflow.com/a/137479}; March 16, 2016 was the last accessed date. Also, includes information from comments to this answer by Sundaram Ramaswamy (May 6, 2013) and Luu V{\~{\i}}nh Ph{\'{u}}c (June 7, 2014): \vspace{-0.2cm}
		\begin{enumerate} \itemsep -2pt
		\item Create assembly code: {\tt c++ -S -fverbose-asm -g -02 {\it [source\_code.cpp]} -o {\it [unassembled\_file.s]}}
		\item Create assembled code interfaced with source lines: {\tt as -alhnd {\it [unassembled\_file.s]} $>$ {\it [listing\_file.lst]}} 
		\item Alternatively, the on line version on {\it Mac OS X}, try: \vspace{-0.1cm}
			\begin{enumerate} \itemsep -1pt
			\item {\tt g++ -g -O0 -c -fverbose-asm -Wa,-adhln test.cpp $>$ test.lst}
			\item {\tt gcc -c -g -Wa,-ahl=test.s test.c}
			\item {\tt gcc -c -g -Wa,-a,-ad [{\it other GCC options}] test.c $>$ test.txt}
			\end{enumerate}
		\end{enumerate}
	\item Reference: Cr McDonough, answer to ``How do you get assembler output from C/C++ source in gcc?,'' Stack Exchange Inc., New York, NY, September 29, 2013. Available online from {\it Stack Exchange Inc.: Stack Overflow: Questions} at: \url{http://stackoverflow.com/a/19083877}; March 16, 2016 was the last accessed date. \vspace{-0.2cm}
		\begin{enumerate} \itemsep -2pt
		\item {\tt g++ -g -O -Wa,-aslh horton\_ex2\_05.cpp $>$ list.txt}
		\end{enumerate}
	\item Reference: Andrew Pennebaker, answer to ``How do you get assembler output from C/C++ source in gcc?,'' Stack Exchange Inc., New York, NY, February 6, 2013. Available online from {\it Stack Exchange Inc.: Stack Overflow: Questions} at: \url{http://stackoverflow.com/a/7871911}; March 16, 2016 was the last accessed date. Also, comment from Grady Player (February 6, 2013) is helpful. \vspace{-0.2cm}
		\begin{enumerate} \itemsep -2pt
		\item To get the LLVM assembly code, try: {\tt llvm-gcc -emit-llvm -S hello.c}
		\item I can also use the same command for {\tt clang}.
		\end{enumerate}
	\end{enumerate}
\end{enumerate}























%%%%%%%%%%%%%%%%%%%%%%%%%%%%%%%%%%%%%%%%%%%
\subsection{Using ``Design By Contract''}
\label{ssec:UsingDesignByContract}

	The ``Design By Contract'' approach shall be used in software development. This approach is also known as: ``contract programming, programming by contract, and design-by-contract programming.'' Adhere strongly to Hoare logic. \\

References: \\
Wikipedia contributors, ``Design by contract,'' in {\it Wikipedia, The Free Encyclopedia: Software design}, Wikimedia Foundation, San Francisco, CA, January 20, 2016. Available online at: \url{https://en.wikipedia.org/wiki/Design_by_contract}; last accessed on February 9, 2016. \\

Wikipedia contributors, ``Hoare logic,'' in {\it Wikipedia, The Free Encyclopedia: Static program analysis}, Wikimedia Foundation, San Francisco, CA, November 8, 2015. Available online at: \url{https://en.wikipedia.org/wiki/Hoare_logic}; last accessed on February 10, 2016. \\

	That is, at the start of an implementation of a ({\it C++}) function, check that its precondition(s) is (/are) met; preconditions shall be chosen to be as weak as possible. Within the implementation of the function, check if the assertions (properties that must be true during execution of the function) hold. Lastly, at the end of the implementation, check that its postcondition(s) is (/are) met; postconditions shall be chosen to be as strong as possible. \\

Reference: \\
Wikipedia contributors, ``Predicate transformer semantics,'' in {\it Wikipedia, The Free Encyclopedia: Formal methods}, Wikimedia Foundation, San Francisco, CA, November 25, 2015. Available online at: \url{https://en.wikipedia.org/wiki/Predicate_transformer_semantics}; last accessed on February 10, 2016.



%%%%%%%%%%%%%%%%%%%%%%%%%%%%%%%%%%%%%%%%%%%
\subsubsection{Hoare Logic for Computer Arithmetic}
\label{sssec:HoareLogicForComputerArithmetic}

	Check if arithmetic and logical operations cause overflows or underflows \cite{Crowl2015,Crowl2012}. When faced with a constrained range for representing real numbers with bits in computer hardware, a constrained resolution such that a representation smaller than single-precision floating-point numbers is required, or low-cost and/or low-power electronic/computer systems that do not have floating-point arithmetic circuits, use a circuit-based implementation of fixed-point arithmetic.



%%%%%%%%%%%%%%%%%%%%%%%%%%%%%%%%%%%%%%%%%%%
\subsection{Debugging {\it C++} Software}
\label{ssec:DebuggingCppSoftware}

	This section covers debugging syntax errors (in \S\ref{sssec:InterpretingCppCompilationErrors}) that are reported by {\it C++} compilers. It also covers semantic errors that can be discovered via software testing and formal verification tools; see \cite[\S4.4, pp. 119]{Pradhan2009}. In addition, it includes performance debugging \cite[\S4.5, pp. 119--120]{Pradhan2009}, using software profilers \cite[Figure 7.1, pp. 292; and \S7.2.10, pp. 302]{Fisher2005} \cite[pp. 148]{Kernighan1999} \cite[pp. 35-9 -- 35-10]{Lee2008b} \cite[\S3.2, pp. 16--18]{Fog2014c} \cite[\S3.3.2, pp. 21--22; pp. 23; Figure 40, pp. 102; Appendix C, \S C.1.7, pp. 104]{Baik2013} and static analysis software \cite[\S5.4, 65-66; and Figure 40, pp. 102]{Baik2013} \cite{Adve1997} \cite[\S7.1, 176--183]{Bailey2007} \cite[\S5.2.4, pp. 82--83; and \S5.4.2, pp. 90--92]{Debbabi2010} \cite[Chapter 7, pp. 109--117]{Ford2008} \cite[\S35.5, pp. 35-10 -- 35-14]{Lee2008b} \cite{Boulanger2013}. These phases cannot be highly overlapped consider by more than a significant amount.
\ \\

	To  functionally debug software with success, these four steps should be carried out: ``test input generation, error detection, error diagnosis, and error correction'' \cite{Kirovski1997}. 
	
	

%%%%%%%%%%%%%%%%%%%%%%%%%%%%%%%%%%%%%%%%%%%
\subsubsection{Interpreting {\it C++} Compilation Errors}
\label{sssec:InterpretingCppCompilationErrors}

	A missing semiconlon ``;'', or lack of matching curly braces ``\}'' (or parentheses ``)'', square brackets ``]'', or angle brackets ``$>$''), can cause a lot of compilation errors \cite{Husted2000}.






%%%%%%%%%%%%%%%%%%%%%%%%%%%%%%%%%%%%%%%%%%%
\subsection{Parser Development}
\label{ssec:ParserDevelopment}

Parser development via {\it Lex/Yacc}, {\it Flex/Bison}, {\it ANTLR}, {\it Parsec}, {\it Ragel}, {\it Spirit Parser Framework}, {\it JetPAG (Jet Parser Auto-Generator)}, {\it Monkey}, {\it MyParser}, {\it SableCC}, \dots \\

	Reference: Wikipedia contributors, ``Comparison of parser generators,'' in {\it Wikipedia, The Free Encyclopedia: Parser generators}, Wikimedia Foundation, San Francisco, CA, February 18, 2016. Available online at: \url{https://en.wikipedia.org/wiki/Comparison_of_parser_generators}; last accessed on February 18, 2016. \\

{\it Determine if {\tt LLVM} can be used for parser development.}


An example of a {\it C++} parser is shown in \cite[Chp. 40, pp. 959--993]{Schildt1998a}.











%%%%%%%%%%%%%%%%%%%%%%%%%%%%%%%%%%%%%%%%%%%
\section{Parallel Programming in C++}
\label{sec:ParallelProgrammingInCpp}


Resources for parallel programming in C++: \vspace{-0.3cm}
\begin{enumerate} \itemsep -4pt
\item C++ -based MPI programming: \cite{Karniadakis2003}
\end{enumerate}











%%%%%%%%%%%%%%%%%%%%%%%%%%%%%%%%%%%%%%%%%%%
\section{Numerical Computing in C++}
\label{sec:NumericalComputingInCpp}

Numerical computing resources for {\it C++}: \vspace{-0.3cm}
\begin{enumerate} \itemsep -4pt
\item Scientific computing: \cite{PittFrancis2012}
\item 
\end{enumerate}



















%%%%%%%%%%%%%%%%%%%%%%%%%%%%%%%%%%%%%%%%%
%	Questions
%	This is written by Zhiyang Ong to document my C++ questions.

%	The MIT License (MIT)

%	Copyright (c) <2015> <Zhiyang Ong>

%	Permission is hereby granted, free of charge, to any person obtaining a copy of this software and associated documentation files (the "Software"), to deal in the Software without restriction, including without limitation the rights to use, copy, modify, merge, publish, distribute, sublicense, and/or sell copies of the Software, and to permit persons to whom the Software is furnished to do so, subject to the following conditions:

%	The above copyright notice and this permission notice shall be included in all copies or substantial portions of the Software.

%	THE SOFTWARE IS PROVIDED "AS IS", WITHOUT WARRANTY OF ANY KIND, EXPRESS OR IMPLIED, INCLUDING BUT NOT LIMITED TO THE WARRANTIES OF MERCHANTABILITY, FITNESS FOR A PARTICULAR PURPOSE AND NONINFRINGEMENT. IN NO EVENT SHALL THE AUTHORS OR COPYRIGHT HOLDERS BE LIABLE FOR ANY CLAIM, DAMAGES OR OTHER LIABILITY, WHETHER IN AN ACTION OF CONTRACT, TORT OR OTHERWISE, ARISING FROM, OUT OF OR IN CONNECTION WITH THE SOFTWARE OR THE USE OR OTHER DEALINGS IN THE SOFTWARE.

%	Email address: echo "cukj -wb- 23wU4X5M589 TROJANS cqkH wiuz2y 0f Mw Stanford" | awk '{ sub("23wU4X5M589","F.d_c_b. ") sub("Stanford","d0mA1n"); print $5, $2, $8; for (i=1; i<=1; i++) print "6\b"; print $9, $7, $6 }' | sed y/kqcbuHwM62z/gnotrzadqmC/ | tr 'q' ' ' | tr -d [:cntrl:] | tr -d 'ir' | tr y "\n"

%%%%%%%%%%%%%%%%%%%%%%%%%%%%%%%%%%%%%%%%%%%%%%



%%%%%%%%%%%%%%%%%%%%%%%%%%%%%%%%%%%%%%%%%%%
\chapter{Questions}
\label{chp:Questions}




%%%%%%%%%%%%%%%%%%%%%%%%%%%%%%%%%%%%%%%%%%%
\section{Unresolved C++ Questions}
\label{sec:UnresolvedCppQuestions}

Questions about {\tt C++}: \vspace{-0.3cm}
\begin{enumerate} \itemsep -4pt
\item 
\end{enumerate}








%%%%%%%%%%%%%%%%%%%%%%%%%%%%%%%%%%%%%%%%%%%
\section{Resolved C++ Questions}
\label{sec:ResolvedCppQuestions}


Difference between pointers and references: \vspace{-0.3cm}
\begin{enumerate} \itemsep -4pt
\item Yusuf Kemal {\"{O}}zcan (``BFaceCoder''), ``Is there any difference between pointers and references? [duplicate],'' Stack Exchange Inc., New York, NY, April 18, 2013. Available online from {\it Stack Exchange Inc.: Programmers Stack Exchange: Questions} at: \url{http://programmers.stackexchange.com/questions/195337/is-there-any-difference-between-pointers-and-references}; October 6, 2015 was the last accessed date. \vspace{-0.3cm}
	\begin{enumerate} \itemsep -2pt
	\item Answer from {\it dan1111}, April 18, 2013: \url{http://programmers.stackexchange.com/a/195343} and \url{http://programmers.stackexchange.com/questions/195337/is-there-any-difference-between-pointers-and-references/195343#195343}. 
	\item 
	\end{enumerate}
\item Macneil Shonle and Programmers Stack Exchange contributors, ``What's a nice explanation for pointers? [closed],'' Stack Exchange Inc., New York, NY, July 30, 2015. Available online from {\it Stack Exchange Inc.: Programmers Stack Exchange: Questions} at: \url{http://programmers.stackexchange.com/questions/17898/whats-a-nice-explanation-for-pointers}; October 6, 2015 was the last accessed date. \vspace{-0.3cm}
	\begin{enumerate} \itemsep -2pt
	\item Answer from Kevin, November 10, 2010: \url{http://programmers.stackexchange.com/a/17919} and \url{http://programmers.stackexchange.com/questions/17898/whats-a-nice-explanation-for-pointers/17919#17919}. ``A pointer is a variable that contains an address to a variable. A pointer is both defined and dereferenced (yielding the value stored at the memory location that it points to) with the `$^{\ast}$' operator; the expression is mnemonic.'' \dots\ {\tt char ($^{\ast}$(x())[])()}
	\item Answer from Barfield, November 10, 2010: \url{http://programmers.stackexchange.com/a/18087} and \url{http://programmers.stackexchange.com/questions/17898/whats-a-nice-explanation-for-pointers/18087#18087}. ``Pointer[s] are a bit like the application shortcuts on your desktop.''
	\item Answer from Gulshan, November 10, 2010: \url{http://programmers.stackexchange.com/a/17915} and \url{http://programmers.stackexchange.com/questions/17898/whats-a-nice-explanation-for-pointers/17915#17915}.  Pointers point to instance and static variables. A pointer can point to different variables during the execution of the program, but must point to one variable at any instance (i.e., point in time) during execution. Also, the pointer must point to variables of the same type. Associate a pointer with a variable via the reference to the variable; e.g., {\tt int $^{\ast}$pointer; pointer = \& variable;} \dots\ According to {\it Ptolemy}, December 2, 2010: \url{http://programmers.stackexchange.com/a/23016}. {\tt int $^{\ast}$pointer = \& variable;} creates a pointer to the variable. \dots\ Dereference the pointer (add $^{\ast}$ as a prefix) to store the value of an expression (based on variables, strings, or constants). According to {\it Ptolemy}, {\tt \& variable} is the ``address of the variable'' and it ``represents the literal value for'' the pointer. ``The pointer'' refers to the data that the pointer points to, or something ``pointed to by'' the pointer.
	\item Answer from Sridhar Iyer, November 11, 2010: \url{http://programmers.stackexchange.com/a/18529} and \url{http://programmers.stackexchange.com/questions/17898/whats-a-nice-explanation-for-pointers/18529#18529}. A ``pointer is a variable that store[s] the address of another variable (or just any variable). $^{\ast}$ is used to get the value at the memory location that is stored in the pointer variable. $\&$ operator gives the address of a memory location.''
	\item Answer from {\it rwong}, November 2, 2010: \url{http://programmers.stackexchange.com/a/18054} and \url{http://programmers.stackexchange.com/questions/17898/whats-a-nice-explanation-for-pointers/18054#18054}. Each pointer, which is a special type of variable, must point to only one variable. Variables that are not pointers must not point to anything; however, such variables can be pointed to by any number of pointers.
	\item Answer from {\it back2dos}, November 10, 2010: \url{http://programmers.stackexchange.com/a/18092} and \url{http://programmers.stackexchange.com/questions/17898/whats-a-nice-explanation-for-pointers/18092#18092}. The pointer [variable] interprets the value of the pointer [variable] as the address of another variable that it points to. Hence, the value of the pointer [variable] refers to a specific location in memory (specified by the address), and is called the reference. Dereferencing is the process of accessing the value of the memory location that it points/refers to. That is, $^{\ast}v$ dereferences the value of $v$, and provides the value at the memory location referred to by the address in $v$. $\&v$ provides a reference (or the address of the memory location for $v$) to the variable $v$. 
	\item Answer from {\it Ptolemy}, December 2, 2010: \url{http://programmers.stackexchange.com/a/23016} and \url{http://programmers.stackexchange.com/questions/17898/whats-a-nice-explanation-for-pointers/23016#23016}. At a low level, the concept of memory can be viewed as a massive array. ``Any position in the array'' can be accessed ``by its index location.'' ``Passing the index location rather than copying the entire memory'' is more efficient in terms of performance and memory usage. Hence, ``pointers are useful.'' ``For [a] method to store the index location [of] where all the data [in the array] is stored,'' ``a memory index location'' can be passed in as a parameter. Pointers can be chained indefinitely; ``keep track of how many times [I] need to look at the addresses to find the actual data object.'' While pointers to heap memory are safe, ``pointers to stack memory are dangerous when passed outside the method.''
	\item Also, see \url{http://www.udel.edu/CIS/105/pconrad/03F/2003.fall.doc} by ``P. Conrad.''
	\end{enumerate}
\item \cite[pp. 15, second last paragraph]{Jensen2003} \vspace{-0.3cm}
	\begin{enumerate} \itemsep -2pt
	\item ``The value of a pointer is the address to which it points''; or, the ``the value of a pointer is the address.''
	\end{enumerate}
\item \cite{EliteHussar2010} \vspace{-0.3cm}
	\begin{enumerate} \itemsep -2pt
	\item ``pointers use the $^{\ast}$ and $->$ operators, references use $.$''
	\item ``Both pointers and references let you refer to other objects indirectly.''
	\item ``there is no such thing as a null reference''
	\item ``A reference must always refer to some object.''
	\item {\bf ``As a result, if you have a variable whose purpose is to refer to another object, but it is possible that there might not be an object to refer to, you should make the variable a pointer, because then you can set it to null.''}
	\item {\it ``On the other hand, if the variable must always refer to an object, i.e., if your design does not allow for the possibility that the variable is null, you should probably make the variable a reference.''}
	\item ``Because a reference must refer to an object, C++ requires that references be initialized.'' \dots\ Pointers do not have to be initialized; i.e., pointers can be uninitialized. However, ``uninitialized pointers'' are ``valid but risky.''
	\item Since null references do not exist, references can be used more efficiently than pointers. This is because the validity of a reference does not have to be tested prior to usage.
	\item Before using pointers, they should be tested against null (i.e., check the validity of a reference prior to usage).
	\item ``Pointers may be reassigned to refer to different objects.'' ``A reference \dots\ always refer to the object with which it is initialized.''
	\item ``You should use a pointer whenever you need to take into account the possibility that there's nothing to refer to (in which case you can set the pointer to null) or whenever you need to be able to refer to different things at different times (in which case you can change where the pointer points).''
	\item ``You should use a reference whenever you know there will always be an object to refer to and you also know that once you're referring to that object, you'll never want to refer to anything else.''
	\item ``There is one other situation in which you should use a reference, and that's when you're implementing certain operators. The most common example is operator[]. This operator typically needs to return something that can be used as the target of an assignment.''
	\item ``References, then, are the feature of choice when you know you have something to refer to, when you'll never want to refer to anything else, and when implementing operators whose syntactic requirements make the use of pointers undesirable. In all other cases, stick with pointers.''
	\end{enumerate}
\item Prakash Rajendran, Theodore Logan (Commodore Jaeger), Josh Lee, sbi, Rob$_{\varphi}$, Sudhanshu Aggarwal, lpapp, Alf, Deduplicator, Sam, and Siddhant Saraf, ``What are the differences between a pointer variable and a reference variable in C++?,'' Stack Exchange Inc., New York, NY, March 2, 2015. Available online from {\it Stack Exchange Inc.: Stack Overflow: Questions} at: \url{http://stackoverflow.com/questions/57483/what-are-the-differences-between-a-pointer-variable-and-a-reference-variable-in}; October 8, 2015 was the last accessed date. \vspace{-0.3cm}
	\begin{enumerate} \itemsep -2pt
	\item A pointer can be re-assigned any number of times while a reference can not be re-seated after binding.
	\item Pointers can point nowhere (NULL), whereas reference always refer to an object.
	\item You can't take the address of a reference like you can with pointers.
	\item There's no ``reference arithmetics'' (but you can take the address of an object pointed by a reference and do pointer arithmetics on it as in \&obj + 5).
	\item Use references in function parameters and return types to define useful and self-documenting interfaces.
	\item Use pointers to implement algorithms and data structures.
	\end{enumerate}
\item  \vspace{-0.3cm}
	\begin{enumerate} \itemsep -2pt
	\item 
	\end{enumerate}
\item  \vspace{-0.3cm}
	\begin{enumerate} \itemsep -2pt
	\item 
	\end{enumerate}
\item  \vspace{-0.3cm}
	\begin{enumerate} \itemsep -2pt
	\item 
	\end{enumerate}
\item  \vspace{-0.3cm}
	\begin{enumerate} \itemsep -2pt
	\item 
	\end{enumerate}
\item  \vspace{-0.3cm}
	\begin{enumerate} \itemsep -2pt
	\item 
	\end{enumerate}
\item  \vspace{-0.3cm}
	\begin{enumerate} \itemsep -2pt
	\item 
	\end{enumerate}
\item  \vspace{-0.3cm}
	\begin{enumerate} \itemsep -2pt
	\item 
	\end{enumerate}
\item  \vspace{-0.3cm}
	\begin{enumerate} \itemsep -2pt
	\item 
	\end{enumerate}
\item  \vspace{-0.3cm}
	\begin{enumerate} \itemsep -2pt
	\item 
	\end{enumerate}
\item  \vspace{-0.3cm}
	\begin{enumerate} \itemsep -2pt
	\item 
	\end{enumerate}
\item  \vspace{-0.3cm}
	\begin{enumerate} \itemsep -2pt
	\item 
	\end{enumerate}
\end{enumerate}
















%%%%%%%%%%%%%%%%%%%%%%%%%%%%%%%%%%%%%%%%%
%	Miscellaneous
%	This is written by Zhiyang Ong to document my C++ questions.

%	The MIT License (MIT)

%	Copyright (c) <2015> <Zhiyang Ong>

%	Permission is hereby granted, free of charge, to any person obtaining a copy of this software and associated documentation files (the "Software"), to deal in the Software without restriction, including without limitation the rights to use, copy, modify, merge, publish, distribute, sublicense, and/or sell copies of the Software, and to permit persons to whom the Software is furnished to do so, subject to the following conditions:

%	The above copyright notice and this permission notice shall be included in all copies or substantial portions of the Software.

%	THE SOFTWARE IS PROVIDED "AS IS", WITHOUT WARRANTY OF ANY KIND, EXPRESS OR IMPLIED, INCLUDING BUT NOT LIMITED TO THE WARRANTIES OF MERCHANTABILITY, FITNESS FOR A PARTICULAR PURPOSE AND NONINFRINGEMENT. IN NO EVENT SHALL THE AUTHORS OR COPYRIGHT HOLDERS BE LIABLE FOR ANY CLAIM, DAMAGES OR OTHER LIABILITY, WHETHER IN AN ACTION OF CONTRACT, TORT OR OTHERWISE, ARISING FROM, OUT OF OR IN CONNECTION WITH THE SOFTWARE OR THE USE OR OTHER DEALINGS IN THE SOFTWARE.

%	Email address: echo "cukj -wb- 23wU4X5M589 TROJANS cqkH wiuz2y 0f Mw Stanford" | awk '{ sub("23wU4X5M589","F.d_c_b. ") sub("Stanford","d0mA1n"); print $5, $2, $8; for (i=1; i<=1; i++) print "6\b"; print $9, $7, $6 }' | sed y/kqcbuHwM62z/gnotrzadqmC/ | tr 'q' ' ' | tr -d [:cntrl:] | tr -d 'ir' | tr y "\n"

%%%%%%%%%%%%%%%%%%%%%%%%%%%%%%%%%%%%%%%%%%%%%%



%%%%%%%%%%%%%%%%%%%%%%%%%%%%%%%%%%%%%%%%%%%
\chapter{Miscellaneous}
\label{chp:Miscellaneous}




%%%%%%%%%%%%%%%%%%%%%%%%%%%%%%%%%%%%%%%%%%%
\section{Setting Up Software Development Environment}
\label{sec:SettingUpSoftwareDevelopmentEnvironment}

Setting up software development environment for {\tt C++}: \vspace{-0.3cm}
\begin{enumerate} \itemsep -4pt
\item Platform-independent environments and software: \vspace{-0.3cm}
	\begin{enumerate} \itemsep -2pt
	\item {\it Linux}, {\it Mac OS X}, and {\it Microsoft Windows}: \vspace{-0.2cm}
		\begin{enumerate} \itemsep -2pt
		\item 
		\end{enumerate}
	\item Truly platform independent: \vspace{-0.2cm}
		\begin{enumerate} \itemsep -2pt
		\item 
		\end{enumerate}
	\end{enumerate}
\item {\it Mac OS X}: \vspace{-0.3cm}
	\begin{enumerate} \itemsep -2pt
	\item Integrated development environments (IDEs): \vspace{-0.2cm}
		\begin{enumerate} \itemsep -2pt
		\item {\it Xcode}: \vspace{-0.1cm}
			\begin{enumerate} \itemsep -1pt
			\item {\it Preferences} $\Longrightarrow$ {\it Text Editing} $\Longrightarrow$ {\it Editing} $\Longrightarrow$ {\it Code folding ribbon}
			\item {\it Preferences} $\Longrightarrow$ {\it Text Editing} $\Longrightarrow$ {\it Indentation} $\Longrightarrow$ {\it Syntax-aware indenting: Automatically indent based on syntax} $+$ {\it Indent ``//'' comments one level deeper} $+$ {\it Align consecutive ``//'' comments}
			\end{enumerate}
		\end{enumerate}
	\end{enumerate}
\item {\it Linux}: \vspace{-0.3cm}
	\begin{enumerate} \itemsep -2pt
	\item Text editors: \vspace{-0.2cm}
		\begin{enumerate} \itemsep -2pt
		\item {\it gedit}: \vspace{-0.1cm}
			\begin{enumerate} \itemsep -1pt
			\item 
			\end{enumerate}
		\item {\it NEdit}: \vspace{-0.1cm}
			\begin{enumerate} \itemsep -1pt
			\item 
			\end{enumerate}
		\end{enumerate}
	\item 
	\end{enumerate}
\end{enumerate}








%%%%%%%%%%%%%%%%%%%%%%%%%%%%%%%%%%%%%%%%%%%
\section{Software Dependencies of The Boilerplate Code Project}
\label{sec:SoftwareDependenciesOfTheBoilerplateCodeProject}


The software dependencies of the ``Boilerplate Code'' project are found in the following {\it Markdown} file: {\tt \dots/lamiera-per-caldaie/notes/miscellaneo/software-dependencies.md}.














%%%%%%%%%%%%%%%%%%%%%%%%%%%%%%%%%%%%%%%%%
%	Bibliography
%\input{./others/bibliography}


%%%%%%%%%%%%%%%%%%%%%%%%%%%%%%%%%%%%%%%%%%%
\chapter*{Acknowledgments}
\addcontentsline{toc}{chapter}{Acknowledgments}
\label{chp:Acknowledgments}

I would like to thank Ms. Deepika Panchalingam for motivating me to revise basic data structures and algorithms for internship and job interviews.

I would also like to thank the following users on {\tt Stack Overflow} for their help: sdsmith2016 \cite{sdsmith2016}.

%	References: \vspace{-0.3cm}
%	\begin{enumerate} \itemsep -4pt
%	\item sdsmith, comment to the question ``Why is the discrepancy in these two cases of using C++ templates?,'' Stack Exchange Inc., New York, NY, March 30, 2016. Available online from {\it Stack Exchange Inc.: Stack Overflow: Questions} at: \url{http://stackoverflow.com/q/36319028/1531728}; March 30, 2016 was the last accessed date.
%	\end{enumerate}


%%%%%%%%%%%%%%%%%%%%%%%%%%%%%%%%%%%%%%%%%%%%%
%%%%%%%%%%%%%%%%%%%%%%%%%%%%%%%%%%%%%%%%%%%%%
%
%	End of document
%
%	Inserting references
%
%%%%%%%%%%%%%%%%%%%%%%%%%%%%%%%%%%%%%%%%%%%%%
%%%%%%%%%%%%%%%%%%%%%%%%%%%%%%%%%%%%%%%%%%%%%
%	Beginning of BACK MATTER: bibliography, indexes and colophon
%\backmatter
\appendix

{\linespread{1}
\bibliographystyle{plain}
%\bibliography{./references/references}
%\bibliography{~/Documents/ricerca/antipastobibtex/references}
%\bibliography{/Users/zhiyang/Documents/ricerca/antipastobibtex/references}
%\bibliography{/Users/zhiyang/Documents/ricerca/lassi-bibtex/references}
\bibliography{/Users/zhiyang/Documents/ricerca/saag-bibtex/references}
%\bibliography{/data/research/antipastobibtex/references}
\addcontentsline{toc}{chapter}{Bibliography}
}
\end{document}