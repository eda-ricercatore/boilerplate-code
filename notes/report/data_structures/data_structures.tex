%	This is written by Zhiyang Ong to document data structures that I have implemented for my C++ -based boilerplate code repository.

%	The MIT License (MIT)

%	Copyright (c) <2014> <Zhiyang Ong>

%	Permission is hereby granted, free of charge, to any person obtaining a copy of this software and associated documentation files (the "Software"), to deal in the Software without restriction, including without limitation the rights to use, copy, modify, merge, publish, distribute, sublicense, and/or sell copies of the Software, and to permit persons to whom the Software is furnished to do so, subject to the following conditions:

%	The above copyright notice and this permission notice shall be included in all copies or substantial portions of the Software.

%	THE SOFTWARE IS PROVIDED "AS IS", WITHOUT WARRANTY OF ANY KIND, EXPRESS OR IMPLIED, INCLUDING BUT NOT LIMITED TO THE WARRANTIES OF MERCHANTABILITY, FITNESS FOR A PARTICULAR PURPOSE AND NONINFRINGEMENT. IN NO EVENT SHALL THE AUTHORS OR COPYRIGHT HOLDERS BE LIABLE FOR ANY CLAIM, DAMAGES OR OTHER LIABILITY, WHETHER IN AN ACTION OF CONTRACT, TORT OR OTHERWISE, ARISING FROM, OUT OF OR IN CONNECTION WITH THE SOFTWARE OR THE USE OR OTHER DEALINGS IN THE SOFTWARE.

%	Email address: echo "cukj -wb- 23wU4X5M589 TROJANS cqkH wiuz2y 0f Mw Stanford" | awk '{ sub("23wU4X5M589","F.d_c_b. ") sub("Stanford","d0mA1n"); print $5, $2, $8; for (i=1; i<=1; i++) print "6\b"; print $9, $7, $6 }' | sed y/kqcbuHwM62z/gnotrzadqmC/ | tr 'q' ' ' | tr -d [:cntrl:] | tr -d 'ir' | tr y "\n"

%%%%%%%%%%%%%%%%%%%%%%%%%%%%%%%%%%%%%%%%%%%%%%



%%%%%%%%%%%%%%%%%%%%%%%%%%%%%%%%%%%%%%%%%%%
\chapter{Data Structures}
\label{chp:DataStructures}


% Check all books with the following keywords in my BibTeX database:
%	data structure analysis
%	data structures



%%%%%%%%%%%%%%%%%%%%%%%%%%%%%%%%%%%%%%%%%%%
\section{Basic Data Structures}
\label{sec:BasicDataStructures}

``[A list is a] container of variable length , and [a tuple is a] container [of] fixed length'' \cite[\S4.3 pp. 111]{Tate2010}.

A list (in {\it Prolog}) can be deconstructed into $[${\it Head} $|$ {\it Tail}$]$, where {\it Head} refers to the first element of the list and {\it Tail} refers to the rest of the list; on the other hand, tuples cannot be similarly deconstructed \cite[\S4.3 pp. 113]{Tate2010}.





%%%%%%%%%%%%%%%%%%%%%%%%%%%%%%%%%%%%%%%%%%%
\section{Graphs}
\label{sec:Graphs}

A graph $G$ is an ordered pair, $G = (V,E)$, of a vertex/node set and an edge set.

Types of finite graphs: \vspace{-0.3cm}
\begin{enumerate} \itemsep -4pt
\item undirected graph: \vspace{-0.3cm}
	\begin{enumerate} \itemsep -2pt
	\item simple graph: \vspace{-0.2cm}
		\begin{enumerate} \itemsep -2pt
		\item Does not allow multiple edges nor loops.
		\item Therefore, the edges of a simple graph form a set, as opposed to multigraphs that have multisets.
		\item An edge is a two-element subset of $V$; other graphs (i.e., multiple graphs) can have more than two nodes.
		\end{enumerate}
	\item hypergraph
	\end{enumerate}
\item directed graph: \vspace{-0.3cm}
	\begin{enumerate} \itemsep -2pt
	\item directed acyclic graphs (DAGs)
	\end{enumerate}
\end{enumerate}


%%%%%%%%%%%%%%%%%%%%%%%%%%%%%%%%%%%%%%%%%%%
\subsection{Graph Representations}
\label{ssec:GraphRepresentations}

Focus on sparse graph representations, which are common in modeling digital integrated circuits and neural networks (certain types), and dense graphs (e.g., neural networks). \\

For sparse graphs, use list or map -based graph representations for better memory efficiency. \\

For dense graphs, use matrix-based graph representation for faster access time at the expense of worse member efficiency. \\

Hence, there exists a trade-off between access time and member efficiency in graph representations.\\

The ways to represent graphs are listed as follows: \vspace{-0.3cm}
\begin{enumerate} \itemsep -4pt
\item adjacency matrix: \vspace{-0.3cm}
	\begin{enumerate} \itemsep -2pt
	\item 
	\end{enumerate}
\item adjacency list: \vspace{-0.3cm}
	\begin{enumerate} \itemsep -2pt
	\item 
	\end{enumerate}
\item adjacency map: \vspace{-0.3cm}
	\begin{enumerate} \itemsep -2pt
	\item 
	\end{enumerate}
\item edge list: \vspace{-0.3cm}
	\begin{enumerate} \itemsep -2pt
	\item Is this equivalent to the ``incidence list'' graph representation? {\Huge Cite this!!!}
	\item 
	\end{enumerate}
\end{enumerate}


Alternate graph representations that I am not exploring: \vspace{-0.3cm}
\begin{enumerate} \itemsep -4pt
\item distance matrix
\item incidence matrix
\end{enumerate}









%%%%%%%%%%%%%%%%%%%%%%%%%%%%%%%%%%%%%%%%%%%
\subsection{Directed Graphs}
\label{ssec:DirectedGraphs}



%%%%%%%%%%%%%%%%%%%%%%%%%%%%%%%%%%%%%%%%%%%
\subsubsection{Functions that need to be implemented}
\label{sssec:FunctionsThatNeedToBeImplemented}




Solvers for the following problems (or to perform the following functions) regarding: \vspace{-0.3cm}
\begin{enumerate} \itemsep -4pt
\item 
\end{enumerate}





Solvers for the following problems (or to perform the following functions) regarding subgraphs, induced subgraphs, and minors: \vspace{-0.3cm}
\begin{enumerate} \itemsep -4pt
\item subgraph isomorphism problem: \vspace{-0.3cm}
	\begin{enumerate} \itemsep -2pt
	\item Find a fixed graph as a subgraph in a given graph: \vspace{-0.2cm}
		\begin{enumerate} \itemsep -2pt
		\item ``graph properties are hereditary for subgraphs''\dots\ ``A graph has a property if and only if all its subgraphs also have it''; see \url{https://en.wikipedia.org/wiki/Graph_theory}.
		\item Finding a specific type/kind of maximal subgraph is an NP-complete problem, such as the largest complete subgraph.
		\end{enumerate}
	\end{enumerate}
\item Finding induced subgraphs in a given graph: \vspace{-0.3cm}
	\begin{enumerate} \itemsep -2pt
	\item ``graph properties are hereditary'' for induced subgraphs\dots\ ``A graph has a property if and only if all its induced subgraphs also have it''; see \url{https://en.wikipedia.org/wiki/Graph_theory}.
	\item Finding a specific type/kind of maximal induced subgraph is an NP-complete problem: \vspace{-0.2cm}
		\begin{enumerate} \itemsep -2pt
		\item Independent set problem: Finding the largest edgeless induced subgraph (or independent set); see the following references: \vspace{-0.1cm}
			\begin{enumerate} \itemsep -1pt
			\item \url{https://en.wikipedia.org/wiki/Graph_theory}
			\end{enumerate}
		\end{enumerate}
	\end{enumerate}
\item : \vspace{-0.3cm}
	\begin{enumerate} \itemsep -2pt
	\item 
	\end{enumerate}
\item : \vspace{-0.3cm}
	\begin{enumerate} \itemsep -2pt
	\item 
	\end{enumerate}
\item : \vspace{-0.3cm}
	\begin{enumerate} \itemsep -2pt
	\item 
	\end{enumerate}
\item 
\item 
\item 
\item 
\item 
\end{enumerate}

%%%%%%%%%%%%%%%%%%%%%%%%%%%%%%%%%%%%%%%%%%%
\subsubsection{Binary Decision Diagrams (BDDs)}
\label{sssec:BinaryDecisionDiagramsBDDs}




%%%%%%%%%%%%%%%%%%%%%%%%%%%%%%%%%%%%%%%%%%%
\subsubsection{AND-Inverter Graphs (AIGs)}
\label{sssec:ANDInverterGraphsAIGs}













%%%%%%%%%%%%%%%%%%%%%%%%%%%%%%%%%%%%%%%%%%%
\subsection{Undirected Graphs}
\label{ssec:UndirectedGraphs}















