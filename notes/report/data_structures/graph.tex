%	This is written by Zhiyang Ong to document data structures that I have implemented for my C++ -based boilerplate code repository.

%	The MIT License (MIT)

%	Copyright (c) <2014> <Zhiyang Ong>

%	Permission is hereby granted, free of charge, to any person obtaining a copy of this software and associated documentation files (the "Software"), to deal in the Software without restriction, including without limitation the rights to use, copy, modify, merge, publish, distribute, sublicense, and/or sell copies of the Software, and to permit persons to whom the Software is furnished to do so, subject to the following conditions:

%	The above copyright notice and this permission notice shall be included in all copies or substantial portions of the Software.

%	THE SOFTWARE IS PROVIDED "AS IS", WITHOUT WARRANTY OF ANY KIND, EXPRESS OR IMPLIED, INCLUDING BUT NOT LIMITED TO THE WARRANTIES OF MERCHANTABILITY, FITNESS FOR A PARTICULAR PURPOSE AND NONINFRINGEMENT. IN NO EVENT SHALL THE AUTHORS OR COPYRIGHT HOLDERS BE LIABLE FOR ANY CLAIM, DAMAGES OR OTHER LIABILITY, WHETHER IN AN ACTION OF CONTRACT, TORT OR OTHERWISE, ARISING FROM, OUT OF OR IN CONNECTION WITH THE SOFTWARE OR THE USE OR OTHER DEALINGS IN THE SOFTWARE.

%	Email address: echo "cukj -wb- 23wU4X5M589 TROJANS cqkH wiuz2y 0f Mw Stanford" | awk '{ sub("23wU4X5M589","F.d_c_b. ") sub("Stanford","d0mA1n"); print $5, $2, $8; for (i=1; i<=1; i++) print "6\b"; print $9, $7, $6 }' | sed y/kqcbuHwM62z/gnotrzadqmC/ | tr 'q' ' ' | tr -d [:cntrl:] | tr -d 'ir' | tr y "\n"

%%%%%%%%%%%%%%%%%%%%%%%%%%%%%%%%%%%%%%%%%%%%%%



%%%%%%%%%%%%%%%%%%%%%%%%%%%%%%%%%%%%%%%%%%%
\section{Graphs}
\label{sec:Graphs}

A {\bf graph} $G$ is an ordered pair, $G = (V,E)$, of a vertex/node set and an edge set \cite{WikipediaContributors2018a38}.

Types of {\bf finite graphs} \cite{WikipediaContributors2018a38}: \vspace{-0.2cm}
\begin{enumerate} \itemsep -4pt
\item {\bf undirected graph} \cite{WikipediaContributors2018a38}: \vspace{-0.2cm}
	\begin{enumerate} \itemsep -2pt
	\item {\bf simple graph}, or {\bf undirected simple finite graph} \cite{WikipediaContributors2018a38}: \vspace{-0.2cm}
		\begin{enumerate} \itemsep -1pt
		\item Does not allow {\bf multiple edges} (or, {\bf parallel edges}, or {\bf multi-edges}) between any pair of vertices/nodes in the graph, nor {\bf (self-)loops}.
		\item A {\bf symmetric loopless directed graph} is a graph that has an edge $(v_{2}, v_{1})$ for each edge $(v_{1}, v_{2})$, and it does not contain any self-loops (i.e., edges $v_{i}, v_{i}, \forall i \in V$).
		\item Therefore, the edges of a simple graph form a set, as opposed to multigraphs that have multisets of edges.
		\item An edge is a two-element subset of $V$; other graphs (i.e., hypergraphs) can have edges with more than two nodes.
		\end{enumerate}
	\item {\bf multigraph} \cite{WikipediaContributors2018a39}: \vspace{-0.2cm}
		\begin{enumerate} %\itemsep -1pt
		\item A {\bf multigraph} allows {\bf multiple edges} (or, {\bf parallel edges}, or {\bf multi-edges}) to exist between any pair of vertices (or, nodes) in the graph. Alternatively, for any pair of vertices (or, nodes) in the multigraph, they allow multiple edges to be incident to them \cite{WikipediaContributors2018a39}.
		\item {\bf pseudograph} \cite{WikipediaContributors2018a39}: \vspace{-0.1cm}
			\begin{itemize} \itemsep -1pt
			\item Some authors/people use pseudographs and multigraphs interchangeably/synonymously.
			\item Other people use pseudographs to refer to multigraphs that allow self-loops (or loops).
			\end{itemize}
		\item A {\bf planar graph} remains planar, or preserves its planarity, when its edges become multiple edges by the addition of edges to edges in the original graph {\Huge Reference this!!!}.
		\item Representations for different types of {\bf multigraphs} \cite{WikipediaContributors2018a39}: \vspace{-0.1cm}
			\begin{itemize} \itemsep -1pt
			\item A {\bf multidigraph} is also known as {\bf quiver} \cite{WikipediaContributors2018a40}.
			\item {\bf undirected multigraph, which edges have no identity}, is an ordered pair (or 2-tuple) $G = (V, E)$, where $V$ is a set of vertices (or, nodes), $E$ is a multi-set of undirected edges (or unordered pairs of vertices).
			\item {\bf undirected multigraph, which edges have an identity each (or, where each edge has an identity)}, is an ordered triple (or 3-tuple) $G = (V, E, r)$, such that $r: E \rightarrow \{\{x, y\} : x, y \in V\}$ is a function that assigns each edge to an unordered pair of vertices (i.e., endpoint nodes).
			\item {\bf directed multigraph (or multidigraph or quiver), which edges have no identity}, is an ordered pair (or 2-tuple) $G = (V,E)$. Here, $V$ is a set of vertices (or, nodes), and $E$ is a multi-set of ordered pairs of vertices (i.e., directed edges, directed arcs, or arrows)
			\item A {\bf mixed multigraph} is a(n) (ordered) 3-tuple $G = (V, E, A)$, where $V$ is a set of vertices (or, nodes), $E$ is a set of undirected edges, and $A$ is a multi-set of directed edges/arcs.
			\item A {\bf directed multigraph (or multidigraph or quiver), which edges have an identity each (or, where each edge has an identity)}, is a 4-tuple $G = (V, E, s, t)$. Here, $V$ is a set of vertices (or nodes), $E$ is a multi-set of ordered pairs of vertices (i.e., directed edges, directed arcs, or arrows), $s : E \rightarrow V$ so that each edge is assigned to its source node, and $t : E \rightarrow V$ so that each edge is assigned to its destination/target node.
			\end{itemize}
		\item A {\bf labeled multigraph} is a 6-tuple $G = (\sum_{V}, \sum_{E}, V, E, l_{V}, l_{E})$. Here, $V$ is a set of vertices (or, nodes), and $E$ is a multi-set of ordered pairs of vertices (i.e., directed edges, directed arcs, or arrows), $\sum_{V}$ is the finite alphabet of available vertex labels, $\sum_{E}$ is the finite alphabet of available edge labels, $l_{V} : V \rightarrow \sum_{V}$ is a map describing the labeling of the vertices, and $l_{E} : E \rightarrow \sum_{E}$ is a map describing the labeling of the vertices.
		\item A {\bf labeled multidigraph} is also known as a {\bf labeled, directed multigraph}. It is a 8-tuple $G = (\sum_{V}, \sum_{E}, V, E, s, t, l_{V}, l_{E})$. Here, $V$ is a set of vertices (or, nodes), and $E$ is a multi-set of ordered pairs of vertices (i.e., directed edges, directed arcs, or arrows), $\sum_{V}$ is the finite alphabet of available vertex labels, $\sum_{E}$ is the finite alphabet of available edge labels, $s : E \rightarrow V$ so that each edge is assigned to its source node (or, is a map that assigns each edge to its source node), $t : E \rightarrow V$ so that each edge is assigned to its destination/target node (or, is a map that assigns each edge to its destination/target node), $l_{V} : V \rightarrow \sum_{V}$ is a map describing the labeling of the vertices, and $l_{E} : E \rightarrow \sum_{E}$ is a map describing the labeling of the vertices.
		\item References: %\vspace{-0.1cm}
			\begin{itemize} %\itemsep -1pt
			\item {\bf Multiple edges}: \url{https://en.wikipedia.org/wiki/Multiple_edges}
			\item {\bf Multigraph} \cite{WikipediaContributors2018a39}: \url{https://en.wikipedia.org/wiki/Multigraph}
			\item {\bf Graph labeling} \cite{WikipediaContributors2017a12}: %\vspace{-0.1cm}
				\begin{itemize} %\itemsep -1pt
				\item \url{https://en.wikipedia.org/wiki/Graph_labeling}
				\item ``{\bf Graph labeling} is the assignment of labels,'' which can be represented by numbers and/or strings, to edges and/or vertices of a graph.
				\item {\bf Vertex labeling} is a function of $V$\ that assigns a set of labels to $V$; or, it is a function of $V$that assigns a label to each vertex.
				\item A {\bf vertex-labeled graph} is a graph with a defined vertex labeling function.
				\item {\bf Edge labeling} is a function of $E$\ that assigns a set of labels to $E$; or, it is a function of $E$that assigns a label to each edge.
				\item An {\bf edge-labeled graph} is a graph with a defined edge labeling function.
				\item A {\bf weighted graph} is an {\bf edge-labeled graph}, such that the edge labels are members of an ordered set (e.g., the set of real numbers $\mathbb{R}$).
				\item The term {\bf labeled graph} generally refers to a {\bf vertex-labeled graphs} with unique labels (e.g., $\{1, \dots, |V|\}$, where $|V|$ is the number of vertices in the graph or the cardinality of $V$), unless otherwise specified.
				\end{itemize}
			\end{itemize}
		\end{enumerate}
	\item {\bf hypergraph}: \vspace{-0.2cm}
		\begin{enumerate} \itemsep -2pt
		\item 
		\end{enumerate}
	\item {\bf multidimensional networks}: \vspace{-0.2cm}
		\begin{enumerate} \itemsep -2pt
		\item One of the distinguishing features of multi-dimensional edges is multiple edges.
		\end{enumerate}
	\item {\bf mixed graph}: \vspace{-0.2cm}
		\begin{enumerate} \itemsep -2pt
		\item 
		\end{enumerate}
	\item {\bf planar graph}: \vspace{-0.2cm}
		\begin{enumerate} \itemsep -2pt
		\item 
		\end{enumerate}
	\item {\bf dipole graph}: \vspace{-0.2cm}
		\begin{enumerate} \itemsep -2pt
		\item A {\bf dipole graph} (or {\bf bond graph} -- not that kind of {\bf bond graph}) has a set of only two vertices, and a set of (parallel) edges between these vertices.: \vspace{-0.1cm}
			\begin{itemize} \itemsep -1pt
			\item A {\bf bond graph} is a graphical representation of a physical dynamic system, and represents exchanges of physical energy; see \url{https://en.wikipedia.org/wiki/Bond_graph}.
			\end{itemize}
		\item An {\bf order-$n$ dipole graph} $Dn$ is a dipole graph with $n$ edges, and is a dual to the cycle graph $C_{n}$.
		\item References: \vspace{-0.1cm}
			\begin{itemize} \itemsep -1pt
			\item {\bf Multiple edges}: \url{https://en.wikipedia.org/wiki/Multiple_edges}
			\item {\bf Dipole graph}: \url{https://en.wikipedia.org/wiki/Dipole_graph}
			\end{itemize}
		\end{enumerate}
	\end{enumerate}
\item {\bf directed graph}: \vspace{-0.3cm}
	\begin{enumerate} \itemsep -2pt
	\item {\bf directed multigraphs}: see aforementioned notes on ``{\bf directed multigraph} (or {\bf multidigraph}), which edges have no identity'' and ``{\bf directed multigraph} (or {\bf multidigraph}), which edges have an identity each (or, where each edge has an identity)''
	\item {\bf directed acyclic graphs} ({\bf DAGs})
	\end{enumerate}
\end{enumerate}







Additional resources about graphs: \vspace{-0.3cm}
\begin{enumerate} \itemsep -4pt
\item {\bf Graph property}, or {\bf graph invariant}: \vspace{-0.3cm}
	\begin{enumerate} \itemsep -2pt
	\item \url{https://en.wikipedia.org/wiki/Graph_property}
	\item Notes about vertices/nodes: \vspace{-0.2cm}
		\begin{enumerate} %\itemsep -1pt
		\item If the graph contains an edge connecting vertices $u$ and $v$, $e = (u, v)$, the vertices $u$ and $v$ are {\bf adjacent} to each other \cite{WikipediaContributors2017a11}.
		\item The {\bf degree (or valency) of vertex} $v$, which is denoted by $\delta(v)$ (or, ${\rm deg}(v)$ or ${\rm deg}\ v$), is the number of edges that are incident to $v$ \cite{WikipediaContributors2017a11}.
		\item An {\bf isolated vertex} is a vertex with degree zero, $\delta(v) = 0$. It does not an endpoint of any edge; or, no edge in a (or, any) graph is defined with an isolated vertex \cite{WikipediaContributors2017a11}.
		\item A {\bf leaf vertex}, or {\bf pendent vertex}, is a vertex of degree one, and is an endpoint of only one edge in the graph \cite{WikipediaContributors2017a11}.
		\item For a vertex $v$ of a directed graph, the {\bf outdegree} of $v$ (denoted by $\delta^{+}(v)$) is its number of outgoing edges, and the {\bf indegree} of $v$ (denoted by $\delta^{-}(v)$) is its number of incoming edges. A source vertex is a vertex with a zero indegree ($\delta^{-}(v) = 0$), and a sink vertex is a vertex with a zero outdegree ($\delta^{+}(v) = 0$) \cite{WikipediaContributors2017a11}.
		\item A vertex $v$, which has no incident edges apart from a(n) (undirected) loop from $v$ to itself, has a degree of two; i.e., ${\rm deg}(v) = 0$ \cite{WikipediaContributors2018a37}.
		\item For a graph $G$, its {\bf maximum degree} $\Delta(G)$ and its {\bf minimum degree} $\delta(G)$ are the maximum and minimum degree of its vertices \cite{WikipediaContributors2018a37}.
		\end{enumerate}
	\end{enumerate}
\item {\bf Multi-dimensional networks}: \vspace{-0.3cm}
	\begin{enumerate} \itemsep -2pt
	\item {\bf Multi-dimensional networks} belong to type of multi-layer networks that have multiple types/kinds of relations \cite{WikipediaContributors2018a42}.
	\item A {\bf multi-dimensional network} can be modeled with a {\bf multipartite edge-labeled multigraph} \cite{WikipediaContributors2018a41,WikipediaContributors2018a42}.
	\item An {\bf unweighted multi-layer network} can be represented as a triple $G = (V, E, D)$, where (or, in which) $V$ is a set of vertices, $E$ is a dimension-specific set of edges connecting the vertices and each edge is represented by the triple $(u, v, d)$ such that $u, v, \in V$ and $d \in D$ (or, $E = \{(u, v, d); u, v \in V, d \in D$), and $D$ is a set of dimensions or layers \cite{WikipediaContributors2018a42}. %\vspace{-0.2cm}
		\begin{enumerate} %\itemsep -2pt
		\item For an {\bf unweighted, undirected multi-layer network}, the edges/links $(u, v, d)$ and $(v, u, d)$ are equivalent.
		\item For an {\bf unweighted, directed multi-layer network}, the edges/links $(u, v, d)$ and $(v, u, d)$ are different/distinct.
		\item By convention, {\bf unweighted multi-layer network} are not {\bf multigraphs} in a given dimension; hence, ``the number of $[$edges/$]$links between two nodes in a given dimension is either zero or one''. ``However, the total number of $[$edges/$]$links between two nodes across all dimensions is less than or equal to $| D |$.''
		\end{enumerate}
	\item An edge of a {\bf weighted multi-layer network} can be represented as a 4-tuple (or, quadruplet) $e = (u, v, d, w)$, ``where $w$ is the weight of the $[$edge/$]$link between $[$vertex$]$ $u$ and $[$vertex$]$ $v$ in the dimension $d$'' \cite{WikipediaContributors2018a42} %\vspace{-0.2cm}
		\begin{enumerate} %\itemsep -2pt
		\item {\bf Weighted multi-layer networks}, like {\bf unweighted multi-layer networks}, can also be represented as a triple $G = (V, E, D)$, where $V$ is the set of vertices, $E$ is the set of edges (each edge is represented by $e = (u, v, d, w)$), or $E = \{(u, v, d, w); u, v \in V, d \in D, w \in W\}$, $W$ is a set of weights, and $D$ is a set of dimensions or layers.
		\item A {\bf weighted multi-layer networks}, where an edge is defined as $e = (u, v, d_{1}, \dots, d_{n}, w)$, such that $d_{1}, \dots, d_{n} \in D$, can model {\bf multidimensional temporal networks}, or {\bf multidimensional time-varying networks} \cite{WikipediaContributors2018a42}.
		\end{enumerate}
	\item {\bf Multidimensional network} \cite{WikipediaContributors2018a42}: \url{https://en.wikipedia.org/wiki/Multidimensional_network}.
	\item {\bf Multipartite graph} \cite{WikipediaContributors2018a41}: \url{https://en.wikipedia.org/wiki/Multipartite_graph}
	\item {\bf Temporal network}, or {\bf time-varying network}: \url{https://en.wikipedia.org/wiki/Temporal_network}
	\item A {\bf 1-dimensional (1-D) network} has a {\bf 2-dimensional (2-D) adjacency matrix} with the size $V \times V$; $A^{i}_{j}$ is an adjacency matrix that encodes edges/links/connections between vertices $i$ and $j$.
	\item For a {\bf multi-dimensional network} with $| D |$ dimensions, it has a {\bf multi-layer adjacency tensor (or, 4-dimensional matrix, or 4-D matrix)} with the size $(V \times D) \times (V \times D)$; $M^{i\alpha}_{j\beta}$ is an {\bf multi-layer adjacency tensor} that encodes edges/links/connections between vertices $i$ in layer $\alpha$ and $j$ in layer $\beta$ \cite{WikipediaContributors2018a42}.
	\end{enumerate}
\item {\bf infinite graph}: \vspace{-0.3cm}
	\begin{enumerate} \itemsep -2pt
	\item An {\bf infinite graph} is a graph that is not finite. This can be worded better.
	\end{enumerate}
\item {\bf extremal graphs}: \vspace{-0.3cm}
	\begin{enumerate} \itemsep -2pt
	\item ``{\bf Extremal graph theory} studies {\bf extremal (maximal or minimal) graphs} which satisfy a certain property. Extremality can be taken with respect to different graph invariants, such as order, size or girth.''
	\item \url{https://en.wikipedia.org/wiki/Extremal_graph_theory}
	\end{enumerate}
\end{enumerate}




%%%%%%%%%%%%%%%%%%%%%%%%%%%%%%%%%%%%%%%%%%%
\subsection{Graph Representations}
\label{ssec:GraphRepresentations}

Focus on {\bf sparse graph representations}, which are common in modeling digital integrated circuits and neural networks (certain types), and {\bf dense graphs} (e.g., neural networks). \\

For {\bf sparse graphs}, use {\bf list or map -based graph representations} for better {\bf memory efficiency}. \\

For {\bf dense graphs}, use {\bf matrix-based graph representation} for faster {\bf access time} at the expense of worse {\bf member efficiency}; see \url{https://en.wikipedia.org/wiki/Dense_graph}. Also, see \url{https://en.wikipedia.org/wiki/Dense_subgraph} regarding dense subgraphs. \\

Hence, there exists a {\bf trade-off between access time and member efficiency in graph representations}.\\

The ways to represent graphs are listed as follows: \vspace{-0.3cm}
\begin{enumerate} \itemsep -4pt
\item {\bf adjacency matrix}: \vspace{-0.3cm}
	\begin{enumerate} \itemsep -2pt
	\item 
	\end{enumerate}
\item {\bf adjacency list}: \vspace{-0.3cm}
	\begin{enumerate} \itemsep -2pt
	\item 
	\end{enumerate}
\item {\bf adjacency map}: \vspace{-0.3cm}
	\begin{enumerate} \itemsep -2pt
	\item 
	\end{enumerate}
\item {\bf edge list}: \vspace{-0.3cm}
	\begin{enumerate} \itemsep -2pt
	\item Is this equivalent to the ``{\bf incidence list}'' graph representation? {\Huge Cite this!!!}
	\item 
	\end{enumerate}
\end{enumerate}


Alternate graph representations that I am not exploring: \vspace{-0.3cm}
\begin{enumerate} \itemsep -4pt
\item {\bf distance matrix}: \url{https://en.wikipedia.org/wiki/Distance_matrix}
\item {\bf incidence matrix}
\end{enumerate}












%%%%%%%%%%%%%%%%%%%%%%%%%%%%%%%%%%%%%%%%%%%
\subsection{Functions that need to be implemented}
\label{ssec:FunctionsThatNeedToBeImplemented}




Solvers for the following problems (or to perform the following functions) regarding: \vspace{-0.3cm}
\begin{enumerate} \itemsep -4pt
\item {\bf graph traversal}: \vspace{-0.3cm}
	\begin{enumerate} \itemsep -2pt
	\item {\bf breadth-first search}: \vspace{-0.2cm}
		\begin{enumerate} \itemsep -2pt
		\item \url{https://en.wikipedia.org/wiki/Breadth-first_search	}
		\item Also, see {\bf BFS ordering}: \url{https://en.wikipedia.org/wiki/Breadth-first_search}
		\end{enumerate}
	\item {\bf depth-first search}: \vspace{-0.2cm}
		\begin{enumerate} \itemsep -2pt
		\item \url{https://en.wikipedia.org/wiki/Depth-first_search}
		\item {\bf iterative deepening search}, or more specifically {\bf iterative deepening depth-first search (IDS or IDDFS)}: \url{https://en.wikipedia.org/wiki/Iterative_deepening_depth-first_search}
		\end{enumerate}
	\item {\bf graph factorization}: \url{}
	\item References: \vspace{-0.2cm}
		\begin{enumerate} \itemsep -2pt
		\item \url{https://en.wikipedia.org/wiki/Graph_theory}
		\end{enumerate}
	\end{enumerate}
\item {\bf graph coloring}: \vspace{-0.3cm}
	\begin{enumerate} \itemsep -2pt
	\item {\bf vertex coloring}
	\item {\bf edge coloring}: \url{https://en.wikipedia.org/wiki/Edge_coloring}
	\item {\bf four color theorem}, or the {\bf four color map theorem}: \url{https://en.wikipedia.org/wiki/Four_color_theorem}
	\item References: \vspace{-0.2cm}
		\begin{enumerate} \itemsep -2pt
		\item \url{https://en.wikipedia.org/wiki/Graph_coloring}
		\end{enumerate}
	\end{enumerate}
\item {\bf routing problems}: \vspace{-0.3cm}
	\begin{enumerate} \itemsep -2pt
	\item {\bf shortest path problem}: \vspace{-0.2cm}
		\begin{enumerate} \itemsep -2pt
		\item {\bf Dijkstra's algorithm}: \url{https://en.wikipedia.org/wiki/Dijkstra%27s_algorithm}
		\item {\bf Bellman-Ford algorithm} (or, {\bf Bellman-Ford-Moore algorithm}): \url{https://en.wikipedia.org/wiki/Bellman%E2%80%93Ford_algorithm}
		\item {\bf Ford-Fulkerson algorithm (FFA)}, or {\bf Ford-Fulkerson method}: \url{https://en.wikipedia.org/wiki/Ford%E2%80%93Fulkerson_algorithm}
		\item {\bf Categorize solutions} into those for {\bf directed graphs} and {\bf undirected graphs}. Also, determine solutions for common variants of the problem.
		\item \url{https://en.wikipedia.org/wiki/Shortest_path_problem}
		\end{enumerate}
	\item {\bf longest path problem}: \vspace{-0.2cm}
		\begin{enumerate} \itemsep -2pt
		\item Note that the difficulty of the problem (in terms of {\bf computational time complexity}) is different for different types of graphs: \vspace{-0.1cm}
			\begin{itemize} \itemsep -1pt
			\item E.g., for {\bf undirected graphs}, it is {\bf NP-hard}, while {\bf linear time solutions} exist for {\bf directed acyclic graphs (DAGs)}.
			\end{itemize}
		\item \url{https://en.wikipedia.org/wiki/Longest_path_problem}
		\end{enumerate}
	\item {\bf minimum spanning tree}: \vspace{-0.2cm}
		\begin{itemize} \itemsep -2pt
		\item \item {\bf Prim-Jarn{\'{i}}k algorithm}: \vspace{-0.1cm}
			\begin{itemize} \itemsep -1pt
			\item Also, known as: \vspace{-0.1cm}
				\begin{itemize} \itemsep -1pt
				\item {\bf Prim's algorithm}
				\item {\bf Jarn�k's algorithm}
				\item {\bf Prim-Dijkstra algorithm}
				\item {\bf DJP algorithm}
				\end{itemize}
			\item \url{https://en.wikipedia.org/wiki/Prim%27s_algorithm}
			\end{itemize}
		\item {\bf Kruskal's algorithm}: \url{https://en.wikipedia.org/wiki/Kruskal%27s_algorithm}
		\item \url{https://en.wikipedia.org/wiki/Minimum_spanning_tree}
		\end{itemize}
	\item {\bf Steiner tree}: \vspace{-0.2cm}
		\begin{enumerate} \itemsep -2pt
		\item {\bf rectilinear minimum Steiner tree (RMST)} problem: \url{https://en.wikipedia.org/wiki/Rectilinear_Steiner_tree}
		\item \url{https://en.wikipedia.org/wiki/Steiner_tree_problem}
		\end{enumerate}
	\item {\bf traveling salesperson problem} (NP-hard): \vspace{-0.2cm}
		\begin{enumerate} \itemsep -2pt
		\item {\bf nearest neighbor algorithm}: \vspace{-0.1cm}
			\begin{itemize} \itemsep -1pt
			\item \url{https://en.wikipedia.org/wiki/Nearest_neighbour_algorithm}
			\item This is different from the k-nearest neighbors algorithm ($k$-NN); see \url{https://en.wikipedia.org/wiki/K-nearest_neighbors_algorithm}.
			\end{itemize}
		\item \url{https://en.wikipedia.org/wiki/Travelling_salesman_problem}
		\end{enumerate}
	\item {\bf strongly connected components}: \vspace{-0.2cm}
		\begin{enumerate} \itemsep -2pt
		\item \url{https://en.wikipedia.org/wiki/Strongly_connected_component}
		\end{enumerate}
	\end{enumerate}
\item {\bf network flow}: \vspace{-0.3cm}
	\begin{enumerate} \itemsep -2pt
	\item {\bf max-flow min-cut theorem}: \url{https://en.wikipedia.org/wiki/Max-flow_min-cut_theorem}
	\item {\bf minimum-cost flow problem (MCFP)}: \url{https://en.wikipedia.org/wiki/Minimum-cost_flow_problem}
	\item {\bf maximum flow problems}: \url{https://en.wikipedia.org/wiki/Maximum_flow_problem}
	\item {\bf circulation problem}: \url{https://en.wikipedia.org/wiki/Circulation_problem}
	\item References: \vspace{-0.2cm}
		\begin{enumerate} \itemsep -2pt
		\item {\bf Flow network} (or {\bf transportation network}): \url{https://en.wikipedia.org/wiki/Flow_network}
		\item 
		\end{enumerate}
	\end{enumerate}
\item {\bf graph partitioning}: \vspace{-0.3cm}
	\begin{enumerate} \itemsep -2pt
	\item {\bf force-directed graph partitioning}
	\item {\bf min-cut graph partitioning}
	\item References: \vspace{-0.2cm}
		\begin{enumerate} \itemsep -2pt
		\item \url{https://en.wikipedia.org/wiki/Connectivity_(graph_theory)}
		\end{enumerate}
	\end{enumerate}
\item {\bf graph-based floorplanning/placement}: \vspace{-0.3cm}
	\begin{enumerate} \itemsep -2pt
	\item use {\bf constraint graphs} for {\bf graph-based floorplanning/placement}
	\item References: \vspace{-0.2cm}
		\begin{enumerate} \itemsep -2pt
		\item \url{https://en.wikipedia.org/wiki/Constraint_graph_(layout)}
		\end{enumerate}
	\end{enumerate}
\item {\bf covering problems}: %\vspace{-0.3cm}
	\begin{enumerate} %\itemsep -2pt
	\item In graph theory, ``{\bf covering problems} are specific instances of {\bf subgraph-finding problems}''; see \url{https://en.wikipedia.org/wiki/Graph_theory}
	\item {\bf Set cover problem}: \vspace{-0.2cm}
		\begin{enumerate} \itemsep -2pt
		\item \url{https://en.wikipedia.org/wiki/Set_cover_problem}
		\item {\bf hitting set problem}
		\end{enumerate}
	\item {\bf Vertex cover problem}: \url{https://en.wikipedia.org/wiki/Vertex_cover}
	\item {\bf edge cover problem}: \url{https://en.wikipedia.org/wiki/Edge_cover}
	\item Related problems: %\vspace{-0.2cm}
		\begin{enumerate} %\itemsep -2pt
		\item {\bf clique problem}: \url{https://en.wikipedia.org/wiki/Clique_problem}
		\item {\bf independent set problem}, and {\bf maximum independent set}: \url{https://en.wikipedia.org/wiki/Independent_set_(graph_theory)}
		\item {\bf Covering/packing-problem pairs}, or {\bf covering/packing dualities}: \url{https://en.wikipedia.org/wiki/Linear_programming#Covering/packing_dualities}
		\item {\bf Packing problems}: \url{https://en.wikipedia.org/wiki/Packing_problems}
		\item {\bf Reconstruction conjecture}: \vspace{-0.1cm}
			\begin{itemize} \itemsep -1pt
			\item ``Informally, the {\bf reconstruction conjecture} in graph theory says that graphs are determined uniquely by their subgraphs.''
			\item \url{https://en.wikipedia.org/wiki/Reconstruction_conjecture}
			\end{itemize}
		\end{enumerate}
	\item References: \vspace{-0.2cm}
		\begin{enumerate} \itemsep -2pt
		\item \url{https://en.wikipedia.org/wiki/Covering_problems}
		\end{enumerate}
	\end{enumerate}
\item {\bf graph decomposition problems}: \vspace{-0.3cm}
	\begin{enumerate} \itemsep -2pt
	\item {\bf arboricity}: \url{https://en.wikipedia.org/wiki/Arboricity}
	\item {\bf cycle double cover}: \url{https://en.wikipedia.org/wiki/Cycle_double_cover}
	\item {\bf graph factorization}: \url{https://en.wikipedia.org/wiki/Graph_factorization}
	\item : \vspace{-0.2cm}
		\begin{enumerate} \itemsep -2pt
		\item 
		\end{enumerate}
	\item References: \vspace{-0.2cm}
		\begin{enumerate} \itemsep -2pt
		\item \url{https://en.wikipedia.org/wiki/Graph_theory}
		\end{enumerate}
	\end{enumerate}
\item {\bf closure problem}: \vspace{-0.3cm}
	\begin{enumerate} \itemsep -2pt
	\item \url{https://en.wikipedia.org/wiki/Closure_problem}
	\end{enumerate}
\item {\bf spectral graph theory}: \vspace{-0.3cm}
	\begin{enumerate} \itemsep -2pt
	\item ``In mathematics, {\bf spectral graph theory} is the study of the properties of a graph in relationship to the characteristic polynomial, eigenvalues, and eigenvectors of matrices associated with the graph, such as its adjacency matrix or Laplacian matrix.''
	\item References: \vspace{-0.2cm}
		\begin{enumerate} \itemsep -2pt
		\item \url{https://en.wikipedia.org/wiki/Spectral_graph_theory}
		\end{enumerate}
	\end{enumerate}
\item {\bf probabilistic graphical model} ({\bf PGM}): \vspace{-0.3cm}
	\begin{enumerate} \itemsep -2pt
	\item Also known as: \vspace{-0.2cm}
		\begin{enumerate} \itemsep -2pt
		\item {\bf graphical model}
		\item {\bf structured probabilistic model}
		\end{enumerate}
	\item References: \vspace{-0.2cm}
		\begin{enumerate} %\itemsep -2pt
		\item \url{https://en.wikipedia.org/wiki/Graphical_model}
		\item \cite{Barber2012}
		\item \cite{Bishop2006}
		\item Cowell, Robert G.; Dawid, A. Philip; Lauritzen, Steffen L.; Spiegelhalter, David J. (1999). Probabilistic networks and expert systems. Berlin: Springer. A more advanced and statistically oriented book
		\item Jensen, Finn (1996). An introduction to Bayesian networks. Berlin: Springer.
		\item Pearl, Judea (1988). Probabilistic Reasoning in Intelligent Systems (2nd revised ed.). San Mateo, CA: Morgan Kaufmann. 
		\end{enumerate}
	\end{enumerate}
\item {\bf quantum graph}: \vspace{-0.3cm}
	\begin{enumerate} \itemsep -2pt
	\item ``In mathematics and physics, a {\bf quantum graph} is a linear, network-shaped structure of vertices connected by bonds (or edges) with a differential or pseudo-differential operator acting on functions defined on the bonds.''
	\item References: \vspace{-0.2cm}
		\begin{enumerate} \itemsep -2pt
		\item \url{https://en.wikipedia.org/wiki/Quantum_graph}
		\item \cite{Lovasz2012}
		\end{enumerate}
	\end{enumerate}
\end{enumerate}





Solvers for the following problems (or to perform the following functions) regarding subgraphs, induced subgraphs, and minors: \vspace{-0.3cm}
\begin{enumerate} \itemsep -4pt
\item {\bf subgraph isomorphism problem}: \vspace{-0.3cm}
	\begin{enumerate} \itemsep -2pt
	\item {\bf Find a fixed graph as a subgraph in a given graph}: \vspace{-0.2cm}
		\begin{enumerate} \itemsep -2pt
		\item ``graph properties are hereditary for subgraphs''\dots\ ``A graph has a property if and only if all its subgraphs also have it''; see \url{https://en.wikipedia.org/wiki/Graph_theory}.
		\item Finding a specific type/kind of maximal subgraph is an NP-complete problem, such as the largest complete subgraph.
		\end{enumerate}
	\item Also, see {\bf subgraph matching}.
	\item {\bf Graph isomorphism}: \url{https://en.wikipedia.org/wiki/Graph_isomorphism}
	\item \url{https://en.wikipedia.org/wiki/Subgraph_isomorphism_problem}
	\end{enumerate}
\item {\bf Finding induced subgraphs} in a given graph: \vspace{-0.3cm}
	\begin{enumerate} \itemsep -2pt
	\item ``graph properties are hereditary'' for induced subgraphs\dots\ ``A graph has a property if and only if all its induced subgraphs also have it''; see \url{https://en.wikipedia.org/wiki/Graph_theory}.
	\item Finding a specific type/kind of {\bf maximal induced subgraph} is an NP-complete problem: \vspace{-0.2cm}
		\begin{enumerate} \itemsep -2pt
		\item {\bf Independent set problem}: Finding the largest {\bf edgeless induced subgraph} (or {\bf independent set}); see the following references: \vspace{-0.1cm}
			\begin{itemize} \itemsep -1pt
			\item \url{https://en.wikipedia.org/wiki/Graph_theory}
			\end{itemize}
		\end{enumerate}
	\item {\bf Induced subgraph}: \url{https://en.wikipedia.org/wiki/Induced_subgraph}
	\end{enumerate}
\item {\bf minor containment problem}: \vspace{-0.3cm}
	\begin{enumerate} \itemsep -2pt
	\item Find a {\bf fixed graph} as a minor of a given graph.
	\item ``A {\bf minor} or {\bf subcontraction of a graph} is any graph obtained by taking a {\bf subgraph} and contracting some (or no) edges''\dots\ ``A graph has a property if and only if all its {\bf minors} [also] have it''
	\item ``[{\bf Minor containment}] is related to graph properties such as planarity.'' See Wagner's Theorem about planar graphs.
	\item References: \vspace{-0.2cm}
		\begin{enumerate} \itemsep -2pt
		\item \url{https://en.wikipedia.org/wiki/Graph_theory}
		\item Graph minor: \url{https://en.wikipedia.org/wiki/Graph_minor}
		\end{enumerate}
	\end{enumerate}
\item {\bf subdivision containment problems}: \vspace{-0.3cm}
	\begin{enumerate} \itemsep -2pt
	\item ``{\bf Find a fixed graph as a subdivision of a given graph}'': \vspace{-0.2cm}
		\begin{enumerate} \itemsep -2pt
		\item ``A {\bf subdivision} or {\bf homeomorphism} of a graph is any graph obtained by subdividing some (or no) edges.''
		\item ``{\bf Subdivision containment} is related to graph properties such as planarity.'' See Kuratowski's Theorem and the Kelmans-Seymour conjecture about planar graphs.
		\end{enumerate}
	\item References: \vspace{-0.2cm}
		\begin{enumerate} \itemsep -2pt
		\item \url{https://en.wikipedia.org/wiki/Graph_theory}
		\item {\bf Homeomorphism}: \url{https://en.wikipedia.org/wiki/Homeomorphism_(graph_theory)#Subdivision_and_smoothing}
		\end{enumerate}
	\end{enumerate}
\item common topics: \vspace{-0.3cm}
	\begin{enumerate} \itemsep -2pt
	\item {\bf Planar graph} \cite{WikipediaContributors2018a43}: %\vspace{-0.2cm}
		\begin{enumerate} %\itemsep -2pt
		\item \url{https://en.wikipedia.org/wiki/Planar_graph}
		\item Are all planar graphs sparse graphs? \cite{WikipediaContributors2018a43}
		\end{enumerate}
	\item {\bf network science}: \vspace{-0.2cm}
		\begin{itemize} \itemsep -2pt
		\item \url{https://en.wikipedia.org/wiki/Network_science}
		\item {\bf Interdependent networks}: \url{https://en.wikipedia.org/wiki/Interdependent_networks}
		\item {\bf Modularity} (networks): \url{https://en.wikipedia.org/wiki/Modularity_(networks)}
		\item {\bf Community structure}: \url{https://en.wikipedia.org/wiki/Community_structure}
		\item {\bf Distance} (graph theory): \url{https://en.wikipedia.org/wiki/Distance_(graph_theory)}
		\item {\bf Assortativity}: \url{https://en.wikipedia.org/wiki/Assortativity}
		\item {\bf Degree distribution}: \url{https://en.wikipedia.org/wiki/Degree_distribution}
		\item {\bf Centrality}: \url{https://en.wikipedia.org/wiki/Centrality#Closeness_centrality}
		\item {\bf Betweenness centrality}: \url{https://en.wikipedia.org/wiki/Betweenness_centrality}
		\end{itemize}
	\end{enumerate}
\end{enumerate}


Other problems and types of graphs, not of comparable importance: \vspace{-0.3cm}
\begin{enumerate} \itemsep -4pt
\item nearest neighbor graph (NNG): \url{https://en.wikipedia.org/wiki/Nearest_neighbor_graph}
\end{enumerate}






%%%%%%%%%%%%%%%%%%%%%%%%%%%%%%%%%%%%%%%%%%%
\subsection{Directed Graphs, and Directed Acyclic Graphs}
\label{ssec:DirectedGraphsAndDirectedAcyclicGraphs}


Notes on directed graphs (digraphs), and directed acyclic graphs (DAGs): \vspace{-0.3cm}
\begin{enumerate} \itemsep -4pt
\item Functions to implement, and solves to solve the following problems: \vspace{-0.3cm}
	\begin{enumerate} \itemsep -2pt
	\item For DAGs: \vspace{-0.2cm}
		\begin{enumerate} \itemsep -2pt
		\item {\bf topological sort}: \vspace{-0.1cm}
			\begin{itemize} \itemsep -1pt
			\item Also known as: {\bf topological sorting}, or {\bf topological ordering}
			\item \url{https://en.wikipedia.org/wiki/Topological_sorting}
			\end{itemize}
		\item algorithms, heuristics, and meta-heuristics to optimize: \vspace{-0.1cm}
			\begin{itemize} \itemsep -1pt
			\item BDDs; see \S\ref{sssec:BinaryDecisionDiagramsBDDs}.
			\item AIGs; see \S\ref{sssec:ANDInverterGraphsAIGs}
			\item MIGs; see \S\ref{sssec:MajorityInverterGraphsMIGs}
			\end{itemize}
		\end{enumerate}
	\end{enumerate}
\end{enumerate}











%%%%%%%%%%%%%%%%%%%%%%%%%%%%%%%%%%%%%%%%%%%
\subsubsection{Binary Decision Diagrams (BDDs)}
\label{sssec:BinaryDecisionDiagramsBDDs}


Resources for BDDs: \vspace{-0.3cm}
\begin{enumerate} \itemsep -4pt
\item \url{https://en.wikipedia.org/wiki/Binary_decision_diagram}
\end{enumerate}



%%%%%%%%%%%%%%%%%%%%%%%%%%%%%%%%%%%%%%%%%%%
\subsubsection{AND-Inverter Graphs (AIGs)}
\label{sssec:ANDInverterGraphsAIGs}




Resources for AIGs: \vspace{-0.3cm}
\begin{enumerate} \itemsep -4pt
\item \url{https://en.wikipedia.org/wiki/And-inverter_graph}
\end{enumerate}



%%%%%%%%%%%%%%%%%%%%%%%%%%%%%%%%%%%%%%%%%%%
\subsubsection{Majority-Inverter Graphs (MIGs)}
\label{sssec:MajorityInverterGraphsMIGs}



Resources for MIGs: \vspace{-0.3cm}
\begin{enumerate} \itemsep -4pt
\item 
\end{enumerate}







%%%%%%%%%%%%%%%%%%%%%%%%%%%%%%%%%%%%%%%%%%%
\subsection{Undirected Graphs}
\label{ssec:UndirectedGraphs}






%%%%%%%%%%%%%%%%%%%%%%%%%%%%%%%%%%%%%%%%%%%
\subsection{Resources for Graphs}
\label{ssec:ResourcesForGraphs}

Resources for graphs: \vspace{-0.3cm}
\begin{enumerate} \itemsep -4pt
\item ``The Stanford GraphBase: A Platform for Combinatorial Computing'': \vspace{-0.3cm}
	\begin{enumerate} \itemsep -2pt
	\item \cite{Knuth1993}
	\item \url{https://www-cs-staff.stanford.edu/~knuth/sgb.html}
	\end{enumerate}
\item {\it Wikipedia}: \vspace{-0.3cm}
	\begin{enumerate} \itemsep -2pt
	\item \url{https://en.wikipedia.org/wiki/List_of_algorithms#Graph_algorithms}
	\item {\bf Gallery of named graphs}: \url{https://en.wikipedia.org/wiki/Gallery_of_named_graphs}
	\item {\bf List of graph theory topics}: \url{https://en.wikipedia.org/wiki/List_of_graph_theory_topics}
	\item {\bf Glossary of graph theory terms}: \url{https://en.wikipedia.org/wiki/Glossary_of_graph_theory_terms}
	\item : \url{}
	\item : \url{}
	\item : \url{}
	\item : \url{}
	\item : \url{}
	\item : \url{}
	\item : \url{}
	\item : \url{}
	\item : \url{}
	\item {\bf Graph algebra}: \url{https://en.wikipedia.org/wiki/Graph_algebra}
	\end{enumerate}
\end{enumerate}











